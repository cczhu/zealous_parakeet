\chapter{Introduction}

\section{White Dwarf Mergers}
\label{sec:wdmergers}

\subsection{The Panopoly of Stellar Mergers}
\label{ssec:stellarmergers}

Approximately two out of every three stars are born into a binary system.  A substantial fraction of these stars will interact, some due to their orbital separation at birth, while others following the expansion of one or both constituent stars as they evolve off of the main sequence.  These interactions primarily take form as mass transfer between the stars \citep{yung05}, and if mass transfer becomes unstable (increases exponentially over time), it ends with the violent coalescence of the two stars into one.  These stellar mergers, like other forms of binary interaction, disrupt single star evolution and create merged products, or ``merger remnants'', with unusual properties including blue stragglers (eg. \citealt{andrpt06, knigs09}), luminous blue variables \citep{justpv14}, subdwarf OB and R Coronae Borealis (RCrB) stars.

Mergers also liberate tremendous amounts of energy and eject significant amounts of mass, giving rise to a cornucopia of electromagnetic and gravitational-wave transients ranging from luminous red novae (from the merger of two (post-) main-sequence stars; eg. V838 Monocerotis and V1309 Scorpii \citep{tyle+11, nandil14}) to short gamma-ray bursts (from two neutron stars; eg. \citealt{ross15}) and the gravitational wave outburst from coalescing stellar-mass black holes (as recently found by the LIGO detector; \citealt{ligo16}).  Indeed, with current deep and short-cadence optical/near-infrared survey projects such as the Palomar Transient Factory \citep{rau+09} and Pan-STARRS \citep{kais+10} continuing to uncover more rare and even hitherto-unknown transients, and the ambitious Large Synoptic Survey Telescope \citep{lsst09} under construction, a much more complete picture of merger-generated transients will form over the next decade.

% Sec 5.2.3 of Tylenda talks about MS - MS pre-merger orbital evolution.

\subsection{Mergers of WD Binaries}
\label{ssec:wdmergers_sub}

One common end-product of binary stellar evolution is a pair of white dwarf (WDs) in a close binary orbit.  These binaries are formed as a result of at least two phases of mass transfer (at least one of which is a common envelope event) during the binary's prior stellar evolution.  These mass transfer phases act to sap the orbital angular momentum of the 

Following their formation, these binaries can lose orbital angular momentum through a number of of mechanisms, including gravitational radiation (eg. \citealt{XXX}), magnetic braking \citep{XXX} or the influence of a third body \citep{XXX}.  {\charles TIDAL EFFECTS?}  In the absence of a magnetized wind or third body, gravitational radiation is generally thought to be the dominant driver of angular momentum loss, and has a characteristic timescale of \citep{segrcm97}

\eqbegin
\tau_{\mrm{grav}} = 5 \times 10^5 \left(\frac{a}{10^5 \mrm{km}}\right)^4 \frac{\Msun}{\Ma} \frac{\Msun}{\Md} \frac{\Msun}{\Mtot}\,\mrm{yr}.
\label{eq:c1_gravtimescale}
\eqend

\noindent From this, we see that WD binaries with orbital periods on the order of hours or less will merge within a Hubble time.  The outcome of such mergers will depend on the compositions of the WDs involved.  

For two helium WDs, a low-mass helium star might result, which would be observed as an sdOB star.  

For a helium WD merging with a carbon-oxygen one, a helium giant could form, observable as a hydrogen-deficient giant or RCrB star.  

For two carbon-oxygen WDs (CO WDs), the outcome could vary between simply a more massive WD, a carbon-burning star, an explosion, or collapse to a neutron star, depending on whether stable or unstable carbon fusion is ignited, and whether the total mass exceeds the critical mass for pycnonuclear ignition or electron captures (both close to the Chandrasekhar mass \Mch).  For mergers involving an oxygen-neon WD, the mass will always be high, and explosive demise or transmutation seems inevitable.

White dwarf (WD) binaries are common end products of binary stellar evolution.  Gravitational wave emission, magnetic braking or the influence of a third body will cause a fraction of these to merge, producing a diversity of unusual stars and electromagnetic transients.  In particular, for double carbon-oxygen (CO) WD mergers, the final outcome could be a massive and rapidly rotating WD (eg. \citealt{segrcm97}), an accretion-induced collapse into a neutron star (NS) \citep{saion85}, or a nuclear explosion that might resemble a type Ia supernova (SN Ia).

%The merger of two white dwarfs (WDs) originally in a short-period binary is estimated (eg. \citealt{badem12}) to occur about once every century in a Milky Way-like galaxy, making the products of such events common throughout the universe.  They have been held responsible for producing a variety of stars with strange properties, including helium-burning sdOB stars \citep{saioj00, justph11}, RCrB stars (eg. \citealt{webb84, clay+07, clay13}), and massive and highly magnetized WDs (eg. \citealt{segrcm97, garc+12, kule+13}) that could resemble the hot DQ WDs (eg. \citealt{dunlc15}, Dunlap and Clements in preparation).  They may, however, also be responsible for spectacular transient events including accretion-induced collapses (eg. \citealt{saion85, abdi+10}) and type Ia supernovae (SNe Ia; eg. \citealt{howe11, hill+13, maozmn14}).  Determining the final outcome of a particular merger requires an understanding of the detailed dynamics of the merging process, which cannot directly be seen using current observational capabilities.  Thus, studies of merger physics have primarily utilized hydrodynamic simulations.
