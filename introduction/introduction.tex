\chapter{Introduction}

\section{White Dwarf Mergers}
\label{sec:wdmergers}

\subsection{The Panopoly of Stellar Mergers}
\label{ssec:stellarmergers}

Approximately two out of every three stars are born into a binary system.  A substantial fraction of these stars will interact, some due to their orbital separation at birth, while others following the expansion of one or both constituent stars as they evolve off of the main sequence.  These interactions primarily take form as mass transfer between the stars \citep{yung05}, and if mass transfer becomes unstable (increases exponentially over time), it ends with the violent coalescence of the two stars into one.  These stellar mergers, like other forms of binary interaction, disrupt single star evolution and create merged products, or ``merger remnants'', with unusual properties including blue stragglers (eg. \citealt{andrpt06, knigs09}), luminous blue variables \citep{justpv14}, subdwarf OB and R Corona Borealis stars.  They also liberate tremendous amounts of energy and eject significant amounts of mass, giving rise to a cornucopia of electromagnetic and gravitational-wave transients ranging from luminous red novae (from the merger of two (post-) main-sequence stars; eg. V838 Monocerotis and V1309 Scorpii \citep{tyle+11, nandil14}) to short gamma-ray bursts (from two neutron stars; eg. \cite{ross15}) and the gravitational wave outburst from coalescing stellar-mass black holes (as recently found by the LIGO detector; \citealt{ligo16}).  Indeed, with current deep and short-cadence optical/near-infrared survey projects such as the Palomar Transient Factory \citep{rau+09} and Pan-STARRS \citep{kais+10} continuing to uncover more rare and even hitherto-unknown transients, and the ambitious Large Synoptic Survey Telescope \citep{lsst09} under construction, a much more complete picture of merger-generated transients will form over the next decade.

% Sec 5.2.3 of Tylenda talks about MS - MS pre-merger orbital evolution.

\subsection{Mergers of WD Binaries}
\label{ssec:wdmergers_sub}

One common end-product of binary stellar evolution is the merger of two white dwarfs (WDs) in a close binary orbit.  Close WD binaries are formed as a result of at least two phases of mass transfer (at least one of which is a common envelope event) during the binary's prior stellar evolution.  These mass transfer phases act to sap the orbital angular momentum of the 

%A few percent of all white dwarfs (WDs) will eventually merge with another white dwarf.  



%The outcome of such mergers will depend on the compositions of the WDs involved.  For two helium WDs, a low-mass helium star might result, which would be observed as an sdOB star.  For a helium WD merging with a carbon-oxygen one, a helium giant could form, observable as a hydrogen-deficient giant or R CrB star.  For two carbon-oxygen WDs (CO WDs), the outcome could vary between simply a more massive WD, a carbon-burning star, an explosion, or collapse to a neutron star, depending on whether stable or unstable carbon fusion is ignited, and whether the total mass exceeds the critical mass for pycnonuclear ignition or electron captures (both close to the Chandrasekhar mass \Mch).  For mergers involving an oxygen-neon WD, the mass will always be high, and explosive demise or transmutation seems inevitable.


\section{The Mystery of Type Ia Supernovae}

\subsection{Properties and Traditional Formation Channels of SNe Ia}
\label{ssec:old_typeia}

\subsection{Traditional Shortfalls}

\subsection{Brave New Channels}

\section{The \citeal{vkercj10} SN Ia Channel}

\subsection{Channel Overview and Advantages}

\subsection{But Will it Blend? (Thesis Motivation)}

\section{Physics of the WD Merging Process}

\subsection{Stable and Unstable Mass Transfer}

\subsection{Are Merging WDs Co-Rotating?}

\section{Post-Merger Evolution}

\section{Thesis Overview}
