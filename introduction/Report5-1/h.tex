\section{Brown Dwarf Mergers}

\subsection{With Brown Dwarfs}

The existence of BD-BD binary systems has been observationally well-established.  Since neither object has ever underwent steady H-fusion, common-envelope evolution cannot be used to create a close-in BD-BD system - this requires other mechanisms, such as three-body interactions, to come into play \citep{dangkr06,bearks11}.  As exactly how BD and BD systems are formed is still being debated, such a scenario cannot be ruled out \citep{bearks11}.

\cite{bearks11} has studied the merger of a BD with a 1 - 10 M$_{J}$ gas giant.  Despite the mass ratio $q$ being tiny, the density difference betwene the two objects means that a gas giant can actually approach a BD until it is tidally disrupted and forms an accretion disk, which falls onto the BD in a matter of several days.  This highly super-Eddington accretion powers the light curve.  This process is called a ``mergeburst'', and similar mergeres between main-sequence stars are covered in another section of the report.  As this process is qualitatively similar to what was described in Sec. \ref{ssec:mechanicsofwdmergers}, it might be useful in describing cold (no nuclear processing) mergers between massive and extremely light BDs.  \citeauthor{bearks11} suggests that the light curve of this mergeburst in the V-band will to first order look like a scaled-down version of other intermediate-luminosity optical transients (ILOTs), and scales the V-band light curve of V838 Mon's ILOT to the parameters of the BD to obtain \citeauthor{bearks11} Eqn. 7.  Using this same equation for a 60 M$_{J}$ - 30 M$_{J}$ BD-BD merger, we obtain a peak V-band luminosity of $10^{38}$ - $10^{39}$ erg s$^{-1}$ ($\sim$ -7 - -10 mag), and a transient timescale (from the disk's viscous timescale) of a few days.

There are several limitations with this simple scale-up from \citeauthor{bearks11} to a BD-BD merger.  For one, more equal-mass BD-BD mergers should disrupt both stars, creating a remnant hottest at its centre - subsequent disk accretion may result in enough compressional heating to ignite central burning, or the stars might even ignite during the merger.  An H-flash during accretion followed by steadier nuclear burning might also be possible in cases of unequal mass mergers (see Sec. \ref{ssec:sdbstarformationchannel} for similar evolution with He).  Lastly, any explosive event or violent accretion event should eject material, which will modify light curves.  An in-depth study of BD-BD mergers would be needed to address these issues.

ASK MARTEN: WHY NOT ALL MERGERS W/O NUCLEAR PROCESSING MERGEBURSTS?

\subsection{With White Dwarfs}

A number of BD-WD systems are known to exist in nature.  Observationally there is, however, a distinct lack of close-in MS-BD systems, a trend known as the ``brown dwarf desert'' \citep{stamw08}.  The most massive BDs should be $\sim 0.08$ {\Msun}, while the least massive known WDs are approximately double this value, suggesting that $q = 0.5$ BD-WD systems (unstable by the analysis in Sec. \ref{ssec:stabilityofmasstransfer}) may exist \citep{brow+11}.  Note there is extensive work on WDs stably accreting hydrogen-rich material, which is covered in other sections of this report.

%It is posited that a BD-WD system is the end result of cataclysmic variable evolution in a WD-MS binary, and therefore such systems should be, to first order, as common as CV binaries are - these systems are, however, stable to mass transfer \citep{}

Like with BD-BD mergers, our literature search did not find any studies of WD-BD mergers.