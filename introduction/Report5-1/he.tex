\section{Helium-Helium White Dwarf Mergers}

\subsection{SdB Star Formation Channel}
\label{ssec:sdbstarformationchannel}

%\citeauthor{guerig04} simulated the merger of two 0.4 He WDs, and found that the remnant was hottest at the centre of the core, with a temperature of $2 \times 10^8 K$.  While this is high enough to ignite He burning, is insufficient to drive a thermonuclear flash, and as a result the WD expands and cools to $1.6 \times 10^8 K$.  \citeauthor{loreig09} perform a 0.3 - 0.5 {\Msun} He-CO WD merger that results in a corona hot enough for significant He nuclear processing, though there is no nuclear flash or significant ejection of material.  We would expect that an unequal mass He-He merger would result in a similar configuration, but we cannot be sure that no explosive He detonation would occur during the merger process.

\cite{saioj00} simulated He accretion at $10^{-5}$ {\Msun} yr$^{-1}$ onto 0.3 and 0.4 {\Msun} He WDs, which they stated was a rough approximation of the He-He merging process.  From Sec. \ref{ssec:mechanicsofwdmergers} it is obvious this is not the case for the initial merger, but such a situation could approximate the accretion of the disk onto the remnant core.  Accretion was halted when the WD reached a predetermined mass (between 0.5 {\Msun} and twice the initial mass).  After a certain mass had been accreted, a He flash results, with a nuclear luminosity on order of $10^{39}$ to $10^{41}$ erg s$^{-1}$.  This energy, rather than forming a transient, trickles into the envelope and increases the radius and luminosity of the star, which eventually forms a yellow giant after $\sim 10^3$ years.  Mass transfer stops during the yellow giant phase, and as the He shell burning continues to propagate inward, the merger remnant evolves (over $10^6$ yr) toward the helium main sequence, with additional periodic He shell flashes.  The models considered by \citeauthor{saioj00} pass, on a log g vs. log T$_{\mathrm{eff}}$, near the observed position of low-luminosity extreme helium star (EHe) V652 Her, and the models can also reproduce the pulsation and period change, as well as the surface composition, of V652 Her.  The star eventually becomes a hot subdwarf B (sdB) star \citep{saioj00}.  \cite{han+02} obtains similar results, and \cite{han+03} give the formation rate of sdB stars from mergers, cited below.  Follow-up work published in \cite{saioj02} indicate that He WD mergers combined with heavier CO WDs might account for a much larger fraction of EHes.

It is worth noting that a post-merger remnant He WD would have very different temperatures and mass distributions compared to a cold WD evolved from a single star.  It would therefore be worthwhile to repeat such a study using initial conditions explicitly determined from merger simulations.

\subsection{Possible Explosive Nuclear Burning?}
\label{ssec:possibleexplosivenuclearburning}

An accretion rate of $10^{-5}$ {\Msun} yr$^{-1}$ is approximately half the Eddington limit \citep{saioj00}.  \cite{loreig09} and \cite{vkercj10} both note that the accretion rate could be enormous, which may suggest other evolutionary channels such as a violent core or off-centre He detonation through compressional heating (\textit{a la} \citeauthor{vkercj10}'s work on CO WD mergers; see Sec. \ref{sssec:sub-chandrasekharmasscodetonations}).  

If this were the case, then we would expect that the (calculated) rate of He-He WD mergers that ignite He, which we shall refer to as the ``formation rate from mergers'', is much higher than the birth rate of single sdB stars.  From Monte Carlo studies of binary evolution, \citeauthor{han+03} gives the rate of sdB formation at $\sim 0.014$ - 0.063 yr$^{-1}$, and the formation rate from mergers at $0.003$ - $0.017$ yr$^{-1}$.  Observationally the birth rate of sdBs is $\sim 0.01$ yr$^{-1}$, and simulations show the binary fraction for sdBs is around 75\% - 90\%, giving a value of single sdB star formation roughly consistent with the formation rate from mergers \citep{nele+01,han+03,hebe09}.  

There are, however, multiple methods, both through binary (with main-sequence and sub-stellar companions rather than WDs) and single star evolution, of potentially producing an sdB star \citep{hebe09,nele10}.  \cite{nele10}'s population synthesis study shows that more than 90\% of all He WD mergers should be massive enough to ignite He, meaning that many more sdBs are created than single He WDs from mergers.  Common-envelope evolution with a sub-stellar companion, on the other hand, produces several times more single He WDs than sdBs.  The rate of single He WD formation can be estimated from observations to be $7.5 \times 10^{-14}$ pc$^{-3}$yr$^{-1}$ ($\times \sim 5 \times 10^{11}$ for birth rate in the Milky Way), while the birth rate of single sdBs is less than $2 \times 10^{-14}$ pc$^{-3}$yr$^{-1}$.  This suggests that common-envelope evolution with sub-stellar companions creates the bulk of single He WDs.  \cite{nele10} find that if they assume all single He WDs are formed by common-envelope evolution with sub-stellar companions, enough sdBs are also created to roughly match observed sdB birth rates.  This argument appears to leave room for some He-He WD mergers to explode instead of forming sdBs.  Further problems, however, arise from the fact that there are also multiple paths to forming not only sdBs but also single He WDs.  Details in binary evolution model parameters such as common envelope ejection efficiency vary between papers, which must be accounted for.

It is quite obvious that due to multiple formation channels it is not possible to tell if there is a descrepancy between the rate of He-He mergers and the rate of sdB star formation from such mergers.  Better constrained statistics on merger and formation rates may make such a thought experiment viable in the future.

%From Monte Carlo studies of binary evolution, \citeauthor{han+03} gives the rate of sdB formation at $\sim 0.014 � 0.063$ yr$^{-1}$, and the formation rate from mergers at $0.003 - 0.017$ yr$^{-1}$.  The binary fraction for sdBs is around 75\% - 90\% \citep{nele+01,han+03,hebe09}.  Observationally the birth rate of sdBs is $\sim 0.01$ yr$^{-1}$, higher still if observational selection effects are considered, consistent with simulations.  The formation rate from mergers is also consistent with the numbers obtained by \cite{han98}.

%On the other hand, \cite{nele10} performs another survey and determines that more than 90\% of all He WD mergers should be massive enough to ignite He, while an alternative channel for creating sdB stars, interaction with sub-stellar companions (planets and brown dwarfs) creating about 5 times more He WDs than sdB stars \footnote{As the binary fraction of RGB/planets, and the mass distribution of planets orbiting giants is not well constrained, \citeauthor{nele10} made a number of simplifying assumptions on the population of sub-stellar companions.  The ratio cited is therefore a rough estimate.}.  The rate of single He WD formation can be estimated from observations to be $7.5 \times 10^{-14}$ pc$^{-3}$yr$^{-1}$, while the birth rate of single sdBs is $2 \times 10^{-14}$ pc$^{-3}$yr$^{-1}$.  This suggests sub-stellar companion interaction as the primary formation mechanism for both He WDs and sdBs.  Moreover, if we take the effective volume of the galaxy to be $5 \times 10^{11}$ pc$^3$, we obtain a single He formation rate of 0.037 yr$^{-1}$, an order of magnitude larger than the $5.7 \times 10^{-3}$ yr$^{-1}$ from \citeauthor{han98}.  It is the case, however, that various other mechanisms also exist for creating both types of stars, such as evolution from AM CVn systems, or even SN Ia ejecting companions \citep{nele10}.  Moreover, all these estimates depend sensitively on the number of single sdBs, and sdBs with wide, faint binary companions may be classified as single sdBs \citep{nele10}.
