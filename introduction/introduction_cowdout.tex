\section{The Mystery of Type Ia Supernovae}
\label{sec:c1_mysteryofsneia}

%It was only in the 20th century, however, that the they were recognized, through photometric and spectroscopic observations, as being a phenomena separate from novae \citep{baadz34}, and supernovae caused by the core-collapse of massive stars \citep{mink41, elias}.

SNe Ia have been observed by astronomers for centuries.  SN 1572, for example, was observed by Danish astronomer Tycho Brahe to be ``far beyond the Moon'', and helped lead to the abandonment of the Aristotelian concept that the heavens were immutable (eg. \citealt{clars77}).

% "among the most optically luminous": 2011BASI...39..375K

Today, SNe Ia are classified (eg. \citealt{fili97, li+11}) by the lack of H and He, as well as the strong presence of Si II, in their spectra.  They comprise $24$\% of all supernovae in the local universe \citep{li+11}, but are spotted disproportionately often by surveys because they are among those that are the most optically luminous \citep{howe11}.  ``Normal'' SNe Ia \citep{bran98, bran+06} typically reach a maximum bolometric luminosity of $\sim10^{43}\,\ergpsec$ after 18 - 20 days, followed by an order-of-magnitude decline in brightness over a month, and finally a slower, exponential decline of a factor of $\sim2.5$ every month due to the decreasing rate of heating by radioactive decay within the ejecta (eg. \citealt{fili97, hill+13}).  Their spectroscopic features show they are composed of a combination of intermediate-mass elements such as Si and Ca, and peak-iron elements such as Fe and Ni \citep{arne96, fili97}.  They seed both these elements and their kinetic energy into the interstellar medium, and so play an important role in star formation and galactic chemical evolution \citep{maozmn14}.

% Not going to be specific about the fractions of certain elements that come from SNe Ia - see 2010MNRAS.402..173A and its errata (2012MNRAS.424..800A) for why it's difficult to assess.

Normal SNe Ia are remarkably homogeneous, and exhibit variations that can -- to first order -- be parameterized by a single variable \citep{hilln00, howe11}.  This is reflected most famously in the \cite{phil93} relation, where SNe Ia with greater peak brightnesses tend to evolve more slowly in time.  Secondary correlations also exist, including the ``color-luminosity'' relation where SNe Ia that reach lower peak brightnesses have redder colors around then \citep{ries+96}.  Their parameterizability, as well as their intrinsic brightness, make SNe Ia outstanding cosmological distance indicators.  They were most famously used in this context in the much-celebrated discovery of \cite{ries+98} and \cite{perl+99} that the expansion of the universe is accelerating under the influence of a ``dark energy'', the exact nature of which remains mysterious.

% Hill+13's fig 1 has a good plot of peculiar SNe Ia brightness and decay times

Up to $30$\% \citep{li+11} of explosions classified as SNe Ia are ``peculiar'' SNe Ia that are an order of magnitude fainter (eg. SN 2002cx \citep{li+02, fole+13} and SN 1991bg \citep{mazz+97}) or substantially brighter (eg. SN 2009dc; \citealt{yama+09, taub+09}) than normal ones.  These do not adhere to normal SNe Ia correlations above, and are generally believed to have different progenitors or explosion mechanisms from normal SNe Ia.  A number of these progenitors have already been discussed in Sec. \ref{sec:c1_mergeroutcomes}, and several more are mentioned in Sec. \ref{ssec:c1_new_typeia}.

Despite their ubiquity and utility, however, the exact nature (or natures) of the progenitors to normal SNe Ia remains mysterious.  They were first proposed to be explosions of CO WDs by \cite{hoylf60} based on the composition of SNe Ia ejecta.  This is now well-established by the similarity of the light curve, energetics and spectral evolution of a typical SN Ia to those calculated for an exploding CO WD.  Also, early-time observations of the recent SN 2011fe have constrained the radius of the exploding object to be $\lesssim0.1\,\Rsun$ \citep{nuge+11, bloo+12,maozmn14}, consistent with a CO WD, while late-time observations of SN 2014J have detected gamma-ray emission from the decay of \Ni\ \citep{chur+14}, produced by the burning of carbon and oxygen to nuclear statistical equilibrium.  What is much less well-understood is how the CO WD is made to explode, leading to a vast body of literature exploring the various theoretical and observational lines of evidence.  The references below are necessarily only a subsample of this literature; see \cite{howe11}, \cite{hill+13}, \cite{maozmn14}, and \cite{tsebs15} for excellent reviews and further references.

\subsection{Traditional Formation Scenarios and Their Pitfalls}
\label{ssec:c1_old_typeia}

Until recently, the most widely accepted progenitor scenarios have involved pushing a CO WD to carbon ignition by slowly adding mass to it \citep{hilln00}.  The added mass compresses and heats the WD's interior, but the latter is at least partially balanced by cooling from neutrino emissions, which prevents carbon ignition due to high temperatures.  As the CO WD approaches \Mch\ and its central density approaches $\sim2\times10^9\,\gcc$, the rate of heating from pycnonuclear carbon fusion -- i.e. carbon fusion due to extreme density -- starts to exceed that for neutrino cooling.  Because ignition occurs under highly degenerate conditions, the WD does not respond to this heating by expanding, and so, unlike a non-degenerate star, experiences a nuclear runaway.  This lasts $\sim1000\,\mrm{yr}$, until the timescale for nuclear heating at the WD's center becomes shorter than the star's dynamical time.  Dynamical burning then begins, and some kind of explosion is inevitable.

The various scenarios to get a CO WD to accrete slowly can be subdivided into two classes, or ``channels'': the single-degenerate (SD) channel \citep{wheli73}, where the WD steadily accretes from a non-degenerate companion (a main sequence star, a giant, or an sdOB star), and the double-degenerate (DD) channel \citep{ibent84, webb84}, where two CO WDs with a total mass $\gtrsim\Mch$ merge, producing a merger remnant composed of a dense, degenerate ``core'' surrounded by a thick accretion disk.  Both scenarios are beset by two issues, which we summarize below using arguments from \citeauthor{vkercj10} (\citeyear{vkercj10}, henceforth \citeal{vkercj10}; see also \citealt{vker13}).

The first issue is that in order to match the observed SN Ia rate of $\sim 0.0023 \pm 0.0006$ for every solar mass of stars formed \citep{maoz+11}, $\sim1$\% of all WDs formed (of any composition and regardless of binarity) must produce SNe Ia.  Compared to this relatively large number, there is an apparent paucity of CO WDs that can reach \Mch\ from either channel.  In hydrogen-accreting SD systems, efficient growth of the WD appears only achievable if the accretion rate is between $10^{-8} - 10^{-7}\,\Msun\,\pyr$.  Slower accretion results in nova outbursts that eject the accreted mass (\citealt{townsb04}; though see \citealt{zorosg11}), while faster accretion builds up an extended envelope that eventually engulfs the donor \citep{ibent84}.  Systems that do accrete at the correct rate -- and steadily burn hydrogen to helium -- should radiate supersoft x-rays, but observations of galactic x-ray flux suggest a factor of $10 - 100$ too few of these systems exist to explain the SNe Ia rate \citep{dist10, gilfb10}, and whether or not these systems can be ``hidden'' from view as rapidly-accreting enshrouded WDs is debatable (eg. \citealt{hachkn10, lepok13, joha+14}).  Even if the accreted matter has been burned to helium, or the donor is He-rich, matter may still be ejected by subsequent helium flashes (\citealt{idanss13, pier+14}; though see \citealt{hill+16}).

% Note that He-accretion leads to SSS as well (http://adsabs.harvard.edu/abs/2007ApJ...660.1444S)

Meanwhile, analytical estimates (\citeal{vkercj10}), binary population syntheses (which simulate the evolution of a population of binaries from the zero-age main sequence onward; \citealt{menn+10, ruitbf09, toonnp12, clae+14}) and empirical counting of candidate systems \citep{badem12} all estimate that the merger rate of CO - CO WD binaries with total mass greater than $\sim\Mch$ falls short of the SN Ia rate by a factor of at least a few.  Not all of these mergers will necessarily end as SNe Ia, either: if post-merger evolution leads to off-center carbon ignition in the merger remnant, carbon burning will transform the remnant into an ONe WD, and an $\gtrsim\Mch$ ONe WD likely ends its life in an AIC, rather than exploding as an SN Ia \citep{nomoi85, saion85, yoonpr07, schw+16}.

% The 0.3 - 0.9 Msun comes from Hill+13, but they quote Stri+06
 
The second issue is the difficulty for the thermonuclear explosion of an \Mch\ mass CO WD to replicate the properties of normal SNe Ia.  Normal SNe Ia synthesize $\sim0.3 - 0.9\,\Msun$ of radioactive \Ni, peaking at $\sim0.6\,\Msun$ (eg. \citealt{stri+06, pirotk14}), and feature ejecta that are compositionally stratified, with intermediate-mass elements sitting above the peak-iron ones \citep{howe11, hill+13}.  If dynamical burning leads to an extended region of high overpressure, a detonation occurs: a supersonic shockwave drives through the WD and triggers nuclear fusion in its wake (eg. \citealt{seit+09}).  As most of the mass in an \Mch\ WD is $\gtrsim10^9\,\gcc$, nuclear burning would convert almost all of the WD to \Ni\ \citep{howe11, hill+13}.  On the other hand, if the explosion propagates as a subsonic deflagration, where a steep temperature gradient -- a flame front -- moves outward via conduction, the WD is able to expand during the explosion.  At lower densities, intermediate-mass elements are produced.  The explosion, however, produces slower velocity ejecta than seen in SNe Ia, and mixes burned and unburned material such that the ejecta would not appear stratified.  To resolve this, an \textit{ad-hoc} deflagration-to-detonation transition (DDT) is often invoked \citep{khok91}, the timing of which can be tuned to vary the amount of \Ni\ generated (eg. \citealt{hill+13}), though it remains unclear if this is a robust mechanism that produces realistic WD explosions (eg. \citealt{fishj15}).  It is also not obvious how invoking the DDT can explain the dependence of observed SNe Ia on the properties of their host galaxies, for example why more luminous SNe Ia tend to be in star-forming galaxies (eg. \citealt{hamu+00, sull+10}).

The SD channel has a number of additional complications \citep{maozmn14, tsebs15} that have led it to decline in popularity compared to the DD one.  For example, it requires a non-degenerate companion, which, under certain conditions, might be detectable, but attempts to spot the companion in pre-explosion archival data \citep{li+11cpn, nielvn13, niel+14}, during the supernova (as it responds to being hit by SN ejecta; \citealt{bloo+12,ollms15}), or after the explosion (eg. \citealt{kerz+14rem}) have come up short.  For SD scenarios involving hydrogen-rich donors, the explosion is also expected to strip and entrain donor material, but attempts to find such material either do not detect hydrogen, or, in one recent case, apparently too little hydrogen to be consistent with SD donor stars \citep{magu+16}.  

%While these issues (and several others; see \citealt{maozmn14}) have led the DD channel to fall into favor compared to the SD one, the issues discussed above affect \Mch\ progenitor scenarios in general.

% Mention gravitationally and pulsationally confined detonations??  Modern DDT: 2013MNRAS.429.1156S  GCD: http://adsabs.harvard.edu/abs/2016arXiv160600089S http://adsabs.harvard.edu/abs/2016ApJ...819..132G

\subsection{Brave New Channels}
\label{ssec:c1_new_typeia}

%In the following section, we will argue that the Chandrasekhar scenario shows significant problems when compared to observations of SNe Ia, and suggest that the majority of SNe Ia are caused by mergers of CO WD binaries, including those that produce remnants with masses below \Mch.  This argument is a summarization of the argument in \cite{vankerkwijk}.

The challenges posed by the evidence above have spurred research into alternative scenarios that lead to exploding CO WDs that are more physically viable and better fit observations.  Notably, several of these channels relax the condition that the exploding WD must be at or above \Mch, allowing for a wider range of exploding masses.  The alternate scenarios include:

\begin{itemize}

%($\lesssim10^{-6}\,\Msun\,\pyr$; \citealt{bild+07}) 

	\item The {\bf double-detonation channel} \citep{livn90, woosw94}, which involves a CO WD accreting a thin envelope from an He-rich source -- a He star, merger with an He/hybrid WD, or even the thin He envelope of a CO WD.  At some point -- for slow accretion, when the base of the envelope becomes hot enough to ignite dynamical He burning \citep{woosk11}; for mergers, when a hotspot in the accretion stream or envelope reaches conditions for detonation \citep{guil+10, rask+12, pakm+13} -- the He shell detonates.  This then either drives compressional waves that converge at the core of the CO WD, or directly launches a detonation shock into the CO WD, both of which trigger the secondary detonation of the CO WD (though the former is more plausible; \citealt{mollw13}).  

% 0.03 and 0.8 below are from Shen+14

Initially, double-detonations were believed to need massive He shells of $\gtrsim0.1\,\Msun$, but these produce significant amounts of \Ni\ in the explosion, inconsistent with the stratified nature of SNe Ia ejecta (eg. \citealt{krom+10,woosk11}).  While the robustness of the double-detonation remains a field of active research (eg. \citealt{woosk11, holc+13, shenm14, shenb14, dan+15}), more recent work suggests thin, $\lesssim0.03\,\Msun$ He shells may detonate for both slow accretion and mergers onto $\gtrsim0.8\,\Msun$ CO WDs \citep{woosk11, pakm+13, shenm14}, and that a CO detonation via converging shockwaves is likely as a result \citep{fink+10, mollw13, shenb14}.  Pure detonations of bare sub-\Mch\ CO WDs with masses between $\sim1-1.15\,\Msun$ show light curves and spectra in good agreement with normal SNe Ia \citep{shig+92, sim+10}, while the detonation of the much less massive He shell greatly reduces (but does not entirely eliminate) contamination from He shell nuclear ashes \citep{krom+10, hill+13}.

%Studies of pure detonations of bare sub-\Mch\ CO WDs with masses between $\sim1-1.15\,\Msun$ (eg. \citep{shig+92, sim+10}) show light curves and spectra that are in good agreement with normal SNe Ia (arguably as good or better than the agreement for other explosion models; \citealt{ropk+11})

If the double-detonation channel is indeed physically plausible, it opens up a range of WD masses beyond \Mch\ that can explode, and more naturally explains the explosion.  In particular, since peak luminosity of the explosion is dependent on the mass of the CO WD (since more \Ni\ is generated in the detonation of massive WDs), this could explain both the \cite{phil93} relation and the relationship between SN Ia luminosity and the age of its host stellar population (since lower-mass WDs take longer to form).  The problem remains that to generate typical SN Ia \Ni\ yields of $\sim0.6\,\Msun$, the CO WD needs to be $\sim1.1\,\Msun$, far above the mass of typical field WDs \citep{pirotk14}.  For both slowly accreting and merging systems, it might be possible to grow CO WDs by several $0.1\,\Msun$ through He accretion in a past phase of binary evolution \citep{ruit+11, ruit+13, ruit+14}, but the viability of these scenarios are sensitive to how mass transfer is treated (eg. the He retention efficiency).  For slow accretion, current calculations also favor detonating CO WD masses closer to $\sim0.8\,\Msun$ than $1.1\,\Msun$ \citep{ruit+11, ruit+14}, which would produce highly underluminous supernovae.

%; \citealt{sim+10} also states the luminosity/age relationship, but it's probably a trivial statement
%, though the mass ratio criterion is reduced at higher primary masses \citep{sato+16}.

	\item The {\bf violent merger channel} \citep{pakm+10}, a variant of the double-degenerate channel where, during the merging process, material being accreted from one WD to another is sufficiently superheated and compressed to trigger a detonation, sidestepping the need for accretion following the merger.  Only those binaries where the primary WD is $\gtrsim0.9\,\Msun$ and the WD mass ratio is $\gtrsim0.8-0.9$ (\citealt{pakm+10, pakm+11, sato+16}; \citealt{dan+12} suggests a higher minimum primary WD of $\gtrsim1.0\,\Msun$) are violent enough to trigger a detonation.  Mergers with primaries of $\gtrsim1.0\,\Msun$ produce explosions consistent with normal or overluminous SNe Ia \citep{pakm+12, moll+14}.  Lower-mass systems produce ones that only synthesize $\sim0.1\,\Msun$ of \Ni, are much redder and have much slower velocities than typical SNe Ia, but could resemble the subluminous SN 1991bg-like subclass \citep{lieb+93, pakm+10}.  Whether a CO detonation can robustly be triggered during a merger requires further study, but regardless, the CO WD binaries needed are highly super-\Mch, and may be too rare to explain the majority of SNe Ia \citep{badem12}.

	\item The {\bf direct collision channel}, in which two potentially sub-\Mch\ CO WDs collide, rather than merge, either because they are in a dense stellar environment (such as a globular cluster; \citealt{benzth89, loreig10}) or a hierarchical triple system \citep{katzd12} under the influence of the Kozai-Lidov mechanism \citep{koza62, lido62}.  Hydrodynamic simulations \citep{rask+10, kush+13, garc+13} show that the impact of the WDs leads to strong shocks that plow into both WDs and produce detonations.  Explosion properties can be varied by changing the impact parameter or the masses of the stars, with the collision of two $\sim0.65\,\Msun$ CO WDs able to produce $\sim0.4\,\Msun$ of \Ni\ \citep{garc+13, kush+13}, consistent with normal SNe Ia.  A minority of WDs reside in dense stellar environments, however, and only $\sim10-20$\% of stars are in triples, making it unlikely that this channel alone can reproduce a substantial fraction of SNe Ia (\citealt{maozmn14}; see their Sec. 2.3 for details).

	\item The {\bf core-degenerate channel} \citep{livir03, kashs11, tsebs15}, which occurs when a progenitor system that would otherwise have produced a $>\Mch$ close double WD binary does \textit{not} survive its (second) common envelope phase (or survives in such a tight orbit that it merges within $10^5$ years afterward).  To account for most SNe Ia, the super-\Mch\ remnant would have to live on for $10^{8}-10^{10}\,\mrm{yr}$ before exploding (since SNe Ia can occur in stellar environments that have long ceased star formation; eg. \citealt{prichs08, maozsg10}).  \cite{illks12} propose this extended lifespan is achievable by a gradual loss of solid-body rotational support through magnetic dipole radiation, but post-merger evolution likely occurs over much smaller timescales (Sec. \ref{sec:c1_vkchannel}).  Nonetheless, mergers during or just after common-envelope events naturally explain peculiar SNe Ia that show interaction with substantial amounts of circumstellar material (eg. PTF 11kx; \citealt{dild+12, soke13}).

\end{itemize}

%More unconventional channels have also been proposed that allow a CO WD to detonate due to collisions with planets and planetoids \citep{distfg15} or even lone sub-\Mch\ WDs due to compositional impurities near their core that lower the density required for the onset of pycnonuclear fusion to $\gtrsim0.9\,\Msun$ \citep{chio+15}.  Further work is required to show the physical viability of these channels, and whether they reproduce normal SNe Ia.  

More unconventional channels have also been proposed where a CO WD detonates due to collisions with planets and planetoids \citep{distfg15} or due to compositional impurities near their core that help facilitate pycnonuclear fusion (without the need of a binary companion at all; \citealt{chio+15}).  Further work is required to show the physical viability of these channels, and whether they reproduce normal SNe Ia.  

Regardless of whether or not any of these produce normal SNe Ia, studying them may be useful for explaining the diverse subclasses of SNe Ia.  For instance, synthetic light curves and spectra of pure deflagrations of \Mch\ CO WDs \citep{phil+07, krom+13, fink+14} reproduce the low peak brightness, slow ejecta velocity and hot photospheric emission of the SN IaX subclass (or SN 2002cx-like subclass; \citealt{li+02, fole+13}).  Meanwhile, successful detonations of \Mch\ CO WDs have been linked to the overluminous SN 1991T-like SNe Ia (\citealt{fishj15}, but see \citealt{seit+16}).

%For one IaX, progenitor might have been found (2014Natur.512...54M), but no need to invoke bound remnant? (2015A&A...573A...2S)  Meanwhile, possible bound remnant found in another (2016MNRAS.tmp..977F).

\section{The Sub-Chandrasekhar CO WD Merger Channel}
\label{sec:c1_vkchannel}

Another channel was recently proposed by \citeal{vkercj10}, who hypothesize that the remnant from a double CO WD merger could subsequently become hot enough to ignite thermonuclear fusion (rather than dense enough to ignite pycnonuclear fusion).  This opens the possibility for mergers with a total mass significantly below \Mch\ to also explode.

\citeal{vkercj10} consider a fiducial merger of a $0.6-0.6\,\Msun$ CO WD binary, whose masses are chosen to be near the empirical peak of the WD mass distribution (Sec. \ref{ssec:c1_cowd_massrange}).  Hydrodynamic simulations (\citeal{loreig09}) find that the two WDs tidally destroy one another and coalesce into a remnant that is significantly heated throughout, but does not achieve temperatures sufficient to ignite fusion (as it is much less massive than the violent mergers considered in Sec. \ref{ssec:c1_new_typeia}); moreover, the remnant central density, $\sim2.5\times10^6\,\gcc$, is too low to produce \Ni\ in an explosion.  

Following coalescence, however, the remnant, which is differentially rotating, enters a period of rapid angular momentum redistribution due to hydrodynamically or magnetically-mediated viscosity.  Using the standard $\alpha$-viscosity prescription \citep{shaks73} -- i.e. the kinematic viscosity $\nu = \alpha c_s H_P$, where $c_s$ is the sound speed, $H_P$ the pressure scale height and $\alpha$ a tunable parameter -- the timescale for viscous evolution can be estimated as \citep{shen+12}

\begin{eqnarray}
t_\mrm{visc} &=& \frac{R_\mrm{disk}^2}{\nu} \sim \frac{1}{\alpha}\left(\frac{R_\mrm{disk}}{H_P}\right)^2\taudyn \nonumber \\
			&\sim& 3\times10^4\,\mrm{s}\left(\frac{10^{-2}}{\alpha}\right)\left(\frac{R_\mrm{disk}/H_P}{10}\right)^2\left(\frac{R_\mrm{disk}}{10^9\,\mrm{cm}}\right)^{3/2}\left(\frac{M_\mrm{enc}}{\Msun}\right)^{-1/2},
\end{eqnarray}

\noindent where $M_\mrm{enc}$ is the mass enclosed within the inner boundary of the disk, and we have used $\taudyn \approx H_P/c_s$ and inserted a fiducial viscosity and typical numbers for remnants.  Thus the vast majority of the remnant's angular momentum is transported away, and the remnant (including its disk) loses its rotational support against gravity over a period $\sim10^4\,\mrm{s}$.\footnote{This is in contrast to earlier work (eg. \citealt{nomoi85, yoonpr07}) that assume any rotationally-supported material will slowly accrete onto the dense core of the remnant at a near-Eddington rate of $\dot{M} \sim 10^{-5}\,\Msun\,\pyr$.  Remnants are prone to magnetic instability \citep{shen+12,ji+13}, and will almost certainly evolve over the much shorter timescale given by the $\alpha$-viscosity estimate.}  This loss of rotational support combined with increasing weight from newly accreted disk material leads to compression and heating of the remnant core.  Since $\sim10^4\,\mrm{s}$ is far shorter than the thermal adjustment timescale of $\sim10^4\,\mrm{yr}$ \citep{shen+12}, compressional heating is adiabatic, and \citeal{vkercj10} estimates that for the $0.6-0.6\,\Msun$ remnant it leads the central density and temperature to increase to $\gtrsim1.5\times10^7\,\gcc$ and $\gtrsim10^9\,\mrm{K}$, respectively, at which point the nuclear fusion timescale is smaller than even the compressional heating timescale, and a carbon nuclear runaway becomes inevitable.  If the runaway leads to dynamical burning and an explosion occurs, the generally lower densities of merger remnants compared to \Mch\ WDs means that burning leads to a larger pressure differential between the ashes and their surroundings, perhaps making a detonation favorable over a deflagration (e.g. \citealt{mazumw77}; \citealt{seit+09}).

The sub-\Mch\ merger scenario features a number of advantages over the traditional \Mch\ double-degenerate channel.  Like the double-detonation channel, this one substantially increases the number of binary systems that could potentially explode -- perhaps by a factor of $\sim3$, which would be more consistent with the SN Ia rate (\citeal{vkercj10}, \citealt{badem12}) -- and naturally explains both the explosion mechanism and some of the trends seen in the observed SN Ia population, thus alleviating the issues discussed in Sec. \ref{ssec:c1_old_typeia}.  Unlike the double-detonation channel, however, a He detonation is not invoked to trigger the CO WD to explode, negating the complications of the He detonation ashes on the SN light curve, and the merger process provides a more natural means of obtaining the $\gtrsim1\,\Msun$ of CO needed to synthesize the \Ni\ found in a typical SN Ia.

This scenario requires that carbon ignition occur under highly degenerate conditions close to the center of the merger remnant, which, in turn, requires the merger remnant be heated throughout its interior, which is the case for mergers of nearly equal-mass binaries.  When very unequal masses merge, however, simulations show the less massive secondary WD is completely disrupted and forms a disk and envelope around the largely undisturbed primary (\citeal{loreig09}).  Compression of this remnant will likely trigger off-center rather than central ignition, and could lead to stable carbon burning rather than an explosion (eg. \citealt{yoonpr07, shen+12}).

\section{Massive Merger Remnants}
\label{sec:c1_hotdqs}

Below some mass, the remnant of a double CO WD merger will almost certainly not be destroyed in an explosion, and will live on as a single, massive, rapidly rotating and likely magnetized object.  If off-center carbon burning is ignited, the WD will be transmuted into an ONe WD over $\sim10^4\,\mrm{yr}$ \citep{nomoi85, shen+12, schw+16} but will not collapse into a neutron star, since its mass is below \Mch.\footnote{\cite{schwqb15} find the critical mass needed to trigger an ONe WD to collapse is $1.38\,\Msun$, nearly at \Mch.}  Discovering the products of these mergers, therefore, is an indirect means of investigating SN Ia progenitors \citep{dunlc15}.

%The mass of a WD can be observationally determined by combining spectroscopically determined effective temperature $T_\mrm{eff}$ and logarithmic gravitational acceleration $log(g)$ with a theoretical WD mass-radius relation (eg. \citealt{klei+13}).  \cite{klei+13} and \cite{trem+16} determine the masses of single non-magnetic DA WDs from the the 

%non-magnetic (DA spectral type) CO WDs 

A number of studies of the observed mass distribution of field WDs from sky surveys (eg. \citealt{liebbh05, giambd12, klei+13, reba+15a, reba+15b}) note a high-mass peak near $1\,\Msun$ that is substantially offset from the global peak of the distribution at $\sim0.65\,\Msun$ (Sec. \ref{ssec:c1_cowd_massrange}), which could be evidence for a population of merger remnants.  This interpretation is controversial, as population synthesis studies \citep{reba+15a, trem+16} give conflicting results for whether the peak can be explained by single star evolution.  However, they are unlikely to specifically be double WD merger remnants: theoretical and observational estimates of the double WD merger rate (eg. \citealt{badem12, toonnp12}) indicate it is an order of magnitude too low \citep{trem+16} to contribute the majority of systems in the high-mass peak, and a kinematic study of objects in the peak conclude they are kinematically young \citep{weggp12}.  The excess of massive WDs may alternatively be formed in common-envelope mergers akin to the core-degenerate scenario, which better fits the evidence above \citep{reba+15b, brig+15}.

%; and the peak disappears when only examining brighter WDs, suggesting the population has not been reheated by the merging process \citep{reba+15b}

%and thus more likely to come from single, massive stars
% Double check this later, but I'm pretty sure the peak remains even if the sample is cleaned of magnetic WDs

Studies targeting isolated ``high-field magnetic WDs'' (HFMWDs; \citealt{kepl+13, garc+16}), which possess fields of $B \gtrsim10^6\,\mrm{G}$, suggest that population's mass distribution has a mean of $\sim0.8\,\Msun$, and overall does not resemble the non-magnetic WD one.  While this may suggest a double WD merger origin for these WDs, it is also possible the fields are a vestige of single-star evolution (eg. \citealt{wickf05, kisst15}) or mergers within common envelopes \citep{garc+12, wicktf14, brig+15}.  HFMWDs have not been found in detached binaries with M and K dwarfs, despite such binarity being ubiquitous for non-magnetic WDs, which is strong evidence that HFMWDs are products of some kind of merger \citep{lieb+15, ferrdg15}.  Population synthesis calculations \citep{garc+12, brig+15}, however, find that common envelope mergers produce the overwhelming majority of HFMWDs, especially ones closer to $\sim0.8\,\Msun$, with double CO WD mergers contributing only a minority of objects in the high-mass ($\gtrsim1.1\,\Msun$) tail of the distribution.

% So this is some self-analysis: http://adsabs.harvard.edu/abs/2010MNRAS.402.1072P says that CE events don't necessarily penetrate the accretor, and don't lead to strong magnetic fields in the WD product.  However, if we look at garc+12 or solely degenerate core mergers with common envelopes, their birth rates are still an order of magnitude higher.

%The most massive ($\gtrsim1.1\,\Msun$) and most highly magnetized ($\gtrsim10^9\,\mrm{G}$) HFMWDs are more likely to be WD merger remnants \citep{garc+12, wicktf14}.

%, and SDSS J085523.87+164059.0, with a mass of $\sim1.1\,\Msun$, unknown spin, and a field of $\sim10^7\,\mrm{G}$

Among the most massive HFMWDs is RE J 0317-853 \citep{bars+95, kube+10}, a $\gtrsim1.3\,\Msun$ CO or ONe WD with a high effective temperature of $\gtrsim3\times10^4\,\mrm{K}$, a spin period of $\sim700\,\mrm{s}$ and a magnetic field strength of $B\sim10^8\,\mrm{G}$ that is often cited as a double WD merger remnant.  Another such object is PG 1658+441, with a mass of $\sim1.3\,\Msun$, spin period of $\sim6\,\mrm{hr}$ and field of $\sim10^6\,\mrm{G}$ (\citealt{ferrdg15}, and references therein).  There are observational issues with attributing these objects to WD mergers.  RE J 0317 is in a wide (non-interacting) binary with a $\sim0.8\,\Msun$ companion, and \cite{kube+10} showed that the cooling age of both WDs is approximately the same.  This, they claim, makes it unlikely for RE J 0317 to have formed either from single star evolution (since it should have become a WD before its companion) or a merger (since one of its two merger constituents must be $\lesssim0.7\,\Msun$ and should have become a WD after the companion).  It is also curious that RE J 0317, PG 1658 and most other $>1\,\Msun$ HFMWDs (see the list in \citealt{ferrdg15}) are DA WDs and have spectra with absorption lines of hydrogen, which is unlikely to survive mergers due to the high temperatures achieved in them.  Indeed, these issues might indicate a common envelope merger origin for the WDs (which would naturally leave residual hydrogen and potentially yield an older cooling age than a double WD merger) consistent with the HFMWD population synthesis above.

% I might get in trouble for the sentence above - but here we go.  Kulebi et al 2010 (2010A&A...524A..36K) notes the issue is that there's no mass loss in the merger, so you have to have lower-mass initial binary constituents.  The common envelope between two higher-mass constituents might interrupt the growth of the CO WDs, truncating them at lower mass, and eject all remaining mass in a nebula.  That nebula would (just like a PN) go away within 10^7 years.

Possibly better candidates for merger remnants are the hot DQs, a recently-discovered class of WDs that features effective temperatures of $\gtrsim2\times10^4\,\mrm{K}$ and spectra with C and O absorption lines, suggesting that hot DQ atmospheres are dominated by CO \citep{dufo+07, dufo+08}.  The CO-rich atmospheres make mass-determination difficult, but preliminary estimates on a few hot DQs using either more advanced WD atmosphere models \citep{dufo+11} or combining radius (from parallax) with the WD mass-radius relation \citep{dunl15thesis} suggest they have masses around $1\,\Msun$.  They also generally appear to have monoperiodic photometric variability, possibly due to rapid rotation on the order of minutes \citep{lawr+13, will+16}, and magnetic fields $\gtrsim10^6\,\mrm{G}$ \citep{dufo+13}.  \cite{dunlc15} found that, if most known hot DQs are assumed be near $\sim1\,\Msun$, their population's kinematic age is much older than what would be inferred from their temperatures, suggesting they have been reheated since their birth.  \cite{dunl15thesis} uses a rough estimate of the hot DQ space density, along with their lifetime (given by the difference in cooling age between the hottest and coldest hot DQ), to determine that their formation rate is of order the observed (total) WD merger rate from \cite{badem12}.  These class properties strongly suggest that hot DQs are a population of CO WD merger remnants, but an alternate hypothesis \citep{dufo+07, alth+09} claims they are the progeny of stars that experience a powerful late thermal pulse during their AGB phase.  As many of their properties are still being fleshed out, their origins remain unclear for now.

%Sloan Digital Sky Survey \citep{york+00} Data Release 7 (SDSS DR7)

\section{Thesis Overview}

%-- in particular the fraction of them that have temperatures close to carbon ignition under highly degenerate conditions

If sub-\Mch\ double CO WD mergers could either explode or leave behind isolated massive CO/ONe WDs, which fate might be preferred, and how does it depend on the masses of the two WDs?  Can these mergers, as \citeal{vkercj10} claims, serve as a novel SNe Ia progenitor channel that naturally coincides with SNe Ia rates and explosion properties?  To answer these questions, we must scrutinize the \citeal{vkercj10} sub-\Mch\ merger scenario, and understand what is the range of properties of double CO WD merger remnants, how the subsequent viscous phase proceeds, and whether carbon ignition in a remnant can lead to an explosion.  In this thesis, I shed light on some of these issues through semi-analytical calculations and hydrodynamic and magnetohydrodynamic simulations.

%Thus, the \citeal{vkercj10} channel is both attractive for its many advantages and plausible given the order-of-magnitude estimates above.  Investigating whether these estimates hold under detailed scrutiny, and to determine which, if any, systems in the CO WD binary parameter space could follow the channel, is the purpose of this PhD thesis.

In Chapter \ref{ch:ch2}, I perform simulations of merging CO WD binaries with the smoothed-particle hydrodynamics code \gasoline\ in order to characterize the range of possible merger remnant configurations.  In particular, I discern which mergers are ``similar in mass'', and produce remnants that are substantially heated and rotationally supported throughout.  As discussed in Sec. \ref{sec:c1_vkchannel}, these are potentially SN Ia progenitor candidates via the \citeal{vkercj10} channel.  I also make simple estimates of the compressional heating the remnants will experience during post-merger evolution to find which of them could eventually ignite carbon fusion.

The smoothed-particle hydrodynamics method is used by almost all simulations of WD mergers.  In Chapter \ref{ch:ch3}, I investigate whether the outcome of a merger simulation depends on the hydrodynamic scheme used by performing a merger of a $0.625 - 0.65\,\Msun$ CO WD binary using either \gasoline\ or the moving-mesh magnetohydrodynamics (MHD) code \arepo, and discover phenomena during the earliest phase of post-merger evolution that are unique to \arepo.  I also detail work that led to a number of improvements in \arepo\ to prevent spurious loss of global angular momentum when simulating differentially rotating systems.

In Chapter \ref{ch:ch4}, I use \arepo\ to simulate the evolution of initially weak magnetic fields within the $0.625 - 0.65\,\Msun$ merger.  I find exponential field growth during the coalescence of the WDs that leads to a powerful, $>10^{10}$ G magnetic field in the merger remnant that could affect its subsequent evolution.

Lastly, in Chapter \ref{ch:ch5}, I consider the evolution of idealized and quasi-hydrostatic sub-\Mch\ CO WDs experiencing a nuclear runaway in their centers.  While, as stated in Sec. \ref{ssec:c1_old_typeia}, such a runaway inevitably leads to some form of explosion for \Mch\ WDs, in sub-\Mch\ ones it may instead lead to the lifting of degeneracy and expansion into a carbon-burning star.  I use analytical arguments and simple models to determine the minimum mass required for the runaway to produce an explosion, and discuss the consequences this has on merger remnants after their viscous evolution.

Chapters \ref{ch:ch2} and \ref{ch:ch4} have been published as \cite{zhu+13} and \cite{zhu+15}, respectively, and Chapter \ref{ch:ch5} is being submitted for publication simultaneous to the submission of this thesis.  For the most part, I have reproduced exactly the texts from each paper, only modifying the chapter introductions to minimize repetitious discussion of WD mergers and the \citeal{vkercj10} channel.  I have added postscripts to Chapters \ref{ch:ch2} and \ref{ch:ch4} that discuss their results in light of developments after their publication, and included an extensive appendix to Chapter \ref{ch:ch5} that details our calculation of convective suppression in magnetized or rotating WDs.  Any additional changes are noted at the start of each chapter.

