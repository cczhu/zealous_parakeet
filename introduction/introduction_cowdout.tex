\section{The Mystery of Type Ia Supernovae}

%It was only in the 20th century, however, that the they were recognized, through photometric and spectroscopic observations, as being a phenomena separate from novae \citep{baadz34}, and supernovae caused by the core-collapse of massive stars \citep{mink41, elias}.

Type Ia supernovae, or SNe Ia, have been observed by astronomers for centuries.  SN 1572, for example, was observed by Danish astronomer Tycho Brahe to be ``far beyond the Moon'', and helped lead to the abandonoment of the Aristotilian concept that the heavens were immutable \citep{ruiz04, krau+08}.

Today, SNe Ia are classified (eg. \citealt{fili97, li+11}) by the lack of hydrogen and helium, as well as the strong prescence of ionized silicon (Si II), in their spectra.  They are estimated to comprise a quarter of all supernovae in the local universe \citep{li+11}, but are found disproportionately often by surveys because most of their luminosity is emitted in the optical \citep{howe11}.  ``Normal'' SNe Ia \citep{bran98, bran+06} typically reach a maximum bolometric luminosity of $\sim10^{43}\,\ergpsec$ (see eg. \citep{fili97, hill+13}) after 18 - 20 days, followed by an order-of-magnitude decline in brightness over a month, and a slower, exponential decline of a factor of $\sim2.5$ every month (generally attributed to heat from radioactive decay within the ejecta).  Their spectroscopic features show they are composed of a combination of intermediate-mass elements (eg. \citealt{arne96}) such as silicon and calcium, and peak-iron elements such as iron and nickel \citep{fili97}; indeed they are the primary source of peak-iron elements for the galaxies that host them, and so play a crucial role in star formation and galactic chemical evolution \citep{leib00}.  Up to $\sim30$\% \citep{li+11} of explosions classified as SNe Ia are ``peculiar'' phenomena that are orders of magnitude fainter (eg. SN 2002cx \citep{li+02, fole+13}, SN 1991bg \citep{mazz+97}) and occassionally substantially brighter (eg. SN 2009dc; \citealt{yama+09, taub+09}).  Normal SNe Ia, however, are remarkably homogeneous, and exhibit variations that can -- to first order -- be parameterized by a single variable \citep{hilln00, howe11, moazns14}.  This is reflected most famously in the \cite{phil93} relation, where SNe Ia with greater peak brightnesses tend to evolve more slowly in time.  

Their parameterizability, as well as their intrinsic brightness, make SNe Ia outstanding cosmological distance indicators.  They were most famously used in this context in the much-celebrated discovery of \citep{ries+98} and \cite{perl+99} that the expansion of the universe is accelerating under the influence of a ``dark energy'', the exact nature of which remains mysterious.

Despite their ubiquity and utility, however, the exact nature (or natures) of their progenitor systems remains mysterious.  They were first proposed to be explosions of CO WDs by \citep{hoylf60} based on the composition of SNe Ia ejecta.  This is now well-established by the similarity of the light curve, energetics and spectrum of a typical SN Ia to those calculated for an exploding CO WD.  Also, early-time observations of the recent SN 2011fe have constrained the radius of the exploding object to be $\lesssim0.1\,\Rsun$ \citep{nuge+11, bloo+12, maozmn14}, consistent with a CO WD, and late-time observations of SN 2014J have detected gamma-ray emission from the decay of \Ni, produced by the burning of carbon and oxygen to nuclear statistical equilibrium, into stable $^{56}$Fe \citep{chur+14}.  What is much less well-understood is how the CO WD is made to explode, and a vast body of literature now exists exploring the various theoretical and observational lines of evidence.  For this reason, the references below are necessarily only a sample of the work currently being done; see \cite{howe11}, \cite{hill+13}, \cite{maozmn14}, and \cite{tsebs15} for excellent reviews and further references.

\subsection{Traditional Formation Scenarios and Their Pitfalls}
\label{ssec:old_typeia}

Until recently, the most widely accepted progenitor scenarios have involved getting a CO WD to ignite by slowly adding mass to it \citep{hilln00}.  The added mass leads to compression and heating of its interior, but the latter is at least partly balanced by cooling from neutrino emissions, which prevents carbon ignition due to high temperatures.  As the CO WD approaches \Mch, its central density exceeds $\sim2\times10^9\,\gcc$, and the rate of heating from pycnonuclear carbon fusion -- i.e. carbon fusion due to extreme density -- exceeds that for neutrino cooling.  Because ignition occurs under highly degenerate conditions, the WD does not respond to this heating by expanding, and so, unlike a non-degenerate star, is unable to hydrodynamically regulate the nuclear reaction.  Instead, the WD experiences a runaway reaction that lasts $\sim1000\,\mrm{yr}$, until the timescale for nuclear burning at the WD's center becomes shorter than the star's dynamical time.  Dynamical burning then begins, and some kind of explosion becomes inevitable.

The various scenarios to get a CO WD to accrete slowly can subdivided into two classes, or ``channels'': the single-degenerate (SD) channel \citep{wheli73}, where the WD steadily accretes from a non-degenerate companion (a main sequence star, a giant, or a He-burning subdwarf; see references in \citealt{maozns14}), and the double-degenerate (DD) channel \citep{ibent84, webb84}, where two CO WDs with a total mass $\gtrsim\Mch$ merge, producing a merger remnant composed of a dense, degenerate ``core'' surrounded by a thick accretion disk.  Both scenarios are beset by a number of issues, which we summarize these below, following \citeauthor{vkercj10} (\citeyear{vkercj10}, henceforth \citeal{vkercj10}; see also \citealt{vker13}).

The first issue is that in order to match the observed SN Ia rate of $\sim 0.0023 \pm 0.0006$ for every solar mass of stars formed \citep{mann+05}, $\sim1$\% of all WDs formed (of any composition and regardless of binarity) must produce SNe Ia.  Compared to this relatively large number, there is an apparent paucity of CO WDs that can reach \Mch\ from either channel.  In hydrogen-accreting SD systems, efficient growth of the CO WD appears only achievable if the accretion rate is between $10^{-8} - 10^{-7}\,\Msun\,\pyr$.  Slower accretion results in nova outbusts that eject the accreted mass (\citealt{townsb04}; though see \citealt{zorosg11}), while faster accretion results in the buildup of an extended, red giant-like envelope that eventually engulfs the donor \citep{ibent84}.  Systems that do accrete at the correct rate -- and steadily burn hydrogen to helium -- should radiate supersoft x-rays, but observations of galactic x-ray flux suggest a factor of $10 - 100$ too few of these systems exist to explain the SNe Ia rate \citep{dist10, gilfb10}, and whether or not these systems can be ``hidden'' from view as rapidly-accreting enshrouded WDs is debateable (eg. \citealt{hachkn10, joha+14}).  Even if the accreted matter has been burned to helium, or the donor is He-rich, matter may still be ejected by subsequent helium flashes (\citealt{idanss13}; though see \citealt{hill+16}).  

Meanwhile, analytical estimates (\citealt{vkercj10}), binary population synthesis \citep{menn+10, ruitbf09, clae+14} and empirical counting of candidate systems \citep{badem12} all estimate that the merger rate of CO - CO WD binaries with total mass greater than $\sim\Mch$ falls short of the SN Ia rate by a factor of at least a few.  Not all of these mergers will necessarily end as SNe Ia, either: if post-merger evolution leads to off-center carbon ignition in the merger remnant, it will transform the remnant into an ONe WD, and an $\gtrsim\Mch$ ONe WD ends its life in an AIC, rather than exploding as an SN Ia \citep{nomoi85, saion85, yoonpr07, schw+12, schw+15}.
 
The second issue the difficulty for the thermonuclear explosion of an \Mch\ mass CO WD to replicate the properties of normal SNe Ia.  Normal SNe Ia synthesize $\sim0.5 - 1.3\,\Msun$ of radioactive \Ni\ (estimated from their bolometric light curves; eg \citealt{stri+06}), and feature absorption lines of intermediate-mass elements at maximum light, indicating that the explosion does not burn the entire WD to peak-iron elements, and lower-mass elements are preferentially located in the outer layers of the SN ejecta \citep{howe11, hill+13}.  If dynamical burning leads to an extended plateau of high overpressure a detonation \citep{seit+09} is triggered, where a supersonic shockwave drives through the WD and triggers nuclear fusion in its wake.  As most of the mass in an \Mch\ WD is $\gtrsim10^9\,\gcc$, a detonation would convert almost all of the WD to \Ni\ \citep{howe11, hill+13}.  On the other hand, if the explosion propagates as a subsonic deflagration, where a steep temperature gradient -- a flame front -- moves outward via conduction, the WD is able to expand during the explosion and, at lower densities, intermediate-mass elements are produced.  The explosion, however, produces slower velocity ejecta than seen in SNe Ia, and mixes burned and unburned material such that the ejecta do not appear stratified.  To resolve this, an \textit{ad-hoc} deflagration-to-detonation transition (DDT) is often invoked \citep{khok91}, the timing of which can be tuned to vary the amount of \Ni\ generated (eg. \citealt{hill+13}), though it remains to be seen this is a robust mechanism in realistic WD explosions.  It is also not obvious how invoking the DDT can explain the dependence of observed SNe Ia on the properties of their host galaxies, for example why more luminous SNe Ia tend to be in star-forming galaxies (eg. \citealt{hamu+00, howe+09, sull+10}).

The SD channel has a number of additional complications (and several others; see \citealt{maozns14}) that have led have the DD channel to fall into favor compared to the SD one.  For example, it requires a non-degenerate companion, which, under certain conditions, might be detectable, but attempts to spot the companion in pre-explosion archival data \citep{li+11cpn, nielvn13, niel+14}, during the supernova (as it responds to being hit by SN ejecta; \citealt{bloo+12,ollms15}), or after the explosion (eg. \citealt{kerz+14rem}).  For SD scenarios involving hydrogen-rich donors, the explosion is also expected to strip and entrain donor material, but attempts to find such material either do not detect hydrogen, or, in one recent case, apparently too little hydrogen to be consistent with SD donor stars \citep{magu+16}.  

%While these issues (and several others; see \citealt{maozns14}) have led the DD channel to fall into favor compared to the SD one, the issues discussed above affect \Mch\ progenitor scenarios in general.

% Mention gravitationally and pulsationally confined detonations??  Modern DDT: 2013MNRAS.429.1156S  GCD: http://adsabs.harvard.edu/abs/2016arXiv160600089S http://adsabs.harvard.edu/abs/2016ApJ...819..132G


\subsection{Brave New Channels}


%While these issues (and several others; see \citealt{maozns14}) have led the DD channel to fall into favor compared to the SD one, the issues discussed above affect \Mch\ progenitor scenarios in general.

% Mention gravitationally and pulsationally confined detonations??  Modern DDT: 2013MNRAS.429.1156S  GCD: http://adsabs.harvard.edu/abs/2016arXiv160600089S http://adsabs.harvard.edu/abs/2016ApJ...819..132G

%%%%%%%%%%%%%%%

%In the following section, we will argue that the Chandrasekhar scenario shows significant problems when compared to observations of SNe Ia, and suggest that the majority of SNe Ia are caused by mergers of CO WD binaries, including those that produce remnants with masses below \Mch.  This argument is a summarization of the argument in \cite{vankerkwijk}.

The challenges posed by the evidence above has spurred research into alternative scenarios that lead to exploding CO WDs.  These include:

\begin{itemize}
	\item The {\bf double-detonation channel} has many more viable progenitors, and population synthesis calculations suggest it may be able to account for all SNe Ia \citep{ruit+11}.  Because they are less dense, pure detonations of sub-{\Mch} CO WDs can well reproduce SNe Ia, as was shown by \cite{sim+10}.  This allows them to bypass the need to trigger a DDT.  Luminosity variation is then dependent on the mass of CO WD that explodes, and this relationship naturally explains the luminosity-environment relationship (since stars that generate lower-mass WDs take longer to evolve).  For the double-detonation channel to properly reproduce SN Ia spectra, however, the detonating helium shell must be very light ($\lesssim 0.05${\Msun}; see \citealt{woosk11}), and whether or not it is common (or even possible) for such shells to explode is an unanswered question.
	\item The {\bf violent merger channel} is a variant
	\item The {\bf direct collision}
	\item The {\bf core-degenerate channel} \citep{}, while not a 
\end{itemize}

We note that more unconventional channels have also been proposed that allow a CO WD to detonate due to collisions with planets and planetoids \citep{distfg15} or even lone sub-\Mch\ WDs due to compositional impurities near their core that lower the density required for the onset of pycnonuclear fusion to $\gtrsim0.9\,\Msun$ \citep{chio+15}.  Further work is required to show the physical viability of these channels, and whether they reproduce normal SNe Ia.  Regardless of whether or not they produce normal SNe Ia, study of these mechanisms is useful for explaining peculiar individual or subclasses of SNe Ia: aside from the two cases noted above, .  While either one of these could explain certain classes of peculiar SNe (eg. \citealt{fink+14, fishj15}), they do not produce normal SNe Ia.  

\section{The vK10 SN Ia Channel}

 with masses significantly \textit{below} \Mch\ can also explode.  This obviously bolsters substantially the number of CO WDs that could be SN Ia progenitors.  Sub-\Mch\ explosions also do not need to appeal to the deflagration-to-detonation transition: simulations of pure detonations of sub-\Mch\ CO WDs \citep{shig+92, sim+10} are competitive with \Mch\ explosion models for reproducing the light curves and spectra of SNe Ia \citep{none}.  Some SN Ia population trends, most notably the Phillips relation and the correlation between SN Ia luminosity and host stellar population age (eg. \cite{none}), can also naturally be explained by exploding CO WDs with a range of masses (\citeal{vkercj10}).

%The advantages of this channel are that it accounts for the absence of direct evidence for stellar companions, the observed SN Ia rate, and the dependence of SN Ia peak luminosity on the age of the host stellar population (because lower-mass merger constituents take longer to form).  Since pure detonations of sub-\Mch\ CO WDs produce light curves very similar to observed SNe Ia \citep{shig+92,sim+10}, it also removes the need for imposed deflagration-to-detonation transitions. 

%Important questions, however, remain, including what fraction of mergers leads to remnants that are hot near the center (in highly degenerate conditions), how the subsequent viscous phase proceeds in detail, whether ignition leads to a detonation, and whether the detonation of a remant that may still rotate and be surrounded by a disk would produce an event similar to an SN Ia.  

Hydrodynamic simulations \citep{loreig09} suggest that a $0.6-0.6\,\Msun$ merger never achieves temperatures high enough to ignite fusion (more recent work (eg. \citep{pakm+11, dan+12}) suggest $\sim0.9\,\Msun$ as the minimum accreting WD mass for this to occur); moreover, the remnant central density, $\sim2.5\times10^6\,\gcc$, is too low to produce \Ni\ in an explosion.  Following coalescence, however, the remnant, which is differentially rotating, enters a period of rapid angular momentum redistribution due to hydrodynamically or magnetically-mediated viscosity.  Using the standard $\alpha$-viscosity prescription \cite{shaks73} -- i.e. $\nu = \alpha c_s H_P$, where $c_s$ is the sound speed and $H_P$ the pressure scale height -- the timescale for viscous evolution can be estimated as

\begin{eqnarray}
t_\mrm{visc} &=& \frac{R_\mrm{disk}^2}{\nu} \sim \frac{1}{\alpha}\frac{R_\mrm{disk}^2}{H_P^2}\taudyn \nonumber \\
			&\sim& 3\times10^4\,\mrm{s}\left(\frac{10^{-2}}{\alpha}\right)\left(\frac{R_\mrm{disk}/H_P}{10}\right)^2\left(\frac{R_\mrm{disk}}{10^9\,\mrm{cm}}\right)^{3/2}\left(\frac{M_\mrm{enc}}{1\,\Msun}\right)^{-1/2},
\end{eqnarray}

\noindent where $M_\mrm{enc}$ is we have used $\taudyn \approx H_P/c_s$ and inserted a fiducial viscosity and typical numbers for remnants \citep{shen+12}.  Thus the vast majority of the remnant's angular momentum is transported away, and the remnant (including its disk) loses its rotational support against gravity, over a period $\sim10^4\,\mrm{s}$.\footnote{This is notably in contrast to earlier work (eg. \citep{nomoi85, yoonpr07}) that assume any rotationally-supported material will slowly accrete onto the dense core of the remnant at a near-Eddington mass accretion rate of $\dot{M} \sim 10^{-5}\,\Msun\,\pyr$.  Remnants are prone to magnetic instability (Sec. \ref{sec:intro_pme}), and will almost certainly evolve over the much shorter timescale given by the $\alpha$-viscosity estimate.}  This loss of rotational support combined with increasing weight from newly accreted disk material leads to compression and heating of the remnant core.  Since $\sim10^4\,\mrm{s}$ is far shorter than either the neutrino cooling timescale of $\taunu \sim 10^3\,\mrm{yr}$ or the thermal adjustment timescale of $\sim10^4\,\mrm{yr}$ \citep{shen+12}, compressional heating is adiabatic, and \citeal{vkercj10} estimates that for the $0.6-0.6\,\Msun$ remnant it leads both the central density and temperature to increase to $\gtrsim1.5\times10^7\,\gcc$ and $\gtrsim10^9\,\mrm{K}$, at which point a carbon nuclear runaway is inevitable.

Thus, the \citeal{vkercj10} channel is both attractive for its many advantages and plausible given the order-of-magnitude estimates above.  Investigating whether these estimates hold under detailed scrutiny, and to determine which, if any, systems in the CO WD binary parameter space could follow the channel, is the purpose of this PhD thesis.

\section{Hot DQs, High-Field Magnetic WDs, and Other Oddities}

If the remnant of a CO WD merger is not eventually (at least partially) destroyed in an explosion, it will go on to 

%http://adsabs.harvard.edu/abs/2016ApJ...817...27W

Stable nuclear burning, however, 

%The merger of two white dwarfs (WDs) originally in a short-period binary is estimated (eg. \citealt{badem12}) to occur about once every century in a Milky Way-like galaxy, making the products of such events common throughout the universe.  They have been held responsible for producing a variety of stars with strange properties, including helium-burning sdOB stars \citep{saioj00, justph11}, RCrB stars (eg. \citealt{webb84, clay+07, clay13}), and massive and highly magnetized WDs (eg. \citealt{segrcm97, garc+12, kule+13}) that could resemble the hot DQ WDs (eg. \citealt{dunlc15}), Dunlap and Clements in preparation).  They may, however, also be responsible for spectacular transient events including accretion-induced collapses (eg. \citealt{saion85, abdi+10}) and type Ia supernovae (SNe Ia; eg. \citealt{howe11, hill+13, maozmn14}).  Determining the final outcome of a particular merger requires an understanding of the detailed dynamics of the merging process, which cannot directly be seen using current observational capabilities.  Thus, studies of merger physics have primarily utilized hydrodynamic simulations.



\section{Thesis Overview}

The chapters are ordered in accordance with the proposed evolution of the \citeal{vkercj10} channel.  In Chapter \ref{ch:ch2}, we consider the range of possible merger remnant configurations to arise from the parameter space of merging CO WD binaries.  In Chapter \ref{ch:ch3}

For the most part, I have reproduced exactly the texts of \citeal{zhu+13}, \citeal{zhu+15} and \citeal{zhu+16} in their respective chapters.  The exceptions are the chapter introductions, where I have excised certain paragraphs to eliminate the redundancy of having multiple paragraphs repeating an overview of the \citeal{vkercj10} channel.  The papers' abstracts have also been modified into chapter overviews, and certain figures reformatted for readability.  Any additional changes are noted at the start of each chapter.  Most prominently, we have added a postscript to Chapter \ref{ch:ch2} (Sec. \ref{sec:postscript_pme}) that considers our simple semi-analytical prescription for post-merger viscous evolution in light of new results, and we included an extensive appendix to Chapter \ref{ch:ch5} that details our calculation of convective suppression in magnetized, rotating WDs.

