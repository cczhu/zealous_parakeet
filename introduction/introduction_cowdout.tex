\section{The Mystery of Type Ia Supernovae}

%The outcome of the merger of two CO WDs is uncertain in part because during the merger temperatures do not become hot enough to ignite significant carbon fusion (e.g., \citealt{loreig09}, \citeal{loreig09} hereafter), except possibly for masses above $\sim\!0.9\,M_\odot$ \citep{pakm+11,pakm+12}.  Hence, the final fate depends on subsequent evolution, in which differential rotation is dissipated, the remnant disk accretes, and the whole remnant possibly spins down.  Due to these processes, the remnant could be compressed and heated, which, if it happens faster than the thermal timescale, would lead to increased temperatures and thus potentially to ignition.  

%So far, efforts have focused on merging binaries with total mass $M>\Mch$.  The end result of such mergers is believed to be either stable off-center carbon ignition, which would turn the merger remnant into an oxygen-neon WD and possibly eventually result in accretion-induced collapse \citep{saion98}, or slow accretion, which allows the remnant to stay cool and eventually ignite at high central density \citep{yoonpr07}.  Less massive mergers were usually thought to result in more massive, rapidly rotating CO WDs \citep{segrcm97,kube+10}, but more recently it has been realized these might eventually become hot enough to ignite (\citealt{vkercj10}, \citeal{vkercj10} hereafter; \citealt{shen+12,schw+12}).  Indeed, \citeal{vkercj10} argue that type Ia supernovae result generally from mergers of CO WDs with similar masses, independent of whether or not their total mass exceeds \Mch\ (see below).  For all these studies, the conclusions on whether and where ignition takes place depend critically on the structure of the merger remnnant.

%The merger of two white dwarfs (WDs) originally in a short-period binary is estimated (eg. \citealt{badem12}) to occur about once every century in a Milky Way-like galaxy, making the products of such events common throughout the universe.  They have been held responsible for producing a variety of stars with strange properties, including helium-burning sdOB stars \citep{saioj00, justph11}, RCrB stars (eg. \citealt{webb84, clay+07, clay13}), and massive and highly magnetized WDs (eg. \citealt{segrcm97, garc+12, kule+13}) that could resemble the hot DQ WDs (eg. \citealt{dunlc15}), Dunlap and Clements in preparation).  They may, however, also be responsible for spectacular transient events including accretion-induced collapses (eg. \citealt{saion85, abdi+10}) and type Ia supernovae (SNe Ia; eg. \citealt{howe11, hill+13, maozmn14}).  Determining the final outcome of a particular merger requires an understanding of the detailed dynamics of the merging process, which cannot directly be seen using current observational capabilities.  Thus, studies of merger physics have primarily utilized hydrodynamic simulations.

%It was only in the 20th century, however, that the they were recognized, through photometric and spectroscopic observations, as being a phenomena separate from novae \citep{baadz34}, and supernovae caused by the core-collapse of massive stars \citep{mink41, elias}.  

Type Ia supernovae, or SNe Ia, have been observed by astronomers for centuries.  SN 1572, for example, was observed by Danish astronomer Tycho Brahe to be ``far beyond the Moon'', and helped lead to the abandonoment of the Aristotilian concept that the heavens were immutable (eg. \citealt{ruiz04, krau+08}).  Today, SNe Ia are classified (eg. \citealt{fili97, li+11}) by the lack of hydrogen and helium, as well as the strong prescence of ionized silicon (Si II), in their spectra.  They are estimated to comprise a quarter of all supernovae in the local universe \citep{li+11}, but are found disproportionately often by surveys because most of their peak bolometric luminosity ($\sim10^{43}\,\ergpsec$; eg. \citealt{fili97, li+11}) is emitted in the optical.  As a class of objects, they are remarkably homogeneous, and exhibit only slight variations that can -- to first order -- be parameterized by a single independent variable \citep{hilln00, howe11}.  This parameterizability, as well as their intrinsic brightness, make SNe Ia outstanding cosmological distance indicators.  Their most famous use in this context is the much-celebrated discovery of \citep{ries98} and \cite{perl99} that the expansion of the universe is accelerating under the influence of a ``dark energy'', the exact nature of which remains mysterious.  SNe Ia are also sources of heat and heavy elements (iron, in particular) for the galaxies that host them, and so play a crucial role in the formation of stars and evolution of galaxies \citep{leib00}.

Despite their ubiquity and utility, however, the exact nature (or natures) of their progenitor systems remains mysterious.  They were first proposed to be explosions of CO WDs by \citep{hoylf60} based on the composition of SNe Ia ejecta.  This is now well-established by the similarity of the light curve, energetics and spectrum of a typical SN Ia to those calculated for an exploding CO WD.  Also, early-time observations of the recent SN 2011fe have constrained the radius of the exploding object to be $\lesssim0.1\,\Rsun$ \citep{nuge+11, bloo+12, maozmn14}, consistent with a CO WD, and late-time observations of SN 2014J have detected gamma-ray emission from the decay of \Ni, produced by the burning of carbon and oxygen to nuclear statistical equilibrium, into stable $^{56}$Fe \citep{chur+14}.  What is much less well-understood is how the CO WD is made to explode, and a vast body of literature now exists exploring the various theoretical and observational lines of evidence.  See \cite{howe11}, \cite{hill+13}, \cite{maozmn14}, and \cite{tsebs15} for excellent reviews.

\subsection{Traditional Formation Scenarios and Their Pitfalls}
\label{ssec:old_typeia}

Until recently, the most widely accepted progenitor scenarios have involved getting a CO WD to ignite by slowly adding mass to it (eg. \citealt{hilln00}).  The added mass leads to compression and heating of its interior, but the latter is at least partly balanced by cooling from neutrino emissions, which prevents carbon ignition due to high temperatures.  As the CO WD approaches \Mch, its central density exceeds $\sim2\times10^9\,\gcc$, and the rate of heating from pycnonuclear carbon fusion -- i.e. carbon fusion due to extreme density -- exceeds that for neutrino cooling.  Because ignition occurs under highly degenerate conditions, the WD does not respond to this heating by expanding, and so, unlike a non-degenerate star, is unable to hydrodynamically regulate the nuclear reaction.  Instead, the WD experiences a runaway reaction that lasts $\sim1000\,\mrm{yr}$, until the timescale for nuclear burning at the WD's center becomes shorter than the star's dynamical time.  Some kind of explosion is then inevitable.

 a timescale $\gtrsim 10^6\,\mrm{yr}$

The various scenarios to get a CO WD to accrete slowly can subdivided into two classes, or ``channels'': the single-degenerate (SD) channel \citep{wheli73}, where the WD steadily accretes from a non-degenerate companion (a main sequence star, a giant, or a He-burning subdwarf; see references in \citealt{maozns14}), and the double-degenerate (DD) channel \citep{ibent84, webb84}, where two CO WDs with a total mass $\gtrsim\Mch$ merge.  Both scenarios are beset by a number of issues, some of them common; we summarize these below, following \citeauthor{vkercj10} (\citeyear{vkercj10}, henceforth \citeal{vkercj10}; see also \citealt{vker13}).

The first issue, in order to match the observed SN Ia rate, $\sim1$\% of all WDs formed (of any composition and regardless of binarity) must produce SNe Ia (\citeal{vkercj10}, \citealt{vker13}).  Compared to this relatively large number, there is an apparent paucity of CO WDs that can reach \Mch\ from either channel.  In hydrogen-accreting SD systems, efficient growth of the CO WD appears only achievable if the accretion rate is between $10^{-8} - 10^{-7}\,\Msun\pyr$ {\Msun}/yr.  Slower accretion results in nova outbusts that eject the accreted mass (\citealt{townsb04}; though see \citealt{zorosg11}), while faster accretion results in the buildup of an extended, red giant-like envelope that eventually engulfs the donor \citep{ibent84}.  Systems that do accrete at the correct rate -- and steadily burn hydrogen to helium -- should radiate supersoft x-rays, but observations of galactic x-ray flux suggest a factor of $10 - 100$ too few of these systems exist to explain the SNe Ia rate \citep{dste10, gilfb10}, and whether or not these systems can be ``hidden'' from view as rapidly-accreting enshrouded WDs is debateable (eg. \citealt{hachkn10, joha+14}).  Even if the accreted matter has been burned to helium, or the donor is He-rich, matter may still be ejected by subsequent helium flashes (\citealt{idanss13}; though see \citealt{hill+16}).

Meanwhile, both binary population synthesis \citep{menn+10, ruitbf09, clae and WD binary surveys \citep{badem12} estimate that the merger rate of CO - CO WD binaries with total mass greater than $\sim\Mch$ falls short of the SN Ia rate by a factor of at least a few.  The second is BLAH  Indeed,   Also mention sub-luminous and superluminous SNe Ia.

For DD systems,  suffers from a lack of progenitors: population synthesis (\citealt{menn+10}; \citealt{ruitbf09}, though see discussion within) and empirical counting of candidate systems \citep{badem12} give rates too small to explain all SNe.  Following the merger, accretion of the disk onto the remnant must also be fine-tuned to below $\sim 10^{-5}$ {\Msun}/yr to prevent the accretion stream from lighting off-centre carbon fusion \citep{yoonpr07}.  Off-centre fusion will turn the merger remnant into an oxygen-neon (ONe) WD, and it is commonly believed that a super-{\Mch} ONe WD turns into a neutron star via accretion-induced collapse (AIC), rather than exploding as an SN Ia \citep{yoonpr07}.  Additionally, this channel requires a DDT, and cannot explain the luminosity-environment relation.

The second issue








Surveys for supersoft sources \citep{none} suggest they are an order of magnitude too rare to be the primary source of SNe Ia.  Meanwhile, both binary population synthesis \citep{none} and WD binary surveys \citep{badem12} estimate that the merger rate of CO - CO WD binaries with total mass greater than $\sim\Mch$ falls short of the SN Ia rate by a factor of at least a few.  The second is BLAH  Indeed,   Also mention sub-luminous and superluminous SNe Ia.





%%%%%%%%%%%%%%



Another issue is that a nuclear explosion from an {\Mch} WD can propagate subsonically, known as a deflagration, or supersonically, a detonation.  Pure deflagrations produce ejecta that are too slow compared with observed SNe Ia, and pure detonations produce far too much $^{56}$Ni \citep{hilln00, howe11}.  To resolve this, an \textit{ad-hoc} deflagration-to-detonation transition (DDT) is invoked \citep{khok91, howe11}.  A third issue is that SNe Ia have a distribution of luminosities, and more luminous SNe Ia tend to occur in late-type galaxies \citep{sull10}.  While the timing of the DDT can be adjusted to provide this luminosity distribution \citep{hilln00,vker12}, it is far from obvious what DDT timing has to do with the environment of the SN.




The second is the difficulty for the thermonuclear explosion of an \Mch\ mass CO WD to replicate the nucleosynthetic products and ejecta velocities of ordinary SN Ia, as well as their \cite{phil93} relation between peak luminosity and brightness decay time, without appealing to an \textit{ad-hoc} deflagration-to-detonation transition \citep{khok91}.  The transition would also have to be consistent with the correlation between the luminosity of an SN Ia and the age of its stellar environment \citep{none}.

The second is the difficulty for the thermonuclear explosion of an \Mch\ mass CO WD to replicate the properties and population-level trends of ordinary SN Ia (such as the \cite{phil93} relation between peak luminosity and brightness decay time), without appealing to an \textit{ad-hoc} deflagration-to-detonation transition \citep{khok91}


The single degenerate channel can only operate when the accretion rate onto the WD is somewhere between $10^{-8} - 10^{-7}\,\Msun\pyr$ {\Msun}/yr so that the accreted material can steadily burn to carbon or oxygen (slower accretion results in novae, while faster accretion results in an expanding envelope and mass loss) \citep{howe11}.  Systems that do accrete at the correct rate should radiate very soft x-rays, but observations of galactic x-ray flux suggest a factor of 10 - 100 too few of these systems exist to explain the SNe Ia rate \citep{dste10, gilfb10}.  Whether or not these systems can be ``hidden'' from view is still under debate (ex. \citealt{hachkn10}, \citealt{bojedn11}).  Another issue is that a nuclear explosion from an {\Mch} WD can propagate subsonically, known as a deflagration, or supersonically, a detonation.  Pure deflagrations produce ejecta that are too slow compared with observed SNe Ia, and pure detonations produce far too much $^{56}$Ni \citep{hilln00, howe11}.  To resolve this, an \textit{ad-hoc} deflagration-to-detonation transition (DDT) is invoked \citep{khok91, howe11}.  A third issue is that SNe Ia have a distribution of luminosities, and more luminous SNe Ia tend to occur in late-type galaxies \citep{sull10}.  While the timing of the DDT can be adjusted to provide this luminosity distribution \citep{hilln00,vker12}, it is far from obvious what DDT timing has to do with the environment of the SN.

The double degenerate {\Mch} channel also suffers from a lack of progenitors: population synthesis (\citealt{menn+10}; \citealt{ruitbf09}, though see discussion within) and empirical counting of candidate systems \citep{badem12} give rates too small to explain all SNe.  Following the merger, accretion of the disk onto the remnant must also be fine-tuned to below $\sim 10^{-5}$ {\Msun}/yr to prevent the accretion stream from lighting off-centre carbon fusion \citep{yoonpr07}.  Off-centre fusion will turn the merger remnant into an oxygen-neon (ONe) WD, and it is commonly believed that a super-{\Mch} ONe WD turns into a neutron star via accretion-induced collapse (AIC), rather than exploding as an SN Ia \citep{yoonpr07}.  Additionally, this channel requires a DDT, and cannot explain the luminosity-environment relation.

The double-detonation channel has many more viable progenitors, and population synthesis calculations suggest it may be able to account for all SNe Ia \citep{ruit+11}.  Because they are less dense, pure detonations of sub-{\Mch} CO WDs can well reproduce SNe Ia, as was shown by \cite{sim+10}.  This allows them to bypass the need to trigger a DDT.  Luminosity variation is then dependent on the mass of CO WD that explodes, and this relationship naturally explains the luminosity-environment relationship (since stars that generate lower-mass WDs take longer to evolve).  For the double-detonation channel to properly reproduce SN Ia spectra, however, the detonating helium shell must be very light ($\lesssim 0.05${\Msun}; see \citealt{woosk11}), and whether or not it is common (or even possible) for such shells to explode is an unanswered question.



In the following section, we will argue that the Chandrasekhar scenario shows significant problems when compared to observations of SNe Ia, and suggest that the majority of SNe Ia are caused by mergers of CO WD binaries, including those that produce remnants with masses below \Mch.  This argument is a summarization of the argument in \cite{vankerkwijk}.

The traditional method of generating an SN Ia involves slow accretion of material over $\sim10^6$ yr onto the CO WD from either a non-degenerate companion or an accretion disk following a merger with another WD.  As the CO WD approaches the Chandrasekhar mass \Mch, its central density becomes sufficiently high that the rate of energy generation from pycnonuclear carbon-carbon nuclear fusion exceeds that at which neutrino cooling can transport it away.  The resulting increase in central entropy establishes a convection zone that transports the heat of the nuclear burning region to much of the WD interior.  The WD is highly degenerate, however, and does not expand and cool in response to heating.  Instead, a nuclear runaway ensues over the next thousand years as the WD grows ever hotter, a period of evolution referred to as the ``simmering phase''.  \Mch\ WDs eventually become hot enough that the timescale for nuclear burning becomes shorter than the dynamical time -- we refer to this as ``dynamical burning'' -- at which point an explosion becomes inevitable.


% \citep{woos99}



%This scenario is beset by two major issues (\citealt{vkercj10}, henceforth \citeal{vkercj10}; \citealt{vker13}) that have not been resolved that have not been resolved (or indeed may have even been exacerbated) despite the increasing quality of observations and theoretical/numerical models in the last few decades.  The first is that, in order to match the observed SN Ia rate, $\sim1$\% of all WDs formed (of any composition and regardless of binarity) must produce SNe Ia (\citeal{vkercj10}, \citealt{vker13}).  Compared to this relatively large number, there is an apparent paucity of CO WDs that can reach \Mch.  In hydrogen-accreting single degenerate systems, efficient growth of the CO WD appears only achievable if the accretion rate is $\sim10^{-7}$ \Msun\ yr$^{-1}$, making these systems prominent supersoft x-ray emittors.  Surveys for supersoft sources \citep{none} suggest they are an order of magnitude too rare to be the primary source of SNe Ia (though see discussion on the ``spin-up/spin-down'' model; \citealt{none}).  Meanwhile, both binary population synthesis \citep{none} and WD binary surveys \citep{badem12} estimate that the merger rate of CO - CO WD binaries with total mass greater than $\sim\Mch$ falls short of the SN Ia rate by a factor of at least a few.  The second is BLAH  Indeed,   Also mention sub-luminous and superluminous SNe Ia.

This scenario is beset by several issues (\citealt{vkercj10}, henceforth \citeal{vkercj10}, and references therein).  The first is that, in order to match the observed SN Ia rate, $\sim1$\% of all WDs formed (of any composition and regardless of binarity) must produce SNe Ia \citep{prichs08, vker13}.  Compared to this relatively large number, there is an apparent paucity -- by a factor of at least a few  -- of both CO WDs accreting efficiently from non-degenerate companions \citep{none} and merging CO - CO WD binaries whose total mass exceeds \Mch\ \citep{badem12}.  The second is the difficulty for the thermonuclear explosion of an \Mch\ mass CO WD to replicate the properties and population-level trends of ordinary SN Ia (such as the \cite{phil93} relation between peak luminosity and brightness decay time), without appealing to an \textit{ad-hoc} deflagration-to-detonation transition \citep{khok91}.  Recent years have also revealed substantial populations of sub and superluminous SNe Ia (eg. \cite{none}).  If these SNe are also generated by exploding CO WDs, then either the classical scenario above is robust enough to explain them, or alternative channels to exploding WDs must also occur in nature.

%The second is the difficulty for the thermonuclear explosion of an \Mch\ mass CO WD to replicate the nucleosynthetic products and ejecta velocities of ordinary SN Ia, as well as their \cite{phil93} relation between peak luminosity and brightness decay time, without appealing to an \textit{ad-hoc} deflagration-to-detonation transition \citep{khok91}.  The transition would also have to be consistent with the correlation between the luminosity of an SN Ia and the age of its stellar environment \citep{none}.  Recent years have revealed substantial populations of sub and superluminous SNe Ia that deviate from the Phillips relation.  If these SNe are indeed generated by exploding WDs (SNe Ia resembling SN 2005E may be due to core-collapse \citep{none}), then either the classical pycnonuclear scenario above must be robust enough to explain them, or alternative channels to exploding WDs must also occur in nature.

%http://adsabs.harvard.edu/abs/2012PASA...29..447M for more references and discussion

The challenge posed by these issues has spurred research into alternative scenarios where CO WDs with masses significantly \textit{below} \Mch\ can also explode.  This obviously bolsters substantially the number of CO WDs that could be SN Ia progenitors.  Sub-\Mch\ explosions also do not need to appeal to the deflagration-to-detonation transition: simulations of pure detonations of sub-\Mch\ CO WDs \citep{shig+92, sim+10} are competitive with \Mch\ explosion models for reproducing the light curves and spectra of SNe Ia \citep{none}.  Some SN Ia population trends, most notably the Phillips relation and the correlation between SN Ia luminosity and host stellar population age (eg. \cite{none}), can also naturally be explained by exploding CO WDs with a range of masses (\citeal{vkercj10}).



\subsection{Brave New Channels}

\subsubsection{The Double-Detonation Channel}

\subsubsection{The Core-Degenerate Channel}

\subsubsection{The Direct Collision Channel}

We note that more unconventional channels have also been proposed that allow a CO WD to detonate due to collisions with planets and planetoids \citep{distfg15} or even lone sub-\Mch\ WDs due to compositional impurities near their core that lower the density required for the onset of pycnonuclear fusion to $\gtrsim0.9\,\Msun$ \citep{chio+15}.  Further work is required to show the physical viability of these channels, and whether they reproduce normal SNe Ia.

\section{The vK10 SN Ia Channel}

%The advantages of this channel are that it accounts for the absence of direct evidence for stellar companions, the observed SN Ia rate, and the dependence of SN Ia peak luminosity on the age of the host stellar population (because lower-mass merger constituents take longer to form).  Since pure detonations of sub-\Mch\ CO WDs produce light curves very similar to observed SNe Ia \citep{shig+92,sim+10}, it also removes the need for imposed deflagration-to-detonation transitions. 

%Important questions, however, remain, including what fraction of mergers leads to remnants that are hot near the center (in highly degenerate conditions), how the subsequent viscous phase proceeds in detail, whether ignition leads to a detonation, and whether the detonation of a remant that may still rotate and be surrounded by a disk would produce an event similar to an SN Ia.  

Hydrodynamic simulations \citep{loreig09} suggest that a $0.6-0.6\,\Msun$ merger never achieves temperatures high enough to ignite fusion (more recent work (eg. \citep{pakm+11, dan+12}) suggest $\sim0.9\,\Msun$ as the minimum accreting WD mass for this to occur); moreover, the remnant central density, $\sim2.5\times10^6\,\gcc$, is too low to produce \Ni\ in an explosion.  Following coalescence, however, the remnant, which is differentially rotating, enters a period of rapid angular momentum redistribution due to hydrodynamically or magnetically-mediated viscosity.  Using the standard $\alpha$-viscosity prescription \cite{shaks73} -- i.e. $\nu = \alpha c_s H_P$, where $c_s$ is the sound speed and $H_P$ the pressure scale height -- the timescale for viscous evolution can be estimated as

\begin{eqnarray}
t_\mrm{visc} &=& \frac{R_\mrm{disk}^2}{\nu} \sim \frac{1}{\alpha}\frac{R_\mrm{disk}^2}{H_P^2}\taudyn \nonumber \\
			&\sim& 3\times10^4\,\mrm{s}\left(\frac{10^{-2}}{\alpha}\right)\left(\frac{R_\mrm{disk}/H_P}{10}\right)^2\left(\frac{R_\mrm{disk}}{10^9\,\mrm{cm}}\right)^{3/2}\left(\frac{M_\mrm{enc}}{1\,\Msun}\right)^{-1/2},
\end{eqnarray}

\noindent where $M_\mrm{enc}$ is we have used $\taudyn \approx H_P/c_s$ and inserted a fiducial viscosity and typical numbers for remnants \citep{shen+12}.  Thus the vast majority of the remnant's angular momentum is transported away, and the remnant (including its disk) loses its rotational support against gravity, over a period $\sim10^4\,\mrm{s}$.\footnote{This is notably in contrast to earlier work (eg. \citep{nomoi85, yoonpr07}) that assume any rotationally-supported material will slowly accrete onto the dense core of the remnant at a near-Eddington mass accretion rate of $\dot{M} \sim 10^{-5}\,\Msun\pyr$.  Remnants are prone to magnetic instability (Sec. \ref{sec:intro_pme}), and will almost certainly evolve over the much shorter timescale given by the $\alpha$-viscosity estimate.}  This loss of rotational support combined with increasing weight from newly accreted disk material leads to compression and heating of the remnant core.  Since $\sim10^4\,\mrm{s}$ is far shorter than either the neutrino cooling timescale of $\taunu \sim 10^3\,\mrm{yr}$ or the thermal adjustment timescale of $\sim10^4\,\mrm{yr}$ \citep{shen+12}, compressional heating is adiabatic, and \citeal{vkercj10} estimates that for the $0.6-0.6\,\Msun$ remnant it leads both the central density and temperature to increase to $\gtrsim1.5\times10^7\,\gcc$ and $\gtrsim10^9\,\mrm{K}$, at which point a carbon nuclear runaway is inevitable.

Thus, the \citeal{vkercj10} channel is both attractive for its many advantages and plausible given the order-of-magnitude estimates above.  Investigating whether these estimates hold under detailed scrutiny, and to determine which, if any, systems in the CO WD binary parameter space could follow the channel, is the purpose of this PhD thesis.

\section{Hot DQs, High-Field Magnetic WDs, and Other Oddities}

If the remnant of a CO WD merger is not eventually (at least partially) destroyed in an explosion, it will go on to 

%http://adsabs.harvard.edu/abs/2016ApJ...817...27W

Stable nuclear burning, however, 


\section{Thesis Overview}

The chapters are ordered in accordance with the proposed evolution of the \citeal{vkercj10} channel.  In Chapter \ref{ch:ch2}, we consider the range of possible merger remnant configurations to arise from the parameter space of merging CO WD binaries.  In Chapter \ref{ch:ch3}

For the most part, I have reproduced exactly the texts of \citeal{zhu+13}, \citeal{zhu+15} and \citeal{zhu+16} in their respective chapters.  The exceptions are the chapter introductions, where I have excised certain paragraphs to eliminate the redundancy of having multiple paragraphs repeating an overview of the \citeal{vkercj10} channel.  The papers' abstracts have also been modified into chapter overviews, and certain figures reformatted for readability.  Any additional changes are noted at the start of each chapter.  Most prominently, we have added a postscript to Chapter \ref{ch:ch2} (Sec. \ref{sec:postscript_pme}) that considers our simple semi-analytical prescription for post-merger viscous evolution in light of new results, and we included an extensive appendix to Chapter \ref{ch:ch5} that details our calculation of convective suppression in magnetized, rotating WDs.

