\section{The Mystery of Type Ia Supernovae}
\label{sec:c1_mysteryofsneia}

%It was only in the 20th century, however, that the they were recognized, through photometric and spectroscopic observations, as being a phenomena separate from novae \citep{baadz34}, and supernovae caused by the core-collapse of massive stars \citep{mink41, elias}.

Type Ia supernovae, or SNe Ia, have been observed by astronomers for centuries.  SN 1572, for example, was observed by Danish astronomer Tycho Brahe to be ``far beyond the Moon'', and helped lead to the abandonoment of the Aristotilian concept that the heavens were immutable \citep{ruiz04, krau+08}.

Today, SNe Ia are classified (eg. \citealt{fili97, li+11}) by the lack of hydrogen and helium, as well as the strong prescence of ionized silicon (Si II), in their spectra.  They are estimated to comprise a quarter of all supernovae in the local universe \citep{li+11}, but are found disproportionately often by surveys because most of their luminosity is emitted in the optical \citep{howe11}.  ``Normal'' SNe Ia \citep{bran98, bran+06} typically reach a maximum bolometric luminosity of $\sim10^{43}\,\ergpsec$ (see eg. \citep{fili97, hill+13}) after 18 - 20 days, followed by an order-of-magnitude decline in brightness over a month, and a slower, exponential decline of a factor of $\sim2.5$ every month (generally attributed to heat from radioactive decay within the ejecta).  Their spectroscopic features show they are composed of a combination of intermediate-mass elements (eg. \citealt{arne96}) such as silicon and calcium, and peak-iron elements such as iron and nickel \citep{fili97}; indeed they are the primary source of peak-iron elements for the galaxies that host them, and so play a crucial role in star formation and galactic chemical evolution \citep{leib00}.  Up to $\sim30$\% \citep{li+11} of explosions classified as SNe Ia are ``peculiar'' phenomena that are orders of magnitude fainter (eg. SN 2002cx \citep{li+02, fole+13}, SN 1991bg \citep{mazz+97}) and occassionally substantially brighter (eg. SN 2009dc; \citealt{yama+09, taub+09}).  Normal SNe Ia, however, are remarkably homogeneous, and exhibit variations that can -- to first order -- be parameterized by a single variable \citep{hilln00, howe11, maozns14}.  This is reflected most famously in the \cite{phil93} relation, where SNe Ia with greater peak brightnesses tend to evolve more slowly in time.  

Their parameterizability, as well as their intrinsic brightness, make SNe Ia outstanding cosmological distance indicators.  They were most famously used in this context in the much-celebrated discovery of \citep{ries+98} and \cite{perl+99} that the expansion of the universe is accelerating under the influence of a ``dark energy'', the exact nature of which remains mysterious.

Despite their ubiquity and utility, however, the exact nature (or natures) of their progenitor systems remains mysterious.  They were first proposed to be explosions of CO WDs by \citep{hoylf60} based on the composition of SNe Ia ejecta.  This is now well-established by the similarity of the light curve, energetics and spectrum of a typical SN Ia to those calculated for an exploding CO WD.  Also, early-time observations of the recent SN 2011fe have constrained the radius of the exploding object to be $\lesssim0.1\,\Rsun$ \citep{nuge+11, bloo+12, maozmn14}, consistent with a CO WD, and late-time observations of SN 2014J have detected gamma-ray emission from the decay of \Ni, produced by the burning of carbon and oxygen to nuclear statistical equilibrium, into stable $^{56}$Fe \citep{chur+14}.  What is much less well-understood is how the CO WD is made to explode, and a vast body of literature now exists exploring the various theoretical and observational lines of evidence.  For this reason, the references below are necessarily only a sample of the work currently being done; see \cite{howe11}, \cite{hill+13}, \cite{maozmn14}, and \cite{tsebs15} for excellent reviews and further references.

\subsection{Traditional Formation Scenarios and Their Pitfalls}
\label{ssec:c1_old_typeia}

Until recently, the most widely accepted progenitor scenarios have involved getting a CO WD to ignite by slowly adding mass to it \citep{hilln00}.  The added mass leads to compression and heating of its interior, but the latter is at least partly balanced by cooling from neutrino emissions, which prevents carbon ignition due to high temperatures.  As the CO WD approaches \Mch, its central density exceeds $\sim2\times10^9\,\gcc$, and the rate of heating from pycnonuclear carbon fusion -- i.e. carbon fusion due to extreme density -- exceeds that for neutrino cooling.  Because ignition occurs under highly degenerate conditions, the WD does not respond to this heating by expanding, and so, unlike a non-degenerate star, is unable to hydrodynamically regulate the nuclear reaction.  Instead, the WD experiences a runaway reaction that lasts $\sim1000\,\mrm{yr}$, until the timescale for nuclear burning at the WD's center becomes shorter than the star's dynamical time.  Dynamical burning then begins, and some kind of explosion becomes inevitable.

The various scenarios to get a CO WD to accrete slowly can subdivided into two classes, or ``channels'': the single-degenerate (SD) channel \citep{wheli73}, where the WD steadily accretes from a non-degenerate companion (a main sequence star, a giant, or a He-burning subdwarf; see references in \citealt{maozns14}), and the double-degenerate (DD) channel \citep{ibent84, webb84}, where two CO WDs with a total mass $\gtrsim\Mch$ merge, producing a merger remnant composed of a dense, degenerate ``core'' surrounded by a thick accretion disk.  Both scenarios are beset by a number of issues, which we summarize these below, following \citeauthor{vkercj10} (\citeyear{vkercj10}, henceforth \citeal{vkercj10}; see also \citealt{vker13}).

The first issue is that in order to match the observed SN Ia rate of $\sim 0.0023 \pm 0.0006$ for every solar mass of stars formed \citep{mann+05}, $\sim1$\% of all WDs formed (of any composition and regardless of binarity) must produce SNe Ia.  Compared to this relatively large number, there is an apparent paucity of CO WDs that can reach \Mch\ from either channel.  In hydrogen-accreting SD systems, efficient growth of the CO WD appears only achievable if the accretion rate is between $10^{-8} - 10^{-7}\,\Msun\,\pyr$.  Slower accretion results in nova outbusts that eject the accreted mass (\citealt{townsb04}; though see \citealt{zorosg11}), while faster accretion results in the buildup of an extended, red giant-like envelope that eventually engulfs the donor \citep{ibent84}.  Systems that do accrete at the correct rate -- and steadily burn hydrogen to helium -- should radiate supersoft x-rays, but observations of galactic x-ray flux suggest a factor of $10 - 100$ too few of these systems exist to explain the SNe Ia rate \citep{dist10, gilfb10}, and whether or not these systems can be ``hidden'' from view as rapidly-accreting enshrouded WDs is debateable (eg. \citealt{hachkn10, joha+14}).  Even if the accreted matter has been burned to helium, or the donor is He-rich, matter may still be ejected by subsequent helium flashes (\citealt{idanss13}; though see \citealt{hill+16}).  

Meanwhile, analytical estimates (\citealt{vkercj10}), binary population synthesis \citep{menn+10, ruitbf09, clae+14, maozns14} and empirical counting of candidate systems \citep{badem12} all estimate that the merger rate of CO - CO WD binaries with total mass greater than $\sim\Mch$ falls short of the SN Ia rate by a factor of at least a few.  Not all of these mergers will necessarily end as SNe Ia, either: if post-merger evolution leads to off-center carbon ignition in the merger remnant, it will transform the remnant into an ONe WD, and an $\gtrsim\Mch$ ONe WD ends its life in an AIC, rather than exploding as an SN Ia \citep{nomoi85, saion85, yoonpr07, schw+16}.
 
The second issue the difficulty for the thermonuclear explosion of an \Mch\ mass CO WD to replicate the properties of normal SNe Ia.  Normal SNe Ia synthesize $\sim0.5 - 1.3\,\Msun$ of radioactive \Ni\ (estimated from their bolometric light curves; eg \citealt{stri+06}), and feature absorption lines of intermediate-mass elements at maximum light, indicating that the explosion does not burn the entire WD to peak-iron elements, and lower-mass elements are preferentially located in the outer layers of the SN ejecta \citep{howe11, hill+13}.  If dynamical burning leads to an extended plateau of high overpressure a detonation \citep{seit+09} is triggered, where a supersonic shockwave drives through the WD and triggers nuclear fusion in its wake.  As most of the mass in an \Mch\ WD is $\gtrsim10^9\,\gcc$, a detonation would convert almost all of the WD to \Ni\ \citep{howe11, hill+13}.  On the other hand, if the explosion propagates as a subsonic deflagration, where a steep temperature gradient -- a flame front -- moves outward via conduction, the WD is able to expand during the explosion and, at lower densities, intermediate-mass elements are produced.  The explosion, however, produces slower velocity ejecta than seen in SNe Ia, and mixes burned and unburned material such that the ejecta do not appear stratified.  To resolve this, an \textit{ad-hoc} deflagration-to-detonation transition (DDT) is often invoked \citep{khok91}, the timing of which can be tuned to vary the amount of \Ni\ generated (eg. \citealt{hill+13}), though it remains to be seen this is a robust mechanism in realistic WD explosions (eg. \citealt{fishj15}).  It is also not obvious how invoking the DDT can explain the dependence of observed SNe Ia on the properties of their host galaxies, for example why more luminous SNe Ia tend to be in star-forming galaxies (eg. \citealt{hamu+00, howe+09, sull+10}).

The SD channel has a number of additional complications (and several others; see \citealt{maozns14, tsebs15}) that have led have the DD channel to fall into favor compared to the SD one.  For example, it requires a non-degenerate companion, which, under certain conditions, might be detectable, but attempts to spot the companion in pre-explosion archival data \citep{li+11cpn, nielvn13, niel+14}, during the supernova (as it responds to being hit by SN ejecta; \citealt{bloo+12,ollms15}), or after the explosion (eg. \citealt{kerz+14rem}).  For SD scenarios involving hydrogen-rich donors, the explosion is also expected to strip and entrain donor material, but attempts to find such material either do not detect hydrogen, or, in one recent case, apparently too little hydrogen to be consistent with SD donor stars \citep{magu+16}.  

%While these issues (and several others; see \citealt{maozns14}) have led the DD channel to fall into favor compared to the SD one, the issues discussed above affect \Mch\ progenitor scenarios in general.

% Mention gravitationally and pulsationally confined detonations??  Modern DDT: 2013MNRAS.429.1156S  GCD: http://adsabs.harvard.edu/abs/2016arXiv160600089S http://adsabs.harvard.edu/abs/2016ApJ...819..132G

\subsection{Brave New Channels}
\label{ssec:c1_new_typeia}

%In the following section, we will argue that the Chandrasekhar scenario shows significant problems when compared to observations of SNe Ia, and suggest that the majority of SNe Ia are caused by mergers of CO WD binaries, including those that produce remnants with masses below \Mch.  This argument is a summarization of the argument in \cite{vankerkwijk}.

The challenges posed by the evidence above has spurred research into alternative scenarios that lead to exploding CO WDs that are more physically viable and better fit observations.  Notably, several of these channels (highlighted below) relax the condition that the exploding WD must be at \Mch\ and allow for a range of exploding masses.  The alternate scenarios include:

\begin{itemize}

	\item The {\bf double-detonation channel} \citep{livn90, woosw94}, which involves a CO WD accreting a thin envelope from an He-rich source (a He star, merger with an He/hybrid WD, or even the He atmosphere of a CO WD).  At some point -- for slow ($\lesssim10^{-6}\,\Msun\,\pyr$; \citealt{bild+07}) accretion, when the base of the envelope becomes hot enough to ignite He burning; for mergers, when a hotspot in the accretion stream or envelope reaches conditions for detonation \citep{guil+10, rask+12, pakm+13} -- the He shell detonates, which then either drives compression waves that converge at the core of the CO WD to prompt a detonation, or launches one directly into it.  Either can trigger the secondary detonation of the CO WD (though the latter is perhaps more plausible; \citealt{mollw13}).  Traditionally, this has required massive He shells of $\gtrsim0.1\,\Msun$, but the detonation of these around a CO WD leads to significant production of \Ni, inconsistent with the stratified nature of SNe Ia ejecta (eg. \citealt{krom+10,woosk11}).  While the robustness and conditions most conducive to a double-detonation remains a field of active research (eg. \citealt{woosk11, holc+13, shenm14, shenb14, dan+15}), more recent work has suggested detonations of thin, $\lesssim0.05\,\Msun$ He shells in both stable accretion and mergers onto $\gtrsim1\,\Msun$ CO WDs may be possible \citep{woosk11, pakm+13, shenm14}, and that a CO detonation via converging shockwaves is likely as a result \citep{fink+10, mollw13, shenb14}.  Studies of pure detonations of bare sub-\Mch\ CO WDs with masses between $\sim1-1.15\,\Msun$ (eg. \citep{shig+92, sim+10}) show light curves and spectra that are in good agreement with normal SNe Ia (arguably as good or better than the agreement for other explosion models {\charles CITATION}), while the detonation of the much less massive He shell greatly reduces (but does not entirely eliminate) contamination from He shell nuclear ashes \citep{krom+10, hill+13}.

If the double-detonation channel is indeed physically plausible, it opens up a (somewhat narrow) range of WD masses beyond \Mch\ that can explode, and more naturally explains the explosion that requiren a deflagration-to-detonation transition.  Peak luminosity of the explosion is dependent on the mass of CO WD, which could naturally explain both the \cite{phil93} relation and the relationship between SN Ia luminosity and the age of its host stellar population (since lower-mass merger constituents take longer to form; \citealt{sim+10}).

	\item The {\bf violent merger channel} \citep{pakm+10}, which is a variant of the double-degenerate channel where, during the merging process, material being accreted from one WD to another is sufficiently superheated and compressed to trigger a detonation, destroying both stars.  This sidesteps the need for slow accretion following the merger.  Only those binaries where the primary WD is $\gtrsim0.9\,\Msun$ and the WD mass ratio is $\gtrsim0.8-0.9$ (\citealt{pakm+10, pakm+11, sato+16}, though see \citealt{dan+12}, which suggests a higher minimum primary WD of $\gtrsim1.0\,\Msun$) are violent enough to trigger a detonation, though the mass ratio criterion is reduced at higher primary masses \citep{sato+16}.  Mergers with primaries of $\gtrsim1.0\,\Msun$ produce explosions with \Ni\ yields consistent with normal or overluminous SNe Ia \citep{pakm+12, moll+14}.  Lower-mass systems produce ones that only synthesize $\sim0.1\,\Msun$ of \Ni, and are much redder and have much slower velocities than typical SNe Ia, but could resemble the subluminous SN 1991bg-like SNe Ia subclass \citep{lieb+93, pakm+10}.  Whether a CO detonation can robustly be triggered during a merger requires further study, but regardless, the CO WD binaries considered here are highly super-\Mch, and may be far too rare to explain the majority of SNe Ia \citep{badem12}.

	\item The {\bf direct collision channel}, in which two potentially sub-\Mch\ CO WDs collide, rather than merge, either because they are in a dense stellar environment (such as a globular cluster; \citealt{benzth89, loreig10}) or a hierarchical triple system \citep{katzd12} under the influence of the Kozai-Lidov mechanism \citep{koza62, lido62}.  Hydrodynamic simulations in \cite{kush+13}, \citep{garc+13} and \citep{dong+15} show that the impact of the the WDs leads to strong shocks that plow into both WDs and detonate them.  Explosion properties can be varied by changing impact parameter or the masses of the stars, with the collision of two $\sim0.6\,\Msun$ CO WDs is able to produce $\sim0.2-0.5\,\Msun$ of \Ni\ \citep{garc+13, kush+13}, consistent with normal SNe Ia.  A minority of WD reside in dense stellar environments, however, and only $\sim10-20$\% of stars are in triples, making it unlikely that this channel alone can reproduce a substantial fraction of SNe Ia (see Sec. 2.3 of \citealt{maozmn14} for details).

	\item The {\bf core-degenerate channel} \citep{livir03, kashs11, tsebs15}, which occurs when a progenitor system that would otherwise have produced a $>\Mch$ close-in double WD binary does \textit{not} survive its (second) common envelope phase (or survives in such a tight orbit that it merges within $10^6$ years afterward).  To account for most SNe Ia, the super-\Mch\ remnant would have to survive for $10^{8}-10^{10}\,\mrm{yr}$ before exploding (since SNe Ia can occur in stellar environments that have long ceased star formation; eg. \citealt{prichs08, maozsg10}).  \cite{illks12} proposes this can be done through a gradual loss of solid-body rotational support through magnetic dipole radiation, but this is not obviously physically robust (see post-merger evolution discussion in Ch. \ref{ch:ch2}).  Nonetheless, mergers during or just after common-envelope events naturally explain peculiar SNe Ia that show interaction with a large mass of circumstellar material (eg. PTF 11kx; \citealt{dild+12, soke13}).

\end{itemize}

More unconventional channels have also been proposed that allow a CO WD to detonate due to collisions with planets and planetoids \citep{distfg15} or even lone sub-\Mch\ WDs due to compositional impurities near their core that lower the density required for the onset of pycnonuclear fusion to $\gtrsim0.9\,\Msun$ \citep{chio+15}.  Further work is required to show the physical viability of these channels, and whether they reproduce normal SNe Ia.  

Regardless of whether or not any of these produce normal SNe Ia, study of these mechanisms is useful for explaining the diverse subclasses of SNe Ia.  Aside from the peculiar SNe listed above, synthetic light curves and spectra of pure deflagrations of \Mch\ CO WDs \citep{phil+07, krom+13, fink+14} reproduce the low peak brightness, slow ejecta velocity and hot photospheric emission of the SN IaX subclass (or SN 2002cx-like; \citealt{li+02, fole+13}).  Meanwhile, successful detonations of \Mch\ CO WDs have been linked to the overluminous SN 1991T-like SNe Ia (\citealt{fishj15}, but see \citealt{seit+16}).

%For one IaX, progenitor might have been found (2014Natur.512...54M), but no need to invoke bound remnant? (2015A&A...573A...2S)  Meanwhile, possible bound remnant found in another (2016MNRAS.tmp..977F).

\section{The vK10 SN Ia Channel}
\label{sec:c1_vkchannel}

Another alternate channel was recently proposed in \citeauthor{vkercj10} (\citeyear{vkercj10}; henceforth \citeal{vkercj10}), which considers the possibility that CO WD mergers with masses significantly below \Mch\ can also explode.  They consider a fiducial merger of a $0.6-0.6\,\Msun$ CO WD binary, whose masses are chosen to be near the empirical peak of the WD mass distribution (Sec. \ref{ssec:c1_cowd_massrange}).  Hydrodynamic simulations \citep{loreig09} find that the two WDs tidally destroy one another and coalesce into a remnant that is significantly heated throughout, but does not achieves temperatures sufficient to ignite fusion (as it is much less massive than the violent mergers considered in Sec. \ref{ssec:c1_new_typeia}); moreover, the remnant central density, $\sim2.5\times10^6\,\gcc$, is too low to produce \Ni\ in an explosion.  Following coalescence, however, the remnant, which is differentially rotating, enters a period of rapid angular momentum redistribution due to hydrodynamically or magnetically-mediated viscosity.  Using the standard $\alpha$-viscosity prescription \cite{shaks73} -- i.e. $\nu = \alpha c_s H_P$, where $c_s$ is the sound speed and $H_P$ the pressure scale height -- the timescale for viscous evolution can be estimated as

\begin{eqnarray}
t_\mrm{visc} &=& \frac{R_\mrm{disk}^2}{\nu} \sim \frac{1}{\alpha}\frac{R_\mrm{disk}^2}{H_P^2}\taudyn \nonumber \\
			&\sim& 3\times10^4\,\mrm{s}\left(\frac{10^{-2}}{\alpha}\right)\left(\frac{R_\mrm{disk}/H_P}{10}\right)^2\left(\frac{R_\mrm{disk}}{10^9\,\mrm{cm}}\right)^{3/2}\left(\frac{M_\mrm{enc}}{1\,\Msun}\right)^{-1/2},
\end{eqnarray}

\noindent where $M_\mrm{enc}$ is we have used $\taudyn \approx H_P/c_s$ and inserted a fiducial viscosity and typical numbers for remnants \citep{shen+12}.  Thus the vast majority of the remnant's angular momentum is transported away, and the remnant (including its disk) loses its rotational support against gravity, over a period $\sim10^4\,\mrm{s}$.\footnote{This is notably in contrast to earlier work (eg. \citep{nomoi85, yoonpr07}) that assume any rotationally-supported material will slowly accrete onto the dense core of the remnant at a near-Eddington mass accretion rate of $\dot{M} \sim 10^{-5}\,\Msun\,\pyr$.  Remnants are prone to magnetic instability \citep{shen+12,ji+13}, and will almost certainly evolve over the much shorter timescale given by the $\alpha$-viscosity estimate.}  This loss of rotational support combined with increasing weight from newly accreted disk material leads to compression and heating of the remnant core.  Since $\sim10^4\,\mrm{s}$ is far shorter than either the neutrino cooling timescale of $\taunu \sim 10^3\,\mrm{yr}$ or the thermal adjustment timescale of $\sim10^4\,\mrm{yr}$ \citep{shen+12}, compressional heating is adiabatic, and \citeal{vkercj10} estimates that for the $0.6-0.6\,\Msun$ remnant it leads both the central density and temperature to increase to $\gtrsim1.5\times10^7\,\gcc$ and $\gtrsim10^9\,\mrm{K}$, at which point the nuclear fusion timescale is smaller than even the compressional heating timescale, and a carbon nuclear runaway becomes inevitable.  If the runaway leads to dynamical burning and an explosion occurs, the generally lower densities of merger remnants compared to \Mch\ WDs (at $\sim3\times10^9\,\gcc$) means that burning leads to larger pressure differential between the ashes and their surroundings, perhaps making a detonation favorable over a deflagration (e.g. \citealt{mazumw77}; \citealt{seit+09}).

The \citeal{vkercj10} scenario features a number of advantages over the traditional \Mch\ double-degenerate channel.  Like the double-detonation channel, this one substantially increases the number of binary systems that could potentially explode -- perhaps by a factor of $\sim3$, which would be much more consistent with the SN Ia rate; \citeal{vkercj10}; \citealt{badem12} -- and naturally explains both the explosion mechanism and some of the trends seen in the observed SN Ia population, thus alleviating the issues discussed in Sec. \ref{ssec:c1_old_typeia}.  Unlike the double-detonation channel, however, a He detonation is not invoked to trigger the CO WD to explode -- negating the complications of the He detonation ashes on the SN light curve -- and the merger process perhaps provides a more natural means of producing the $\gtrsim1\,\Msun$ of CO needed to synthesize the \Ni\ found in a typical SN Ia (eg. \citealt{pirotk14}).

For this scenario to proceed as described, the merger remnant must be heated throughout its interior, which only occurs for mergers of nearly equal-mass binaries.  In a scenario where very unequal masses merge, the less massive secondary WD is completely disrupted and forms a disk and envelope around a largely undisturbed primary \citep{loreig09}.  Carbon ignition is then likely to occur off-center, which could lead to stable carbon burning rather than an explosion (eg. \citealt{yoonpr07, shen+12}).

\section{Hot DQs, High-Field Magnetic WDs, and Other Oddities}
\label{sec:c1_hotdqs}

{\charles If the remnant of a double CO WD merger is not eventually (at least partially) destroyed in an explosion, it will go on to 

%http://adsabs.harvard.edu/abs/2016ApJ...817...27W

Stable nuclear burning, however, }

%The merger of two white dwarfs (WDs) originally in a short-period binary is estimated (eg. \citealt{badem12}) to occur about once every century in a Milky Way-like galaxy, making the products of such events common throughout the universe.  They have been held responsible for producing a variety of stars with strange properties, including helium-burning sdOB stars \citep{saioj00, justph11}, RCrB stars (eg. \citealt{webb84, clay+07, clay13}), and massive and highly magnetized WDs (eg. \citealt{segrcm97, garc+12, kule+13}) that could resemble the hot DQ WDs (eg. \citealt{dunlc15}), Dunlap and Clements in preparation).  They may, however, also be responsible for spectacular transient events including accretion-induced collapses (eg. \citealt{saion85, abdi+10}) and type Ia supernovae (SNe Ia; eg. \citealt{howe11, hill+13, maozmn14}).  Determining the final outcome of a particular merger requires an understanding of the detailed dynamics of the merging process, which cannot directly be seen using current observational capabilities.  Thus, studies of merger physics have primarily utilized hydrodynamic simulations.

-Searches for massive remnants have found single instances, possibly statistical bump near 1.2 Msun
-

\section{Thesis Overview}

If sub-\Mch\ double CO WD mergers can either eventually explode or leave behind isolated massive CO/ONe WDs, which fate is preferred, and does that fate depend on the masses of the two WDs?  Can these mergers, as \citeal{vkercj10} claims, serve as a novel SNe Ia progenitor channel that naturally satisfies both explosion, rates and properties?  To answer these questions, we must put the \citeal{vkercj10} scenario under scrutiny, and understand the parameter space of possible merger remnants produced by double CO WDs -- in particular the fraction of them that have temperatures close to carbon ignition under highly degenerate conditions, how the subsequent viscous phase proceeds in detail, and whether carbon ignition in a remnant can lead to an explosion.  This thesis attempts to shed light on some of these issues through a series of theoretical studies utilizing both semi-analytical calculations and hydrodynamic and magnetohydrodynamic computer simulations.

%Thus, the \citeal{vkercj10} channel is both attractive for its many advantages and plausible given the order-of-magnitude estimates above.  Investigating whether these estimates hold under detailed scrutiny, and to determine which, if any, systems in the CO WD binary parameter space could follow the channel, is the purpose of this PhD thesis.

In Chapter \ref{ch:ch2}, I use a battery of 3D merger simulations to characterize the range of possible merger remnant configurations arising from merging CO WD binaries.  As we are particularly interested in those mergers where 

As we discussed in Sec. \ref{sec:c1_vkchannel} and \ref{sec:c1_hotdqs}, the merger of two carbon-oxygen white dwarfs can lead either to a spectacular transient, stable nuclear burning or a massive, rapidly rotating white dwarf.  Previous simulations of mergers have shown that the outcome strongly depends on whether the white dwarfs are similar or dissimilar in mass \citep{loreig09}.  In the similar-mass case, both white dwarfs merge fully and the remnant is hot throughout, while in the dissimilar case, the more massive, denser white dwarf remains cold and essentially intact, with the disrupted lower mass one wrapped around it in a hot envelope and disk.

In order to determine what constitutes ``similar in mass'' and more generally how the properties of the merger remnant depend on the input masses, we simulated unsynchronized carbon-oxygen white dwarf mergers for a large range of masses using smoothed-particle hydrodynamics.  We find that the structure of the merger remnant varies smoothly as a function of the ratio of the central densities of the two white dwarfs.  A density ratio of 0.6 approximately separates similar and dissimilar mass mergers.  Confirming previous work, we find that the temperatures of most merger remnants are not high enough to immediately ignite carbon fusion.  During subsequent viscous evolution, however, the interior will likely be compressed and heated as the disk accretes and the remnant spins down.  We find from simple estimates that this evolution can lead to ignition for many remnants.  For similar-mass mergers, this would likely occur under sufficiently degenerate conditions that a thermonuclear runaway would ensue.

Aside from redundant parts of the introduction, we also do not reproduce here the extensive Appendix to \citeal{zhu+13}, which contains tables of binary input parameters and remnant properties for the simulations.



In Chapter \ref{ch:ch3}


Important questions about the channel, however, remain, including what fraction of mergers leads to remnants that are are raised to temperatures close to carbon ignition under highly degenerate conditions, how the subsequent viscous phase proceeds in detail, and whether ignition leads to a detonation, and whether the detonation of a remant that may still rotate and be surrounded by a disk would produce an event similar to an SN Ia.  

The chapters are ordered in accordance with the proposed evolution of the \citeal{vkercj10} channel.  In Chapter \ref{ch:ch2}, we consider the range of possible merger remnant configurations to arise from the parameter space of merging CO WD binaries.  In Chapter \ref{ch:ch3}

For the most part, I have reproduced exactly the texts of \citeal{zhu+13}, \citeal{zhu+15} and \citeal{zhu+16} in their respective chapters.  The exceptions are the chapter introductions, where I have excised certain paragraphs to eliminate the redundancy of having multiple paragraphs repeating an overview of the \citeal{vkercj10} channel.  The papers' abstracts have also been modified into chapter overviews, and certain figures reformatted for readability.  Any additional changes are noted at the start of each chapter.  Most prominently, we have added a postscript to Chapter \ref{ch:ch2} (Sec. \ref{sec:c2_postscript}) that considers our simple semi-analytical prescription for post-merger viscous evolution in light of new results, and we included an extensive appendix to Chapter \ref{ch:ch5} that details our calculation of convective suppression in magnetized, rotating WDs.

