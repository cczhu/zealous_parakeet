\section{The Physics of Pre-Merger Evolution}

To preface Chapter \ref{ch:ch2}, which focuses on the merger remnants, we briefly cover a few topics related to evolution of the CO WD binary leading up to mass transfer, and the early stages of mass transfer leading up to the merger proper (which we refer to as ``coalescence'').

\subsection{The CO WD Mass Range}

\subsection{Stable and Unstable Mass Transfer}
\label{ssec:stable_mass_transfer}

%http://adsabs.harvard.edu/abs/2016arXiv160404269B
%http://adsabs.harvard.edu/abs/2015ApJ...805L...6S

All of the binary systems we simulate experience unstable mass transfer and merge.  This is a consequence of our use of approximate initial conditions that place two unperturbed, spherical WDs at an initial separation where the less massive WD's radius is equal to its Roche lobe (the radius around a star within which any material is gravitational bound to the star).  Note that the less massive WD is always the first to initiate mass transfer, since by the WD mass-radius relation,

\eqbegin
R \propto M^{\frac{1 - n}{3 - n}} \approx M^{-1/3}
\label{eq:c1_massradiusrelation}
\eqend

\noindent (approximating the low-mass cold WD EOS with an $n = 3/2$ polytrope), it is larger and therefore overflows its Roche lobe first.  Once placed in a binary, the WDs deform and radially pulsate in response to the new potential, resulting in an overestimate of the rate of early mass transfer as material from the less massive WD overshoots the Roche lobe during each pulsation.  Our simulations therefore do not predict which binary WDs can experience unstable mass transfer and merge.

A simple analytic estimate would suggest that binaries with sufficiently small mass ratios would not merge at all.  Stability depends critically on the mass ratio $\qm = \Md/\Md$ between the donor star (\Md) and the accretor star (\Ma; from above, $\Ma \geq \Md$ and $\qm \leq 1$ for all mergers), and whether or not spin and orbital angular momentum can be efficiently coupled to each other.  We sketch a simple argument for stability below; a more accurate calculation can be found in \citep{marsns04}.

Let us consider the case in which some dissipative process, such as tides, is able to return spin angular momentum to orbital angular momentum, thus helping to stabilize mass transfer, meaning that (on the timescales of the merger) $\dot{J}_\mrm{orb} = 0$.  The orbital angular momentum of a binary system is $J_\mrm{orb} = (\Ma\Md/\Mtot)\sqrt{G\Mtot a}$, where $\Mtot = \Md + \Ma$ and $a$ is the orbital separation.  From this we may derive $\dot{J}_\mrm{orb}/J_\mrm{orb} = \dot{\Md}/\Md + \dot{\Ma}/\Ma - \dot{M}_\mrm{tot}/2\Mtot + \dot{a}/2a$.  We shall assume conservative mass transfer (this is backed by the simulations in Ch. \ref{ch:ch2} and \ref{ch:ch3}, which show less than $1$\% of stellar material becomes unbound), meaning $\dot{\Ma} = -\dot{\Md}$.  Putting this together gives us

\eqbegin
\dot{J}_\mrm{orb}/J_\mrm{orb} = 2(q - 1)\frac{\dot{\Md}}{\Md} - \frac{\dot{a}}{a},
\label{eq:c1_adotovera}
\eqend

\noindent noting that $\dot{J}_\mrm{orb}/J_\mrm{orb} = 0$.  We use Paczynski's estimate for the Roche lobe of \Md, $R_L \approx 0.46a(\Md/\Mtot)^{1/3}$, valid for $\qm \lesssim 1$ \citep{eggl83}.  Differentiating and using Eqn. \ref{eq:c1_adotovera}, we obtain

\eqbegin
\frac{\dot{R}_\mrm{L}}{R_\mrm{L}} = 2(q - \frac{5}{6})\frac{\dot{\Md}}{\Md}.
\label{eq:c1_rochedotoverroche}
\eqend

\noindent For mass transfer to be stable, \Rd\ must expand more slowly than $R_L$, or $\dot{R}_\mrm{d}/\Rd < \dot{R}_\mrm{L}/R_\mrm{L}$.  From Eqn. \ref{eq:c1_massradiusrelation}, $\dot{R}_\mrm{d}/\Rd = -\dot{\Md}/3\Md$; combining this with Eqn. \ref{eq:c1_rochedotoverroche}, we obtain the stability criterion $2(q - 5/6) < -1/3$, or:

\eqbegin
q < \frac{2}{3}.
\label{eq:c1_qcrit}
\eqend

In the case where spin and orbital angular momentum coupling is instead negligible, a similar analysis can be peformed, using total angular momentum $J = J_\mrm{orb} + J_{spin} = (\Ma\Md/M)\sqrt{GMa} + J_{spin}$, from which we may derive $\dot{J}/J_\mrm{orb} = \dot{\Ma}/\Ma + \dot{\Md}/\Md - \dot{M}/2M + \dot{a}/2a + \dot{J}_{spin}/J_\mrm{orb}$.  Following \cite{marsns04} and \cite{nele+01}, we assume only the spin of the accretor matters for the equation, and follow \cite{verbr88}'s representation of the spin-up of the accretor from direct impact accretion with $\dot{J}_{spin} = -\sqrt{G\Ma R_h}\dot{\Md}$ ($\dot{\Md}$ is negative).  $R_h$ is the effective radius of the matter transferred onto the accretor, and the ratio $r_h = R_h/a$ is given by a fitting formula: $r_h = 0.0883 - 0.04858\log({q}) + 0.11489{\log}^2(q) + 0.020475{\log}^3(q)$, valid for all plausible WD binary mass ratios \citep{verbr88}.  Dividing this by $J_\mrm{orb}$ to obtain $\dot{J}_{spin}/J_\mrm{orb} = -\sqrt{(1 + q)r_h}\dot{\Md}/\Md$, and assuming conservative mass transfer ($\dot{J} = 0$), we obtain \citep{marsns04,nele+01}

\eqbegin
\frac{\dot{a}}{a} = 2\left(q - 1 + \sqrt{(1 + q)r_h}\right)\frac{\dot{\Md}}{\Md}.
\label{eq:c1_adotovera2}
\eqend

\noindent Using the same argument that gave us Eqn. \ref{eq:c1_qcrit},

\eqbegin
2(q - \frac{5}{6} + \sqrt{(1 + q)r_h}) < - \frac{1}{3},
\label{eq:c1_qcrit2}
\eqend

\noindent which, solved numerically, is

\eqbegin
q \lesssim 0.219
\label{eq:c1_qcrit3}
\eqend

%\noindent Fig \ref{stabilityfig} from \cite{dan+11} summarizes this, indicating the zone of guaranteed instability given by Ineq. \ref{qcrit}, and the boundary between direct impact accretion and disk accretion (calculated from \cite{nele+01}).  Disk accretion results in good coupling between orbital and spin angular momenta, and since the region falls below Ineq. \ref{qcrit}, it is definitely stable.  (See section 4.5 of \citeauthor{marsns04}, however, for evidence that disk accretion does not stabilize mass transfer at all; the fact that disk accretion systems are stable would then be largely due to Ineq. \ref{qcrit2}.)  This leaves a large region of direct impact accretion in the middle of the two stability regions that may or may not be stable - if no coupling returns spin angular momentum to orbital angular momentum, most of the region should be unstable, as given by Ineq. \ref{qcrit2}.  \cite{marsns04} find that the stability in the middle region is dependent primarily on the synchronization timescale of the binary system, a conclusion supported by \cite{gokhpf07}'s stability analysis\footnote{Another fact that needs to be considered is that even in unstable mass transfer, if the donor can survive long enough for $q$ to drop to a value conducive to stable mass transfer, ultimately the system does not merge \citep{gokhpf07}}.  Unfortunately tidal interaction in WD binaries is not well understood (see Sec. \ref{ssec:synchronization}).

%A number of numerical studies have been performed to explore mass ratios of uncertain mass transfer stability, and the results have not been conclusive \citep{mars11}.  For example, \cite{motl+07} used a (grid-based) self-consistent field method to determine that for an initial mass ratio $q_{init} = 0.4$ there is enough spin-orbit momentum coupling to ensure stability.  On the other hand SPH simulations by \cite{dan+11} using the Helmholtz equation of state yielded a merger for $q = 0.4$, and have shown unstable mass transfer for binaries down to $q = 0.25$.  \citeauthor{dan+11} also show that for such mergers a careful construction of the initial conditions results in a long period of mass transfer (several tens of orbital periods) before full disruption of the donor, suggesting that their results can be compatible with the results from \citep{motl+07}.  For the purposes of this paper, we will consider an extreme mass ratio merger to be a plausible (if rare, given WD mass distribution statistics) occurence.

%It should be noted that \cite{dan+11} also show that merger simulations are sensitive to initial conditions, and point out that the approximate initial conditions of other simulation groups such as \citeauthor{loreig09}, \citeauthor{guerig04} and \cite{pakm+10} may distort their findings.  This not only applies to merger stability and merger timescales, but also central temperatures of post-merger remnants.

\subsection{Are Merging WDs Co-Rotating?}
\label{ssec:synchronization}

%Estimates of the synchronization timescale in binary systems give $\tau_{S} \sim 10^{12}$ yr from radiative damping, and $\tau_{S} \sim 10^{15}$ yr from viscosity \citep{marsns04}.  To compare, the timescale for angular momentum loss from gravitational wave radiation is \citep{segrcm97}

%\eqbegin
%\tau_{\mathrm{grav}} = 5 \times 10^5 \left( \frac{a}{10^5 \mathrm{km}}\right)^4 \frac{M_{\odot}}{\Ma} \frac{M_{\odot}}{\Md} \frac{M_{\odot}}{\Ma + \Md} \mathrm{ yr}.
%\label{gravtimescale}
%\eqend

%\noindent In the latter stages of evolution, this value is around $\tau_{S} \sim 10^{6}$ yr.  It is then likely that neither donor nor accretor are synchronized at the time of merger\footnote{A close-in WD binary should merger within $10^8$ to $10^9$ yrs after formation \cite{segrcm97}.  Of course, any transients caused by mergers seen today must have occured within this time!}.  Turbulent viscosity and non-radial mode excitation, on the other hand, can potentially have $\tau_{S} << 500$ yr, and even small magnetic fields, properly oriented, can significantly enhance viscosity \citep{marsns04,ibentf98}.  Also, the viscous timescale scales as $a^6$, while Eqn. \ref{gravtimescale} scales as $a^4$; this indicates that should viscosity ever synchronize a WD binary, this binary will be synchronized for the remainder of its inspiral \citep{ibentf98}.  Whether or not a binary will be synchronized is still largely an unsolved problem \citep{mars11}.

%If we were to suppose a WD system could synchronize, then viscous dissipation should heat up both WDs significantly.  \cite{ibentf98} perform long equal-mass binary evolution calculations that assumes the binary system is synchronized, and the rate of tidal heating is equal to the rate of spin kinetic energy increase. (SO DOES IT SAP ROTATION????)  They find that over the course of the last $10^4$ yrs before merger heating from synchronization can increase the temperature of a 1.0 {\Msun} (with a 1.0 {\Msun} companion) by an order of magnitude.  While most of the thermal energy that is radiated away during this heat-up is in the form of neutrinos, the EM luminosity increases by almost five orders of magnitude.  At the time of merger, each 1.0 {\Msun} WD would shine with $\sim 100$ L$_{\odot}$ and have a temperature of $\sim 10^8$ K, making the system a significant X-ray source.  The luminosity just before merger increases with WD mass, and the period of time over which luminosity increase occurs drops with WD mass: a 0.3 {\Msun} with an equal-mass companion will increase in luminosity by two orders of magnitude over $10^6$ yr.  If the efficiency by which rotational energy is converted to thermal energy is reduced to 10\% the maximum efficiency in the 1.0-1.0 {\Msun} binary, the final luminosity drops by a factor of about 1000.  In all cases simulated, \citeauthor{ibentf98} found temperatures were insufficient to ignite nuclear fusion.  The periods of increased luminosity are all in the range of $10^2 - 10^6$ years, which, while short compared to the liftime of the binary, are far too long to be considered transients.

%While this question is something that we shall discuss at length in Ch. \ref{ch:ch2} (and is also considered in other works such as \cite{XXX} and \cite{ji+13}) the results of Ch. \ref{ch:ch3} suggest that it is likely unimportant for the configuration of the merger remnant (at least to first order).

%Check Schwab+12 intro for synchronization links.


%http://adsabs.harvard.edu/abs/2015ApJ...815...63A
%http://adsabs.harvard.edu/abs/2016ApJ...824...46B
%https://arxiv.org/abs/1606.05292

%http://adsabs.harvard.edu/abs/2015arXiv151203442P
%http://adsabs.harvard.edu/abs/2016MNRAS.tmp..977F



%http://adsabs.harvard.edu/abs/2016ApJ...817...27W

%http://adsabs.harvard.edu/abs/2016ApJ...818...26D

%http://adsabs.harvard.edu/abs/2016arXiv160401021F
%http://adsabs.harvard.edu/abs/2016ApJ...818L..19M
