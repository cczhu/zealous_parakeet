\chapter{Conclusion}

\section{Implications for SN Ia Observations}

\section{The Influence of Merger Remnants Properties on Potential Explosions}

If sub-\Mch\ CO WD merger remnants indeed trigger thermonuclear detonations following post-merger viscous evolution, will these explosions resemble SNe Ia?  As noted in the introduction, SNe Ia observations and radiative transfer models for explosions are now sophisticated enough to distinguish fine details between different progenitors.  This question has been taken up by a number of theorists \citep{frye+10, shen+12}

\cite{frye+10} and \cite{rask+14} do simulations

\begin{itemize}
	\item What are the mass loss rates of non-explosive mergers due to carbon dust superwind?  can we estimate?  could it possibly lead to Kepler's 0.6Msun oxygen WD? Shen+12 (Just under Eqn. 7) note that mass loss in radiation-dominated H/He-deficient envelopes is poor. %http://adsabs.harvard.edu/abs/2016Sci...352...67K
	\item Near-eddington carbon burning star?  See Sec. 4 paragraph 1 of Shen+12.
	\item Discussion with Marten: while we can't rule out further compression leading to nuclear burning during the thermal evolution phase, a combination of magnetically and Eddington-launched outflows will make the surroundings very messy - write more about this in SN Ia appearance stuff.
\end{itemize}
