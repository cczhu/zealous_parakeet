\section{Codes and Initial Conditions}
\label{sec:c3_codes}

All hydrodynamic and magnetohydrodynamic codes seek to properly evolve the continuum dynamics of a fluid on a discrete set of points in space and time.  Most astrophysical fluid codes (including the two we use) explore the simpler regimes of ideal hydro- or magnetohydrodynamics, where molecular viscosity and electrical resistance are negligible -- generally the case astrophysical settings outside of planetary interior dynamics.  The coupled partial differential equations of ideal magnetohydrodynamics, in their conservative form and with Gaussian units (eg. \citealt{goedp04, pakms13, spru13}, \citealt{feidc12} Sec. 3), is

\begin{eqnarray}
\ptl_t\rho + \ptl_j(\rho u^j) &=& 0 \nonumber \\
\ptl_t(\rho u^i) + \ptl_j(\rho u^iu^j + \delta^{ij}P_\mrm{tot} - \frac{1}{4\pi}B^iB^j) &=& \rho \ptl^i\Phi \nonumber \\
\ptl_t(\rho e) + \ptl^j\left(u_j (\rho e + P_\mrm{tot}) - \frac{B_j}{4\pi}(u^lB_l)\right) &=& u_j\rho \ptl^j\Phi \nonumber \\
\ptl_t B^i - \ptl_j(u^iB^j - u^jB^i) &=& 0,
\label{eqn:c3_mhd_eqns}
\end{eqnarray}

\noindent where $\rho$, $u^i$, $B^i$, and $\Phi$ are the density, velocity, magnetic field and gravitational potential, respectively, $P_\mathrm{tot} = P + \frac{1}{8\pi}B_jB^j$ is the total pressure, $e = \frac{1}{2}u_iu^i + e_\mathrm{int} + \frac{B^2}{8\pi\rho}$ is the specific total energy, and the usual Einstein summation convention holds.  This can be written in compact form:

\eqbegin
\frac{\ptl {\bf U}}{\ptl t} + \nabla\cdot{\bf F(U)} = {\bf G}
\label{eqn:c3_mhd_eqns_compact}
\eqend

\noindent where

\eqbegin
{\bf U} = 
\left( \begin{array}{c}
\rho \\
\rho {\bf u} \\
\rho e \\
{\bf B} \end{array} \right),
\label{eqn:c3_mhd_eqns_u}
\eqend

\eqbegin
{\bf F(U)} = 
\left( \begin{array}{c}
\rho {\bf u} \\
\rho {\bf u}{\bf u}^T + P_\mathrm{tot} - \frac{1}{4\pi}{\bf B}{\bf B}^T \\
{\bf u} (e + P_\mathrm{tot}) - \frac{{\bf B}}{4\pi}({\bf u}\cdot{\bf B}) \\
{\bf B}{\bf u}^T - {\bf u}{\bf B}^T\end{array} \right),
\label{eqn:c3_mhd_eqns_fu}
\eqend

\eqbegin
{\bf G} = 
\left( \begin{array}{c}
0 \\
\rho {\bf g} \\
\rho {\bf u}\cdot {\bf g} \\
0 \end{array} \right)
\label{eqn:c3_mhd_eqns_g}.
\eqend

\noindent Eqn. \ref{eqn:c3_mhd_eqns_compact} shows that the time-derivative of the fluid's values is given by the sum of a flux term (since, by Gauss's Law, the integral of $\nabla\cdot{\bf F(U)}$ within a volume is equivalent to a flux across its boundary) and a (self-)gravitational source term ${\bf G}$.  These are generally calculated separately, and then combined.

To better understand the code comparison in this chapter, and to preface the discussion of improving \arepo's angular momentum conservation in Sec. \ref{sec:c3_fixingarepo}, we present an extremely short and mostly qualitative discussion of how these equations are implemented within SPH and \arepo.  The historical development of both methods is long and involved, and, as improving hydrodynamic simulations is not the focus of this thesis, we will refer the reader to a number review articles referenced throughout this section for further details.

\subsection{Traditional Smoothed-Particle Hydrodynamics}
\label{ssec:c3_sph}

SPH, first introduced in \cite{lucy77} and \cite{gingm77}, is a mature simulation method used in a host of astrophysical contexts ranging from star formation to cosmology.  Our overview summarizes the first few sections of \citep{spri10rev}; we also refer readers to \citep{mona05} and \cite{ross09} for further details.

SPH represents a fluid with a set of particles.  The fluid's continuum properties at some point ${\bf r}$ in the simulation are sampled by using these particles as interpolation points.  Representing any given continuum property (the most important of which is density, since it factors into the equations of motion) as $F({\bf r})$, we can use a ``kernel'' $W({\bf r}, h)$ to generate its approximate, locally-averaged value

\eqbegin
F_s({\bf r}) = \int F({\bf r'})W({\bf r} - {\bf r'}, h)d{\bf r'}.
\eqend

\noindent In the (computationally impossible) case of infinite resolution, we can choose $W({\bf r}, h)$ to be a Dirac delta, and $F_s({\bf r}) = F({\bf r})$, but in practice we choose $W({\bf r}, h)$ to extend over some characteristic ``smoothing length'' $h$.  If $W({\bf r}, h)$ were a Gaussian, the $h = \sigma$, the standard deviation.  The most popular form of $W({\bf r}, h)$ is a cubic spline that goes to zero when ${\bf r} > 2h$, and $h$ is generally set to ensure a user-defined number of neighboring particles $N$ fall within the kernel.  For a set of particles with associated mass $m_i$ and known values of $F_i = F({\bf r}_i)$, we can discretize the integral as

\eqbegin
F_s({\bf r}) = \sum_j\frac{m_j}{\rho_j}F_j W({\bf r} - {\bf r}_j), h).
\label{eq:c3_kernelavg}
\eqend

\noindent where $\rho_i$ can be estimated using $\rho_i = \sum_j m_j W({\bf r}_i - {\bf r}_j), h)$.  Derivatives of the field can also be determined using the gradient of the kernel $\nabla_i W_{ij}$.

Meanwhile, the Euler equations (Eqn. \ref{eqn:c3_mhd_eqns_compact} without the gravitational and magnetic terms) can be shown to follow the Lagrangian:

\eqbegin
L = \int \rho\left(\frac{{\bf u}^2}{2} - e\right)dV
\label{eqn:c3_lagrangian}
\eqend

\noindent which can be discretized for a set of particles as

\eqbegin
L_\mathrm{SPH} = \sum_i \frac{1}{2}m_i u_i^2 - m_i e_i.
\eqend

\noindent This suggests a time-evolution scheme for the fluid.  Each particle representing the fluid is given a (time-independent) mass $m_i$, position ${\bf r}_i$, velocity ${\bf u}_i$ and specific internal energy $e_i$; the fluid can then be simulated by time-evolving the latter three terms for all particles.  The equations governing the latter two are derived by applying the Euler-Lagrange equation ($\frac{d}{dt}\frac{\ptl L}{\ptl \dot{{\bf r}}_i} - \frac{\ptl L}{\ptl {\bf r}_i}$) to $L_\mathrm{SPH}$.  They traditionally takes the form:\footnote{We state \cite{wadssq04}'s formulation of Eqns. \ref{eqn:c3_spheqnm} and \ref{eqn:c3_spheqne}, as \cite{spri10rev} assumes a different method of controlling the smoothing length $h_i$.}

\begin{eqnarray}
\frac{d{\bf u}_i}{dt} &=& -\sum_j m_j\left(\frac{P_i}{\rho_i^2} + \frac{P_j}{\rho_j^2} \right)\nabla_i W_{ij} \\
\label{eqn:c3_spheqnm}
\frac{d e_i}{dt} &=& -\frac{P_i}{\rho_i^2}\sum_j m_j\left({\bf u}_{i} - {\bf u}_{j}\right)\cdot\nabla_i W_{ij}
\label{eqn:c3_spheqne}
\end{eqnarray}

\noindent where pressure $P_i$ is determined from $\rho_i$ and $e_i$ using a user-prescribed equation of state.  Note that since Eqn. \ref{eqn:c3_lagrangian} has no time-dependence and is translationally and rotationally invariant, SPH naturally conserves total energy, momentum and angular momentum.  Self-gravity can be added as an additional force to Eqn. \ref{eqn:c3_spheqnm} (see \citealt{spri10rev} Sec. 2.4 and \citealt{wadssq04} Sec. 2.1) using methods originally developed for $N$-body simulations.  Magnetic fields can also be included (eg. \citealt{pric12}), but the resulting ``SPHMHD'' formulation is not used in this thesis.

% Eqn. \ref{eqn:c3_spheqne} can actually be replaced with $d s_i/dt = 0$, except in the presence of shocks

As given, the SPH equations of motion conserve entropy, but entropy must increase in the presence of shocks (as the differential form of the Euler equations breaks down across a shock front).  The most popular solution is to include an artificial viscosity term

\eqbegin
-\sum_j m_j \pi_{ij} \nabla_i W_{ij}
\eqend

\noindent to Eqn. \ref{eqn:c3_spheqnm} and \ref{eqn:c3_spheqne}, where, defining ${\bf r}_{ij} = {\bf r}_i - {\bf r}_j$ and ${\bf u}_{ij} = {\bf u}_i - {\bf u}_j$,

\eqbegin
\pi_{ij} =
    \begin{cases}
      \frac{-\alpha\frac{1}{2}(c_i + c_j)\mu_{ij} + \beta\mu_{ij}^2}{\frac{1}{2}(\rho_i + \rho_j)} & \mrm{for\,} {\bf u}_{ij}\cdot{\bf r}_{ij} < 0 \\
      0 & \mrm{otherwise},
    \end{cases}
\label{eq:c3_artificialvisc}
\eqend

\noindent where $\mu_{ij} = \bar{h}{\bf u}_{ij}\cdot{\bf r}_{ij}/(|{\bf r}_{ij}|^2 + 0.04\bar{h}^2)$, $\bar{h} = \frac{1}{2}(h_i + h_j)$, $c_i$ is the sound speed and $\alpha$ and $\beta$ are tunable parameters ($\beta = 2\alpha$ is used in \gasoline).  In addition to facilitating shock capture, $\pi_{ij}$ also prevents spurious particle interpenetration between colliding flows \citep{hernk89}.  It, however, can also introduce spurious viscous forces outside of shocks.  In shear flows, artificial viscosity can be damped by a ``Balsara switch'', which multiples $\pi_{ij}$ with a prefactor, proportional to the ratio between the divergence and curl of velocity, that goes to zero in the presence of a pure shear flow.  It is also possible to make $\alpha$ and $\beta$ coefficients time-variable (eg. \citealt{morrm97, dola+05}) with

\eqbegin
\frac{d\alpha_i}{dt} = -\frac{\alpha_i - \alpha_\mrm{min}}{\tau} + S_i
\eqend

\noindent where timescale $\tau = h_i/(c_i l)$, $l$ is a tuneable parameter of order unity, $S_i$ is a source term that becomes large in the prescence of shocks, and $\alpha_\mrm{min} > 0$ is a minimum $\alpha$ value to prevent noise buildup and spurious particle interpenetration in smooth flows.  Both these methods can be used in \gasoline.

%{eq:c3_artificialvisc}

As explained in the introduction, SPH's Lagrangian nature allows it to automatically resolve regions of high density, simulate advection without errors, and conserve energy, linear and angular momentum to high accuracy.  These features make it much easier to model mergers in SPH than in Eulerian grid schemes, which discretize the simulation volume on a static grid, and time-evolve the system by tracking fluid fluxes between grid cells.  These have traditionally had issues with simulating advection and tracking orbiting binaries except under specific coordinate systems and symmetries (eg. \citep{hopk15}), and thus have very infrequently been used to simulate WD mergers (see \cite{katz+16}, however for recent developments).

The limitations of SPH (and Eulerian codes) have also been well-covered in literature (see, eg., the introductions to \citealt{spri10, hopk15, katz+16}).  Chief among them is the artificial viscosity discussed above, which can produce spurious heating and angular momentum transport in shear flows even in codes that utilize the Balsara switch and time-variable viscosity \citep{culld10}.  Classical formulations of SPH have also been known to suppress hydrodynamic instabilties (eg. \citealt{ager+07}) due to poor treatment of contact discontinuities, manifesting as a ``surface tension'' (eg. \citealt{readha10, hesss10}).  While this has subsequently been resolved (eg. \citep{hopk13, hu+14, kell+14}) by introducing artificial mixing terms \citep{pric08} and smoothing the pressure as well as the density across discontinuities (eg. by replacing the $P_i/\rho_i^2 + P_j/\rho_j^2$ term in Eqn. \ref{eqn:c3_spheqnm} with $(P_i + P_j)/(\rho_i\rho_j)$; \citealt{kell+14}) the vast majority of merger simulations in the literature come from before these alterations became widely used.  SPH has poor shock and steep gradient resolution compared to Eulerian schemes due to kernel smoothing of the density, and can corrupt smooth flows with particle velocity noise \citep{spri10rev}.  Lastly, it suffers from a resolution-independent ``$E_0$'' error in Eqn. \ref{eqn:c3_spheqnm} (which also contributes to poor treatment of discontinuities; eg. \citealt{readha10}) related to the choice of kernel, which can produce enough noise to drown out large-scale structures \citep{hopk15}.  All of these issues motivate both the further development of SPH and competing hydrodynamic schemes, and simulating mergers in a diversity of codes.

\subsection{GASOLINE SPH Code}
\label{ssec:c3_gasoline}

\gasoline\ is a modular, tree-based SPH code that we previously used to explore the parameter space of CO WD mergers in \citeauthor{zhu+13} (\citeyear{zhu+13}; henceforth \citeal{zhu+13}).  Code settings and initial conditions used in this work are nearly identical to those used in \citeal{zhu+13}, and we refer the reader to that paper for further details.  We utilize \gasoline's default \cite{hernk89} kernel with 100 neighbours, and use the asymmetric energy formulation (\citeauthor{wadssq04}, Eqn. 8) to time-evolve particle internal energy.  Artificial viscosity is dynamically controlled using a combination of the Balsara switch and time-variable coefficients for the $\alpha$ and $\beta$ viscosity terms ($\alpha\,=0.05$, $\beta\,=0.1$ when shocks are not present, and approximately unity when they are).  We utilize the Helmholtz equation of state\footnote{Available at \url{http://cococubed.asu.edu/}.} (EOS; \citealt{timms00}) to properly represent arbitrarily degenerate and relativistic gases.  Since \gasoline\ evolves density and entropy, while Helmholtz uses density and temperature, a Newton-Raphson inverter is included in the EOS to determine the latter from the former.  To keep the energy-temperature relation positive-definite for the inverter, we enable Coulomb corrections even when the total entropy becomes negative.  SPH noise ocassionally brings highly degenerate particles to below the Fermi energy.  Under these conditions we set the pressure to the Fermi pressure, but let the energy freely evolve (see \citealt{zhu+13}, Sec. 4.6).

Like in our previous work, we ignore outer hydrogen and helium layers, composition gradients, and any nuclear reactions, in order to focus on the merger hydrodynamics.  Previous work that did include nuclear reactions \citep{loreig09,dan+12}, and in one case an outer helium layer \citep{rask+12}, have shown that they play a negligible role in the hydrodynamics of a $0.625 - 0.65\,\Msun$ CO WD merger.  More massive binaries, as well as less massive ones involving a CO-He hybrid WD, may experience He or CO detonantions during the merger \citep{pakm+10, rask+12, dan+12}.

We use the same version of \gasoline\ as \citeal{zhu+13}, which does not include the improvements recently introduced in \textsc{Gasoline2} \citep{kell+15, tamb+15} and \textsc{ChaNGa/Gasoline} \citep{gove+15}.  These include a turbulent diffusion scheme to facilitate fluid mixing \citep{shen+10} and the use of the $(P_i + P_j)/(\rho_i\rho_j)$ geometric density-averaged term in the SPH force expression (\citealt{kell+14}; practically equivalent to the pressure-entropy formulation of \cite{hopk13}) to eliminate numerical surface tension at contact discontinuities.  We also do not consider more advanced prescriptions for viscosity, such as a Godunov-SPH scheme (eg. \citealt{chaw16}), as these are generally not implemented in SPH codes.  We again stress that the purpose of this work is to compare the traditional SPH formulation, used in almost all merger simulations to date, to moving mesh \arepo, and we leave comparisons with improved and modified SPH schemes to future work.

\subsection{AREPO Moving Mesh Code}
\label{ssec:c3_arepo}

We now introduce the moving-mesh magnetohydrodynamics of \arepo, summarizing \cite{spri10}, \cite{pakmbs11} and \cite{pakms13}.  \arepo\ discretizes a fluid using a grid, or mesh, much like static-grid Eulerian codes.  To overcome the traditional Eulerian code shortcomings of breakdown of Galilean invariance, large advection errors and inability to adjust spatial resolution for arbitrarily complex flows, \arepo\ moves the mesh cells to follow local fluid motion.  Fluxes between cells are then calculated in the frame of the cell walls that divide them -- this preferred frame choice, in addition to the moving mesh, give the scheme a Lagrangian nature and automatic spatial refinement similar to SPH.  The moving mesh also couples more naturally to $N$-body based gravity solvers (see \citealt{spri10}, Sec. 3), with \arepo\ using a nearly identical TreePM solver to that used by the SPH code \textsc{gadget2} \citep{spri05}.

Allowing the mesh-generating points of a structured grid to move with the fluid can lead to severely mesh deformation that prevent its further evolution.  \arepo, however, circumvents this by utilizing an unstructured mesh defined by Voronoi tessellation (see \citealt{spri10}, Sec. 2) of a set of ``mesh-generating points'', each of which corresponds to a single mesh cell.  The mesh-generating points are given the velocities of the fluid parcels they track, and the mesh itself is reconstructed through tessellation at each timestep.  The result is a mesh that, due to the mathematical properties of Voronoi tessellation, does not suffer from mesh-tangling effects.  To keep the Voronoi mesh regular (improving computational efficiency), mesh-generating point velocities are slightly altered from their pure Lagrangian values, and additional velocity adjustments can be made to keep cells near a constant mass or volume.

%The mesh is constructed by Voronoi tesselation of a set of mesh generating points, and these points are given velocities to follow local fluid motion in a Lagrangian fashion.  This results in a mesh that deforms over time to track the evolution of the fluid, without the mesh-tangling effects that hindered previous moving mesh schemes.  Mesh generating point velocities are slightly altered from their pure Lagrangian values to keep the Voronoi mesh regular, reducing flux calculation errors.  Additional velocity adjustments can be made to keep cells near a constant mass or volume, but in practice this becomes ineffective for highly non-linear flows.  Fluid fluxes between cells are calculated using a second-order Godunov scheme with an exact Riemann solver (in our case HLLD), while self-gravity is handled using a TreePM solver, nearly identical to the one used by SPH code \textsc{gadget2} \citep{spri05}.

On the Voronoi mesh, \arepo\ tracks the finite volume integral of ${\bf U}$ for each cell, i.e. 

\eqbegin
{\bf Q} = \int_{V} {\bf U}dV = \left( \begin{array}{c}
m \\
{\bf p} \\
E \\
{\bf B} V \end{array} \right),
\eqend

\noindent where $m$ is the cell mass, ${\bf p}$ its momentum, $E$ its total energy and ${\bf B} V$ the magnetic field multiplied by the cell volume (${\bf B} V$.  If the divergence-cleaning method of \cite{pakmbs11} is adopted (Sec. \ref{sec:c4_postscript}), an additional scalar quantity coupled to the magnetic field divergence is appended to ${\bf Q}$ and to ${\bf W}$, below, but the overall equations remain unchanged.  The time-evolution for cell $i$ from timestep $n$ to $n+1$ is then given by

\eqbegin
{\bf Q}_i^{n+1} = {\bf Q}_i^n - \Delta t\sum_j A_{ij} {\bf \hat{F}}_{ij}^{n + 1/2}
\label{eq:c3_arepo_timeadv}
\eqend

\noindent where $\Delta t$ is the timestep, $j$ stands for all cells that border cell $i$, $A_{ij}$ is the oriented area of the face dividing cells $i$ and $j$ and ${\bf \hat{F}}_{ij}$ is the estimated flux between them (positive flux means escaping from $i$).  In practice fluxes are more easily calculated using primitive variables 

% They're more easily calculated because of the Galilean transforms needed to determine cell fluxes, and to make self-gravity easier to calculate.

\eqbegin
{\bf W} = \left( \begin{array}{c}
\rho \\
{\bf v} \\
P \\
{\bf B} \end{array} \right),
\eqend

\noindent then converted back to ${\bf Q}$.  Fluxes are calculated in the frame of face $A_{ij}$ (then boosted back into the simulation frame of reference) to maintain Galilean invariance.  This means that each hydrodynamic step first calculates the Voronoi mesh and assigns velocities ${\bf w}_i$ to the mesh-generating points before calculating $W$ and ${\bf \hat{F}}$ and then advancing time using Eqn. \ref{eq:c3_arepo_timeadv} (\citealt{spri10}, Sec. 3).  Under standard \arepo\ operation, ${\bf w}_i$ is the same as cell fluid speed ${\bf v_i}$, with a small corrective term to keep the mesh regular, but ${\bf w}_i$ could also be set to zero, turning \arepo\ into a static unstructured-grid code.

Once the velocities are set, the flux across $A_{ij}$ can be determined from the (boosted) primitive values at either side -- which we term the ``left'' and ``right'' states, respectively -- of the face's centroid.  In \arepo's original formulation \citep{spri10}, these values are determined from their respective cell's ${\bf W}$ using the MUSCL-Hancock approach of a piecewise linear spatial reconstruction and a first-order time-extrapolation by half a timestep:

\eqbegin
{\bf W}_\mrm{L,R}^\mrm{interface} = {\bf W}_\mrm{L,R} + {\bf \frac{\partial W}{\partial r}}\Bigr|_\mrm{L,R}({\bf f} - {\bf s}_\mrm{L,R}) + {\bf \frac{\partial W}{\partial t}}\Bigr|_\mrm{L,R}\frac{\Delta t}{2},
\label{eq:c3_muscl_hancock}
\eqend

\noindent where $f$ is position of $A_{ij}$'s centroid, and $s$ each cell's center-of-mass.  The temporal gradient can be solved using the spatial gradient through the Euler equations, while the spatial gradient is estimated by taking advantage of the Green-Gauss theorem (the surface integral of a scalar field $\phi$ is equal to the volume integral of its divergence):

\eqbegin
\left\langle \nabla \phi \right\rangle_i = \frac{1}{V_i} \sum_j \phi({\bf f}_{ij}){\bf A_{ij}},
\label{eq:c3_gauss_green}
\eqend

\noindent taking advantage of the Voronoi mesh to estimate $\phi({\bf f}_{ij})$.  Once calculated, the gradient estimate is slope-limited before being used in Eqn. \ref{eq:c3_muscl_hancock} (\cite{spri10} Eqns. 28 - 30).  The flux is then resolved from ${\bf W}_\mrm{L,R}^\mrm{interface}$ with a Riemann solver (in all our simulations, HLLD; \citealt{miyok05}).  Eqns. \ref{eq:c3_muscl_hancock} and \ref{eq:c3_gauss_green} were replaced in \cite{pakm+16} order to resolve \arepo's angular momentum conservation issue, but the overall flux calculation procedure remains the same as above.

The self-gravity term ${\bf G}$ from Eqn. \ref{eqn:c3_mhd_eqns_compact} can easily be added to the flux calculation, since, when calculating with ${\bf W}$, gravity only changes the momentum.  This change, and the corresponding one for kinetic energy, can then be appended to ${\bf Q}$ (\citealt{spri10} Sec. 5.2).  

The advantages of \arepo\ -- automatic adaptive resolution enhancement, Galilean invariance, accurate shock capture and low fluid velocity noise and artificial viscosity, and a natural coupling to particle-based gravity solvers -- make it an excellent platform with which to simulate mergers.  

We use the same Helmholtz EOS in \arepo\ that we installed into \gasoline\ and also ignore composition gradients and any nuclear reactions.  To assure a reasonably constant mass resolution (and other, similar criteria), we use an explicit refinement scheme \citep{voge+12} that adds or subtracts mesh-generating points to the grid.  This keeps cell masses near a fixed value, and to keep all cell volumes within one order of magnitude of each other.

\subsection{Initial Conditions and Completion Time}
\label{ssec:c3_initcond}

Our chosen WD masses are typical of the narrowly peaked empirical mass distribution of field CO WDs \citep{tremb09, klei+13}.  As in \citeal{zhu+13}, we generated WDs by rescaling a sphere of particles to the proper enclosed mass-radius relation determined from 1D hydrostatic integration.  We used a 50\% C, 50\% O composition by mass uniform throughout the star, and assumed a uniform temperature of $5\times 10^6$ K.  The stars were placed into \gasoline for approximately 11 dynamical times (33.3 s for the $0.625\,\Msun$ WD and 31.3 s for the $0.65\,\Msun$ WD).  Thermal energy and particle velocity were damped to $\sim 5 \times 10^6$ K and 0 cm s$^{-1}$ during the first dynamical time, and left free during the remaining 10 dynamical times.  64 neighbours, rather than 100, were used during relaxation to minimize the number of particle pairs generated.  These pairs (ex. \citealt{dehna12}, \citealt{spri10rev}) do not change global properties of the relaxed WDs, but do effectively reduce spatial resolution and having too many of them make transferring SPH initial conditions into \arepo\ problematic.  Following relaxation, the density profile of both stars were consistent with the hydrostatic equilibrium solution, with the numerical central densities deviating from the 1D integrated ones by less than 1\%.  Since all particles have identical mass, the tenuous outer layers of the WDs are difficult to capture in \gasoline; consequently the radii of the relaxed stars, as defined by the outermost particle, were $\sim 4$\% smaller than the integrated ones.  Even after energy damping, particle noise prevents the central temperature of the relaxed stars from reaching below $\sim 1\times 10^7$ K, so all particle temperatures were artificially reset to $\sim 5 \times 10^6$ K.

We then placed the relaxed stars in a circular, unsynchronized binary, with initial separation $\azero = 2.2\times10^9\,\mrm{cm}$ chosen (using the approximation of \citealt{eggl83}) so that the $0.625\,\Msun$ donor will just overflow its Roche lobe.\footnote{The hydrostatic equilibrium solution radius was used to calculate \azero, rather than the smaller relaxed SPH star radius.  This accounts for the small differences in initial conditions between this work and the equivalent simulation in \citeal{zhu+13}.}  The corresponding orbital period is $49.5\,\mrm{s}$.  These initial conditions do not account for the tidal bulges of the stars, and so are not fully equilibrated (eg. \citealt{dan+11}).  While this means our initial conditions do not reflect ``real'' CO WD binaries with complete accuracy, we stress that the purpose of this work is to discover any code dependence on merger evolution, rather than providing the final word in WD merger simulations.

We generate initial conditions in \arepo\ by converting the SPH particles of the \gasoline\ initial conditions to be mesh-generating points, while retaining their conservative quantities (mass, momentum and energy).  These initial conditions are not guaranteed to be regular, but \arepo\ regularizes the mesh over just a few timesteps by nudging each cell's mesh-generating points to their cell's center of mass.  

%We perform both low and high mass resolution versions of our simulations.  Our SPH particles all have the same mass, which is set to $1\times10^{28}\,\mrm{g}$ at low resolution, comparable to those used in parameter-space sweeps \citep{dan+12, zhu+13, dan+14}, and to $2\times10^{27}\,\mrm{g}$ at high resolution, comparable to the highest resolutions used in other recent work \citep{pakm+12, rask+13}.  We likewise set the \arepo\ explicit refinement scheme's target mass to $1\times10^{28}\,\mrm{g}$ at low resolution and $2\times10^{27}\,\mrm{g}$ at high resolution to give them approximately the same mass resolution as their \gasoline\ counterparts.  Consequentely, our spatial resolution in \gasoline\ is about a factor of five lower than that in \arepo\ at either low or high mass resolution, but because the two codes differ so greatly from each other regardless of resolution, we believe equivalent mass resolution to be the most appropriate comparison (see \citeauthor{voge+12} for complications in achieving equivalent accuracy in SPH and grid codes).  We additionally initialize a background grid of $10^{-5}$ {\gcc} cells to fill the vacuum surrounding the WDs -- this adds only 0.005 {\Msun} of material to the simulation.

Our SPH particles all have the same mass of $2\times10^{27}\,\mrm{g}$ ($1.3\times10^{6}$ particles are needed to represent the system), comparable to the highest resolutions used in other recent work \citep{pakm+12, rask+14}.  We likewise use the \arepo\ refinement scheme's to keep cell masses within a factor of $2$ of this value, and to keep adjacent cell volumes to within a relative factor of 10.  We additionally initialize a background grid of $10^{-5}\,\mrm{gcc}$ cells in \arepo\ to fill the vacuum surrounding the WDs -- this adds only 0.005 {\Msun} of material to the simulation.  Consequentely, our spatial resolution in \gasoline\ is about a factor of {\charles $3$} lower than that in \arepo\, but because the two codes differ so greatly regardless of resolution, we believe equivalent mass resolution to be the most appropriate comparison (see \citeauthor{voge+12} Sec. 2.3 for complications in achieving equivalent accuracy in SPH and grid codes).  \arepo's grid refinement scheme also naturally increases the resolution of our simulations over time, and so all \arepo\ resolutions stated are for the start of the simulation.  In Sec. \ref{ssec:c3_restest} we check if our results are resolution-dependent.

We run both simulation to 1000 s.  For the \gasoline\ simulation, this is long after its hydrodynamic evolution has completed, at $\sim400\,\mrm{s}$, and it begins a phase of slow secular spin-down due to artificial viscosity redistributing angular momentum.  Hydrodynamic evolution has also ended for the \arepo\ simulation by $\sim400\,\mrm{s}$, and the merger remnant also subsequently evolves much more slowly, but, as we shall see in Sec. \ref{sec:c3_results}, this further evolution is also hydrodynamic.

%our two high resolution runs to a \gasoline\ run with $1\times10^{28}\,\mrm{g}$ particle mass, comparable to those used in parameter-space sweeps \citep{dan+12, zhu+13, dan+14}, and an \arepo\ simulation with the same target cell mass.

%Factor of 5 in resolution comes from gasoline needing 100 neighbours, so its resolution R is given by 100V = 4/3piR^3 -> 2.87V^1/3 , while arepo has R = V^1/3, where V is the characteristic volume for a single point.  Gasoline techinically has a better resolution than that, since a cubic spline kernel counts nearby particles more highly, but there's some ambiguity in comparing between the two codes (for example, should R = 0.5V^1/3 for arepo?)
