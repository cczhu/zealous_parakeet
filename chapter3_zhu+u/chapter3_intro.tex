\section{Introduction}
\label{sec:c3_intro}

%According to Loren-Aguilar et al. 2009, the very, very first WD simulations were made by Mochkovitch and Livio 1989 (1989A%26A...209..111M) using the self-consistent field method (SCF).  SCF isn't, as far as I can tell, time-dependent, so really they were just calculating the self-consistent structure of the merger remnant.  They do cite earlier RLOF work by Gingold and Monaghan 1979, but those simulations aren't of WD mergers per se.

Simulations of WD mergers are a window into the detailed dynamics of the merging process and -- since mergers cannot directly be seen using current observational capabilities -- a link between observations of strange stars and explosive transients and theories about their formation.  Since the pioneering work of \cite{benz+90}, these simulations have overwhelmingly used a single numerical hydrodynamics implementation, smoothed-particle hydrodynamics (SPH; eg. \citealt{mona05, spri10rev}), to model WD mergers.  SPH represents a fluid with a set of particles.  The fluid's continuum properties can be determined by using these particles as interpolation points, while the fluid's evolution is determined by tracking the particles' motions over time.  SPH's Lagrangian nature means regions of high density are automatically more resolved and advection is simulated without errors, and its equations of motion also inherently conserve energy, linear and angular momentum.  These features make it much easier to model the bulk fluid motion, large density contrasts and complicated geometry of mergers in SPH than in Eulerian grid schemes\footnote{Eulerian grid methods discretize the simulation volume into a grid.  Fluid properties are tracked for each grid cell, and fluxes between cells are propagated to facilitate time-evolution.}, which have difficulty simulating mergers except under specific coordinate systems and symmetries (see \cite{katz+16} for recent developments).  Building on \citeauthor{benz+90} and other early works such as \cite{segrcm97} and \cite{guerig04}, more recent efforts has focused on more precise binary initial conditions \citep{dan+11}, exploration of remnant properties across parameter space \citep{loreig09, rask+12, zhu+13, dan+14} and exploring the possible instigation and consequences of a thermodynamic transient caused by the merger (eg. \citealt{pakm+10, dan+12, pakm+13, moll+14, rask+14}).

The standard SPH formulation is not without its problems, however.  In order to provide enough dissipation to properly treat shocks, SPH uses an artificial viscosity, which can produce spurious heating and angular momentum transport in shear flows.  While a number of standard remedies exist (eg. \citealt{spri10rev} Sec. 2.3), two of which we use in this work, their effectiveness in systems where shearing and shocks occur in close proximity is questionable.  Errors in SPH interpolation produce noise on the scale of the interpolating kernel smoothing length, which can smear out shocks and steep gradients \citep{spri10rev}.  Classical formulations of SPH have also been known to suppress hydrodynamic instabilties (eg. \citealt{ager+07}), and while this has subsequently been resolved by better treatments of contact discontinuities (eg. \citep{hopk13, hu+14, kell+14}), the vast majority of SPH simulations in the literature do not include them.  Shocks, large-scale shear flows and the formation of instabilities are all expected for WD mergers, and past comparisons between SPH and Eulerian grid codes for a diversity of astrophysical phenomena (eg. \citealt{dval+06, tracsp07, mitc+09}) have often shown qualitative and resolution-independent differences.  
%This suggests the need to simulate WD mergers with other hydrodynamics schemes.
%and incorrectly represent subsonic turbulence \citep{baues12}

A recent alternative to SPH, as well as Eulerian grid codes, is \arepo\ \citep{spri10}, one of a growing class of hydrodynamics codes (eg. \citealt{duffm11, gabujl12, vandr16}) that render fluid evolution on a dynamically moving unstructured mesh.  \arepo\ retains accurate treatment of shocks and instabilities as well as negligible artificial viscosity that Eulerian grid codes feature, while gaining the automatic refinement and Galilean invariance inherent to SPH.  These features, coupled with a tree-based self-gravity solver, make it highly attractive for astrophysical simulations (eg. \citealt{voge+12, pakms13, hayw+14, marips14, ohlm+16}), and, with the notable exception of formal angular momentum conservation, ideal for simulating WD mergers.  \arepo\ has already been used to investigate helium detonations during the initial mass transfer phase of a 0.9 - 1.1 \Msun\ CO WD merger \citep{pakm+13}.

In this work, we compare the merger of a $0.625 - 0.65\,\Msun$ CO WD binary system simulated in the SPH code \gasoline\ \citep{wadssq04} with one simulated in \arepo.  We generate identical initial conditions for both simulations, and disable chemical and nuclear evolution to focus solely on the hydrodynamic differences.  The purpose of this work is not to compare SPH to \arepo\ in the abstract, but rather to understand if any critical hydrodynamic phenomena is missing or misrepresented from past SPH-based merger simulations.  We find the two simulations closely resemble one another until coalescence, after which the \gasoline\ merger remnant becomes axisymmetric over $\sim10^2\,\mrm{s}$, while the \arepo\ one remains highly asymmetric for much longer, potentially altering post-merger evolution.  This work is also a companion to \citeauthor{zhu+15} (\citeyear{zhu+15}; henceforth \citeal{zhu+15}), which presents the growth of a seed magnetic field inserted into the \arepo\ merger to amplitudes of $\sim10^{10}-10^{11}\,\mrm{G}$ at saturation, which also has significant consequences to post-merger evolution.

In Section~\ref{sec:c3_codes}, we review the formulation of SPH and \arepo, and discuss the parameters and initial conditions used in each simulation.  In Section~\ref{sec:c3_fixingarepo}, we summarize efforts to improve angular momentum conservation within \arepo, essential to both this work and \citeal{zhu+15}.  In Section~\ref{sec:c3_results}, we present the results for each code and compare their outcomes.  Lastly, in Section~\ref{sec:c3_discussion}, we describe which code represents the more physical result, possible causes for the differences between the results, and implications for merger outcomes.
