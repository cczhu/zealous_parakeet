\chapter{White Dwarf Mergers in Smoothed-Particle and Moving Mesh Hydrodynamics}

\begin{center}
\begin{minipage}[c]{4.75in}
Chenchong Zhu, R\"{u}diger Pakmor, Marten H. van Kerkwijk, Philip Chang\\
\vspace{2em}
\end{minipage}
\end{center}

%Context
White dwarf binary mergers, the possible progenitors to a number of unsual stars and transient phenomena, can currently only be directly studied in detail through hydrodynamic simulations.  These simulations have almost completely been performed using smoothed-particle hydrodynamics, a scheme known to produce numerical artifacts under certain conditions.
%Aims
%Methods
In order to determine if there is a code dependence for mergers, we simulated the merging of a 0.625 - 0.65 {\Msun} carbon-oxygen white dwarf binary in both {\gasoline} SPH and {\arepo}, which performs hydrodynamics on a moving mesh.
%Results
We find that, just following the coalesence of the two white dwarfs, the \arepo\ merger remnant shows greater disturbance in the 0.65 {\Msun} accretor, as well as a greater temperature contrast between the dense, degenerate core of the remnant and the hot envelope surrounding it.  Following the coalescence of the two stars, the \gasoline\ simulation rapidly achieves axisymmetry and maintains the merger remnant as an oblate, rotationally supported object, while the \arepo\ remnant core stays asymmetric, and launches spiral waves into its surroundings that transport angular momentum at a rate comparable to that of the standard $\alpha$-viscosity formulation.
%Conclusions
These code-dependent differences in merger simulations could substantially affect processes immediately following the merger, from magnetic field dynamo processes to the possible onset of a nuclear detonation.  The final product, of the merging process, however, likely remains a spherically symmetric dense core surrounded by a hot, non-degenerate envelope, regardless of which code is used.

This chapter is a revised version of an unpublished paper completed in late 2014.  In addition, I give a qualitative overview of SPH and moving mesh \arepo, and includes a summary of our preliminary investigation, which lead to the discovery that \arepo\ failed to conserve angular momentum over relevant timescales.  The latter eventually led to the publication of \cite{pakm+16}, which I also summarize.  Since we performed our study, SPH codes, including \gasoline, have been updated to resolve many of their classic deficiencies, and a comparison with modern codes may yield different results.
