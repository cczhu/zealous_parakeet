\chapter{Mergers in Smoothed-Particle and Moving Mesh Hydrodynamics}
\label{ch:ch3}

\begin{center}
\begin{minipage}[c]{4.75in}
Chenchong Zhu, R\"{u}diger Pakmor, Marten H. van Kerkwijk and Philip Chang\\
\vspace{2em}
\end{minipage}
\end{center}

The physics and final outcomes of the merger of two white dwarfs can currently only be directly studied through 3D hydrodynamic simulations, and to date merger simulations have largely relied on smoothed-particle hydrodynamics, a method known to produce numerical artifacts under certain conditions.  In order to determine if the outcome of these simulations depends on the code being used, we followed the merger of a $0.625 - 0.65\,\Msun$ carbon-oxygen white dwarf binary in both the SPH code \gasoline\ and the moving mesh code \arepo.  We find that the two agree well with one another until the merger is complete.  Afterward, the merger remnant becomes axisymmetric over the course of a few hundred seconds in \gasoline, with most of its mass comprising a dense, oblate-spheroidal core.  The remnant in \arepo, on the other hand, remains non-axisymmetric and features a crescent-shaped core flanked on one side by a hot, underdense ``void''.  This configuration has an offset gravitational potential, which launches an $m = 1$ spiral mode within the surrounding disk that transports disk angular momentum over a timescale of $\sim10^3\,\mrm{s}$, substantially faster than suggested by other post-merger evolution studies.  These code-dependent differences could affect the early phase of post-merger viscous evolution.  The final product of the merging process, however, likely remains a spherically symmetric dense core surrounded by a hot, non-degenerate envelope, regardless of which code is used.

%Since we performed our study, SPH codes, including \gasoline, have been updated to resolve many of their classic deficiencies, and a comparison with modern codes may yield different results.
