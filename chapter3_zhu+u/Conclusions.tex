\section{Conclusions}
\label{sec:conclusions}

We perform a comparison between a simulation of a 0.625 - 0.65 \Msun\ merger performed in SPH \gasoline\ with one performed in \arepo\ moving mesh.  We find higher temperatures for newly accreted material before coalescence, and a more well-defined hot, low density void, as well as a greater temperature contrast between the void and dense remnant core material, just following coalescence.  We believe these differences to be due to SPH's artificial viscosity in concert with poorer shock and instability capturing.  Following coalescence, the merger remnant in \arepo\ begins a phase of rapid angular momentum transport mediated by spiral waves that are launched from the remnant core into the surrounding medium.  These waves cause the remnant core to lose most of its angular momentum and achieve near-spherical symmetry.  As SPH is known to suppress spiral waves and other disk structures, we believe the earliest phase of post-merger evolution is more accurately captured in \arepo .

While we do find novel hydrodynamic behaviour in our \arepo\ simulations we believe are being artificially suppressed in SPH simulations, we also found unphysical angular momentum losses in \arepo\ that were only discovered through our comparative analysis, and future simulations must still be performed with caution.  Despite its shortcomings, SPH still appears superior in maintaining conserved quantites over hundreds of dynamical times.  It remains to be seen if novel formulations of SPH can once again make it the preferred choice for simulating mergers.

Moreover, our simulation were more test cases than accurate representations of mergers: missing are nuclear reactions, possible synchronization \citep{fulll12, burk+13}, proper modelling of the tidal bulges (for either synchronized or unsynchronized binaries) and other essential features of a complete picture of white dwarf mergers.  In particular, the generation of a strong global magnetic field out of a dynamically negligible seed field would likely occur in a merger, since \textbf{EVIDENCE FROM SINGLE WDs AND NON-MERGING BINARIES}.  We believe this hetherto neglected aspects of mergers to be of prime importance to the dynamics of mergers, and are currently exploring its effects using \arepo\ MHD simulations.

We thank Volker Springel, James Wadsley, Christopher Matzner, Ue-Li Pen and Stephen Ro for their insight into hydrodynamics and simulations.  This work was supported by the Natural Sciences and Engineering Research Council (NSERC) Vanier Canada Graduate Scholarship and Reinhardt Endownment Travel Fund.  Computations were performed on the GPC supercomputer at the SciNet HPC Consortium.  SciNet \citep{loke+10} is funded by the Canada Foundation for Innovation under the auspices of Compute Canada, the Government of Ontario, Ontario Research Fund - Research Excellence and the University of Toronto.

