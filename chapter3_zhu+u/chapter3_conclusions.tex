\section{Conclusions and Ramifications for Mergers}
\label{sec:c3_conclusion}

We simulated the merger of a $0.625-0.65\,\Msun$ CO WD merger in the SPH code \gasoline\ and moving mesh code \arepo, to determine if the outcome of these simulations depend on the code being used.  We find differences between the simulations to be small for all phases of the merger up to coalescence.  Immediately following coalescence, there is a greater temperature contrast between the void and dense remnant core material in \arepo\ than in \gasoline, but otherwise the remnants look remarkably similar to one another.  Over the next several hundred seconds, however, the \gasoline\ remnant spins down to axisymmetry, and the remnant's temperature becomes roughly uniform.  The \arepo\ remnant, on the other hand, maintains the integrity of a dense, crescent-shaped cold core surrounded by a hot, tenuous void for at least $\sim1000\,\mrm{s}$.  The core generates a lopsided gravitational potential, which launches an $m = 1$ spiral mode into the disk.  This mode rapidly transports disk angular momentum at a rate equivalent to an $\alpha$-viscosity disk with $\alpha = 10^{-1}$.  We suggest that this non-axisymmetric perturbation may be related to the low $T/|W|$ instability seen in neutron star mergers and core-collapse supernova remnants, and possibly the ARS instability found in protoplanetary disks, and that the perturbation may not be captured as well in \gasoline\ because of SPH surface tension and artificial viscosity.  If so, we expect a longer-lived crescent in simulations using SPH codes updated for improved handling of mixing and contact discontinuities, including \textsc{Gasoline2}.

Our results are of greatest consequence for post-merger evolution, which until now has been believed to occur on a viscous timescale (\citeal{vkercj10}, \citealt{shen+12}), and has been simulated using Eulerian codes on axisymmetric cylindrical grids \citep{schw+12,ji+13}.  These are, of course, unable to directly capture non-axisymmetric features of the remnant, and we therefore stress the need to extend post-merger evolution simulations to three dimensions.  A simulation that extends to $10^4\,\mrm{s}$ could realistically be performed on \arepo, and we are currently investigating the source of the spurious cold ring discussed at the end of Sec. \ref{sec:c3_results} to ensure \arepo\ does not generate systematic errors over this timespan.  Our results likely do not qualitatively change previous conclusions about the outcome of post-merger evolution from \cite{schw+12} and \cite{ji+13}, as hydrodynamic waves will simply hasten the general loss of remnant angular momentum and the transformation of the disk into a hot envelope.  The details of evolution may differ, though, since wave transport appears most relevant for regions of the disk beyond $\varpi \approx 1.5\times10^9\,\mrm{cm}$, while viscosity will also affect the remnant core.  Also, traveling waves do not necessarily dissipate their energy while passing through a medium, whereas viscosity locally dissipates differential rotational energy, and so the heating of the remnant during viscous evolution may also change.  Lastly, since the dense crescent remains relatively cold and maintains its integrity long after coalescence, the merger remnant may not be hottest near its center once it becomes axisymmetric, despite being considered ``similar-mass'' from the results of Ch. \ref{ch:ch2}.  This has consequences for the viability of the sub-\Mch\ merger channel, and is discussed further in Ch. \ref{sec:c6_mergers_pme}.

On the other hand, we find that bulk properties of the merger remnant immediately following coalescence do not substantially differ between our \gasoline\ and \arepo\ simulations.  Many of the conclusions reached by prior SPH merger studies (eg. \citeal{loreig09}, Ch. \ref{ch:ch2}) might therefore be robust.  Definitive evidence, however, will only come by extending our work with \arepo\ to mergers of CO WDs with other masses.  It also remains to be seen if non-axisymmetric perturbations during post-merger evolution depend on the total mass or mass ratio of the merging binary.  A parameter-space study could also pinpoint the range of remnants that experience a detonation due to rapid accretion via spiral modes, as seen in \cite{kash+15}.

% THESIS NOTE: 500 s comes from t_f calculation made on arepo magnetic balance

Finally, our simulation does not contain magnetic fields, which, as will be discussed in Ch. \ref{ch:ch4}, are likely to be significantly amplified during the merger and early phases of post-merger evolution.  This could substantially affect the remnant's non-axisymmetric features: for differentially-rotating magnetized neutron stars that feature the low $T/|W|$ instability, \cite{muhl+14} find fields had either a suppressive or an \textit{amplifying} effect on the instability, depending on their strength.  In our own magnetized merger simulations, the remnant becomes axisymmetric within $\sim500\,\mrm{s}$ after coalescence, but we use the Powell scheme for divergence cleaning (Sec. \ref{sec:c4_postscript}) and likely resolve the remnant disk too poorly to capture the fastest growing magnetorotational instability mode.  Future investigation of post-merger MHD evolution is, therefore, also warranted.

\vspace{10mm}

We thank Christopher Matzner, Volker Springel, Terrence Tricco, James Wadsley, Ue-Li Pen and Stephen Ro for their insight into hydrodynamics and simulations.  This work was supported by the Natural Sciences and Engineering Research Council (NSERC) Vanier Canada Graduate Scholarship and Reinhardt Endowment Travel Fund.  Computations were performed on the GPC supercomputer at the SciNet HPC Consortium.  SciNet \citep{loke+10} is funded by the Canada Foundation for Innovation under the auspices of Compute Canada, the Government of Ontario, Ontario Research Fund - Research Excellence and the University of Toronto.

%The more impactful change to the conclusion of other works we make is that hydrodynamic evolution does not stop at coalescence, and a spiral wave-mediated spin down of the merger remnant occurs over several thousand seconds.  This spin-down originates from the non-axisymmetry of the remnant core, which itself originates from material from the tidally destroyed donor impacting the accretor.  Merger remnants of dissimilar-mass mergers will experience far less accretor disruption than the system we present here, but we find for a low-resolution 0.5 - 1.0 \Msun\ CO WD merger that the \textit{donor} does not disrupt into a completely axisymmetric disk around the accretor, and this non-axisymmetry powers a spiral wave propagating into the disk.  Hydrodynamic spin down is therefore likely prevalent throughout the majority of the WD merger parameter space.

%Because previous works found hydrodynamic evolution to stop at coalescence, it was expected that the next phase of evolution for the remnant would be a magnetically-driven spin-down phase lasting of order hours to days \citep{vkercj10, shen+12}.  \cite{schw+12} and \cite{ji+13} have both simulated this evolution by porting SPH merger results into 2.5D hydrodynamic simulations.  \citeauthor{schw+12} use \textsc{zeus-mp2} fitted with the Helmholtz EoS, a \cite{shaks73} shear viscosity set to $\alpha = 3 \times 10^{-2}$ and a five-isotope nuclear network in order to evolve merger remnants imported from \cite{dan+11} with a span of masses and compositions.  For their fiducial 0.6 - 0.9 \Msun\ remnant, they find it nearly completely spins down over $3\times10^4$ s, transferring its disk mass and most of its angular momentum into a thermally supported thin envelope (the remnant core barely changes in mass).  The material at interface between core and disk, where there is a temperature peak, is compressed by loss of rotational support such that its temperature increases from $5\times10^8$ K to $7.5\times10^8$ K, and its density from $2\times10^5$ \gcc\ to $6\times10^5$ \gcc.  Very little mass is lost.  \citeauthor{ji+13} use \flash\ in AMR mode to simululate the full magnetohydrodynamic evolution of a 0.6 - 0.6 \Msun remnant ported from \citeauthor{loreig09} and augmented with a weak poloidal seed magnetic field.  Over $2\times10^4$ s, the remnant core loses 70\% of its angular momentum, the magnetic field strengthens from $3\times10^5$ G to $2\times10^8$ G, most of the disk mass is accreted onto the remnant (with 0.06 \Msun\ forming a thin envelope at large distance).  The core's central density and temperature rise from $\sim2\times10^6$ \gcc and $\sim4\times10^8$ K to $\sim5\times10^6$ \gcc and $\sim9\times10^8$ K, leading to a core nuclear runaway.

%In addition to being more than an order-of-magnitude faster, our hydrodynamic spin-down phase results in a somewhat different remnant.  All three simulations agree that disk material will spread outward from the disk plane, and that much of the disk will turn into a thermally-supported hot envelope.  The core-envelope interface, which is the location of the off-centre temperature peak, increases in density and temperature from , and the center of the remnant also compresses by a factor of 3.  On the other hand, while \citeauthor{ji+13} find their ``white dwarf merger'' (roughly equivalent to our merger core and dense portion of the thermal envelope \gcc ) accretes 0.16 {\Msun}, and \citeauthor{schw+12} find theirs remains at approximately 1 {\Msun} (their Fig. 4), our \arepo\ ``WD merger'' actually loses mass, from 1.1 \Msun to 0.95 \Msun\ (and of this, only 0.65 \Msun\ is degeneracy energy dominated; see Fig. \ref{fig:comp}).  We also find no unbound outflow of mass, even though our mass resolution, $10^{-6}$ \Msun\, is three orders of magnitude below the mass loss reported by \citeauthor{ji+13} and one below that reported by \citeauthor{schw+12}.  That \cite{ji+13} find a central nuclear runaway, and we do not, is primarily due to their use of \cite{loreig09}'s 0.6 - 0.6 {\Msun} remnant, than differences in post-coalescence evolution.  At \citeauthor{loreig09}'s remnant has its highest temperature at the center of the core, and is overall much hotter than our \arepo\ (and even our \gasoline) remnant just after coalescence.

%With most of the rotational energy in the entire system converted to thermal energy, viscous evolution will be unimportant to the spun-down remnant.  As discussed in \cite{shen+12}, thermal evolution will now dominate the system.  Throughout much of the envelope, radiation pressure is comparable or exceeds gas pressure, and so the remnant's luminosity will be of order the Eddington luminosity \citep{shen+12}, and may also launch a wind.  Our spun-down remnant contains a shell of $\sim10^6$ \gcc\ material at $\sim8\times10^7$ K, which is sufficient to generate a convective carbon burning shell.  The end state of this evolution is likely an oxygen-neon WD \citep{nomoi85, shen+12}.  Since the total system mass is far below the Chandrasekhar mass, the densities required for accretion-induced collapse will not be reached.  While we substitute a viscous angular momentum transport phase for a hydrodynamic one, we do not predict, based on the results in this paper, the final outcome of a merger to be all that different from previous expectations.

