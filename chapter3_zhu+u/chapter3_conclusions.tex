\section{Conclusions and Ramifications for Mergers}
\label{sec:c3_conclusion}

%Our current studies somewhat affect the conclusions made for pre-coalescence merger evolution made in other works.  A number of previous works (DAN, RASKIN, RUEDIGER?) use relatively low-resolution SPH simulations to try to discern if helium or carbon will violently ignite during the early stages of mass transfer.  As we find higher temperatures in newly accreted material during this phase of the merger, we believe that SPH may underestimate the potential for violent nuclear reactions.  This has been long-suspected, and is the reason why \cite{guil+10} used a hybrid approach to simulate early mass transfer during a merger.  As shown in Fig. \ref{fig:comp}, the bulk properties and overall profiles between \arepo\ and \gasoline\ are similar, except for the low-density void near the remnant core centre and a much lower core temperature.  We performed a number of low-resolution diagnostic tests which show that, even for an unsynchronized, nearly equal-mass 0.625 - 0.626 \Msun\ merger, the merger remnant remains centrally cold, surrounded by a warmer envelope, in contrast to our SPH results in \citeauthor{zhu+13}, as well as \citeauthor{loreig09}.

%The more impactful change to the conclusion of other works we make is that hydrodynamic evolution does not stop at coalescence, and a spiral wave-mediated spin down of the merger remnant occurs over several thousand seconds.  This spin-down originates from the non-axisymmetry of the remnant core, which itself originates from material from the tidally destroyed donor impacting the accretor.  Merger remnants of dissimilar-mass mergers will experience far less accretor disruption than the system we present here, but we find for a low-resolution 0.5 - 1.0 \Msun\ CO WD merger that the \textit{donor} does not disrupt into a completely axisymmetric disk around the accretor, and this non-axisymmetry powers a spiral wave propagating into the disk.  Hydrodynamic spin down is therefore likely prevalent throughout the majority of the WD merger parameter space.

%Because previous works found hydrodynamic evolution to stop at coalescence, it was expected that the next phase of evolution for the remnant would be a magnetically-driven spin-down phase lasting of order hours to days \citep{vkercj10, shen+12}.  \cite{schw+12} and \cite{ji+13} have both simulated this evolution by porting SPH merger results into 2.5D hydrodynamic simulations.  \citeauthor{schw+12} use \textsc{zeus-mp2} fitted with the Helmholtz EoS, a \cite{shaks73} shear viscosity set to $\alpha = 3 \times 10^{-2}$ and a five-isotope nuclear network in order to evolve merger remnants imported from \cite{dan+11} with a span of masses and compositions.  For their fiducial 0.6 - 0.9 \Msun\ remnant, they find it nearly completely spins down over $3\times10^4$ s, transferring its disk mass and most of its angular momentum into a thermally supported thin envelope (the remnant core barely changes in mass).  The material at interface between core and disk, where there is a temperature peak, is compressed by loss of rotational support such that its temperature increases from $5\times10^8$ K to $7.5\times10^8$ K, and its density from $2\times10^5$ \gcc\ to $6\times10^5$ \gcc.  Very little mass is lost.  \citeauthor{ji+13} use \flash\ in AMR mode to simululate the full magnetohydrodynamic evolution of a 0.6 - 0.6 \Msun remnant ported from \citeauthor{loreig09} and augmented with a weak poloidal seed magnetic field.  Over $2\times10^4$ s, the remnant core loses 70\% of its angular momentum, the magnetic field strengthens from $3\times10^5$ G to $2\times10^8$ G, most of the disk mass is accreted onto the remnant (with 0.06 \Msun\ forming a thin envelope at large distance).  The core's central density and temperature rise from $\sim2\times10^6$ \gcc and $\sim4\times10^8$ K to $\sim5\times10^6$ \gcc and $\sim9\times10^8$ K, leading to a core nuclear runaway.

%In addition to being more than an order-of-magnitude faster, our hydrodynamic spin-down phase results in a somewhat different remnant.  All three simulations agree that disk material will spread outward from the disk plane, and that much of the disk will turn into a thermally-supported hot envelope.  The core-envelope interface, which is the location of the off-centre temperature peak, increases in density and temperature from , and the center of the remnant also compresses by a factor of 3.  On the other hand, while \citeauthor{ji+13} find their ``white dwarf merger'' (roughly equivalent to our merger core and dense portion of the thermal envelope \gcc ) accretes 0.16 {\Msun}, and \citeauthor{schw+12} find theirs remains at approximately 1 {\Msun} (their Fig. 4), our \arepo\ ``WD merger'' actually loses mass, from 1.1 \Msun to 0.95 \Msun\ (and of this, only 0.65 \Msun\ is degeneracy energy dominated; see Fig. \ref{fig:comp}).  We also find no unbound outflow of mass, even though our mass resolution, $10^{-6}$ \Msun\, is three orders of magnitude below the mass loss reported by \citeauthor{ji+13} and one below that reported by \citeauthor{schw+12}.  That \cite{ji+13} find a central nuclear runaway, and we do not, is primarily due to their use of \cite{loreig09}'s 0.6 - 0.6 {\Msun} remnant, than differences in post-coalescence evolution.  At \citeauthor{loreig09}'s remnant has its highest temperature at the center of the core, and is overall much hotter than our \arepo\ (and even our \gasoline) remnant just after coalescence.

%With most of the rotational energy in the entire system converted to thermal energy, viscous evolution will be unimportant to the spun-down remnant.  As discussed in \cite{shen+12}, thermal evolution will now dominate the system.  Throughout much of the envelope, radiation pressure is comparable or exceeds gas pressure, and so the remnant's luminosity will be of order the Eddington luminosity \citep{shen+12}, and may also launch a wind.  Our spun-down remnant contains a shell of $\sim10^6$ \gcc\ material at $\sim8\times10^7$ K, which is sufficient to generate a convective carbon burning shell.  The end state of this evolution is likely an oxygen-neon WD \citep{nomoi85, shen+12}.  Since the total system mass is far below the Chandrasekhar mass, the densities required for accretion-induced collapse will not be reached.  While we substitute a viscous angular momentum transport phase for a hydrodynamic one, we do not predict, based on the results in this paper, the final outcome of a merger to be all that different from previous expectations.

%We perform a comparison between a simulation of a 0.625 - 0.65 \Msun\ merger performed in SPH \gasoline\ with one performed in \arepo\ moving mesh.  We find higher temperatures for newly accreted material before coalescence, and a more well-defined hot, low density void, as well as a greater temperature contrast between the void and dense remnant core material, just following coalescence.  We believe these differences to be due to SPH's artificial viscosity in concert with poorer shock and instability capturing.  Following coalescence, the merger remnant in \arepo\ begins a phase of rapid angular momentum transport mediated by spiral waves that are launched from the remnant core into the surrounding medium.  These waves cause the remnant core to lose most of its angular momentum and achieve near-spherical symmetry.  As SPH is known to suppress spiral waves and other disk structures, we believe the earliest phase of post-merger evolution is more accurately captured in \arepo .

%While we do find novel hydrodynamic behaviour in our \arepo\ simulations we believe are being artificially suppressed in SPH simulations, we also found unphysical angular momentum losses in \arepo\ that were only discovered through our comparative analysis, and future simulations must still be performed with caution.  Despite its shortcomings, SPH still appears superior in maintaining conserved quantites over hundreds of dynamical times.  It remains to be seen if novel formulations of SPH can once again make it the preferred choice for simulating mergers.

%Moreover, our simulation were more test cases than accurate representations of mergers: missing are nuclear reactions, possible synchronization \citep{fulll12, burk+13}, proper modelling of the tidal bulges (for either synchronized or unsynchronized binaries) and other essential features of a complete picture of white dwarf mergers.  In particular, the generation of a strong global magnetic field out of a dynamically negligible seed field would likely occur in a merger, since \textbf{EVIDENCE FROM SINGLE WDs AND NON-MERGING BINARIES}.  We believe this hetherto neglected aspects of mergers to be of prime importance to the dynamics of mergers, and are currently exploring its effects using \arepo\ MHD simulations.

%XXX NEUTRON STAR STUDIES suggest that the continued transport of angular momentum eventually disrupts the torus and ends the unstable mode.  In our simulation, the underdense void is significantly smaller at $1000\,\mrm[s}$, which suggests the instabiliity will also disappear (probabily coinciding with the depletion of angular momentum in the disk).  Preliminary attempts to carry the simulation even further in time, however, have been stymied by the spurious formation of a $<10^7\,\mrm{K}$ ring of material at the interface between donor and accretor, like in the lowest-resolution \arepo\ simulation in the resolution test of Sec. \ref{ssec:c3_restest}.  We therefore must leave further investigation both the abstract and specifically within our merger remnants, to future work.


We thank Volker Springel, James Wadsley, Christopher Matzner, Ue-Li Pen and Stephen Ro for their insight into hydrodynamics and simulations.  This work was supported by the Natural Sciences and Engineering Research Council (NSERC) Vanier Canada Graduate Scholarship and Reinhardt Endownment Travel Fund.  Computations were performed on the GPC supercomputer at the SciNet HPC Consortium.  SciNet \citep{loke+10} is funded by the Canada Foundation for Innovation under the auspices of Compute Canada, the Government of Ontario, Ontario Research Fund - Research Excellence and the University of Toronto.

