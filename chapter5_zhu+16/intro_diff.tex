2c2
< \label{sec:c5_intro}
---
> \label{sec:intro}
10c10
< Since hydrostatic sub-\Mch\ WDs do not possess central densities high enough for pycnonuclear fusion, burning must be prompted either by a shockwave (that immediately triggers dynamical burning), or by heating material to $T\approx6\times10^8\,\mrm{K}$ to induce fusion through high temperatures, rather than high densities.  The former is presumed in eg. the double-detonation (eg. \citealt{fink+07, woosk11}), violent merger (eg. \citealt{pakm+10}) and collision (eg. \citealt{loreig10}) scenarios.  The latter is presumed in the sub-\Mch\ CO WD merger scenario of \citeal{vkercj10}, in which two CO WDs of mass $\sim0.65$ \Msun\ merge, forming a merger remnant that is differentially rotating throughout its structure, and so is susceptible to magnetohydrodynamic instabilities \citep{shen+12, ji+13}.  It subsequently becomes strongly magnetized, leading to outward transport of angular momentum over a viscous timescale of $\sim10^4 - 10^8\,\mrm{s}$ that robs the remnant of its rotational support (\citeal{vkercj10}, \citealt{shen+12}; though see Ch. \ref{ch:ch3}, \citealt{kash+15} for evidence of transport through spiral hydrodynamic waves).  This viscous evolution compresses and adiabatically heats the remnant's interior, which the simulation of \cite{ji+13} suggests leads to central carbon ignition.
---
> Since hydrostatic sub-\Mch\ WDs do not possess central densities high enough for pycnonuclear fusion, burning must be prompted either by a shockwave (that immediately triggers dynamical burning), or by heating material to $T\approx6\times10^8\,\mrm{K}$ to induce fusion through high temperatures, rather than high densities.  The former is presumed in eg. the double-detonation (eg. \citealt{fink+07, woosk11}), violent merger (eg. \citealt{pakm+10}) and collision (eg. \citealt{loreig10}) scenarios.  The latter is presumed in the sub-\Mch\ CO WD merger scenario of \citeal{vkercj10}, in which two CO WDs of mass $\sim0.65$ \Msun\ merge, forming a merger remnant that is differentially rotating throughout its structure (eg. \citealt{zhu+13}, henceforth \citeal{zhu+13}), and so is suceptible to magnetohydrodynamic instabilities \citep{shen+12, ji+13}.  It subsequently becomes strongly magnetized, leading to outward transport of angular momentum over a viscous timescale of $\sim10^4 - 10^8\,\mrm{s}$ that robs the remnant of its rotational support (\citeal{vkercj10}, \citealt{shen+12}; though see \citealt{kash+15, zhu+15} for evidence of transport through spiral hydrodynamic waves).  This viscous evolution compresses and adiabatically heats the remnant's interior, which the simulation of \cite{ji+13} suggests leads to central carbon ignition.
14c14
< In this chapter, we investigate the simmering phase of sub-\Mch\ WDs undergoing a center-lit thermonuclear runaway.  In Section \ref{sec:c5_modelsim} we detail our semi-analytical model and its assumptions.  While our motivation is to better understand the evolution of merger remnants, we keep these assumptions simple, and we stress that our models do \textit{not} necessarily account for any specific prior evolution.  We present our results in Section \ref{sec:c5_results}, determining both the value of \Mcrit\ and estimating the mass of radioactive \Ni\ produced if a detonation were to follow immediately.  Simmering WDs that formed in mergers are expected to be rotating and magnetized (eg. Ch. \ref{ch:ch4}, \citealt{wicktf14}), and we also try to ascertain how \Mcrit\ and \MNi\ change in their presence using simple prescriptions for rotational and magnetic convective suppression.  Lastly, in Section \ref{sec:c5_discussion}, we discuss how our results apply to the \citeal{vkercj10} scenario as well as implications for SN Ia progenitor studies in general.
---
> In this paper, we investigate the simmering phase of sub-\Mch\ WDs undergoing a center-lit thermonuclear runaway.  In Section \ref{sec:modelsim} we detail our semi-analytical model and its assumptions.  While our motivation is to better understand the evolution of merger remnants, we keep these assumptions simple, and we stress that our models do not necessarily account for any specific prior evolution.  We present our results in Section \ref{sec:results}, determining both the value of \Mcrit\ and estimating the mass of radioactive \Ni\ produced if a detonation were to follow immediately.  Simmering WDs that formed in mergers are expected to be rotating and magnetized (eg. \citealt{wicktf14, zhu+15}), and we also try to ascertain how \Mcrit\ and \MNi\ change in their presence using simple prescriptions for rotational and magnetic convective suppression.  Lastly, in Section \ref{sec:discussion}, we discuss how our results apply to the \citeal{vkercj10} scenario as well as implications for SN Ia progenitor studies in general.
