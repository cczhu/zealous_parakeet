\chapter{The Evolution of Simmering Sub-Chandrasekhar Mass White Dwarfs}
\label{ch:ch5}

\begin{center}
\begin{minipage}[c]{4.75in}
Chenchong Zhu, Philip Chang, Marten H. van Kerkwijk and Ken J. Shen\\
(Zhu et al. in preparation)
\vspace{2em}
\end{minipage}
\end{center}

When fusion is lit in the degenerate interior of a carbon-oxygen white dwarf, the resulting nuclear runaway starts with a ``simmering phase'', in which convection transports energy out of the burning region. 
%Aims
While simmering inevitably leads to some form of explosion for white dwarfs near the Chandrasekhar mass, in ones with lower mass it may instead lead to the lifting of degeneracy and expansion into a carbon-burning star.
%Methods/Results
Using analytical arguments and simple models, we determine that the critical mass for explosions to be possible is $\Mcrit \approx 1.15\,\Msun$.  In principle, effects from rotation and magnetic fields might lead to a change in the critical mass.  For the rotation rates found in merger remnant simulations, the effect is likely minimal, $\sim0.01\,\Msun$.  For magnetic fields the case is less clear, since interaction with convection is poorly understood, but simple order-of-magnitude arguments again suggest only a small effect.
