\section{Introduction}
\label{sec:c5_intro}

Type Ia supernovae (SNe Ia) are generally thought to be the thermonuclear explosions of carbon-oxygen white dwarfs (CO WDs).  While the current body of observational evidence has greatly strengthened this hypothesis, the mechanism(s) by which a WD is agitated into exploding remains mysterious (see \citealt{howe11, hill+13, maozmn14} for recent reviews).  

The traditional scenario involves slow accretion of material onto the CO WD from either a non-degenerate companion or another WD.  As the CO WD approaches the Chandrasekhar mass \Mch, its central density becomes sufficiently high that the rate of heating from pycnonuclear carbon fusion exceeds that of neutrino cooling.  The resulting increase in central entropy establishes a convection zone that transports the heat of the nuclear burning region to the rest of the WD interior.  The WD is highly degenerate, however, and does not expand in response to the heating.  {\charles Instead, a ``simmering phase'' of increasingly rapid nuclear burning within an increasingly hot WD ensues over the next thousand years.  Near-\Mch\ WDs eventually become hot enough that the timescale for nuclear burning becomes shorter than the dynamical time -- we call this ``dynamical burning'' -- at which point a thermonuclear burning wave is initialized and a deflagration, detonation or some other explosive event} becomes inevitable.

This scenario is beset by two major issues: an apparent paucity of accreting or merging CO WDs that can achieve \Mch, and the difficulty for the thermonuclear explosion of a \Mch\ WD to reproduce the properties and population-level trends of observed SNe Ia (\citealt{vkercj10}, henceforth \citeal{vkercj10}, and references therein).  These, in turn, have spurred research into alternative formation channels where CO WDs with masses significantly \textit{below} \Mch\ also explode.  Including them would bolster substantially the number of progenitors (eg. \citealt{badem12}), and their explosions could closely resemble ordinary SNe Ia \citep{shig+92, sim+10}.

Since hydrostatic sub-\Mch\ WDs do not possess central densities high enough for pycnonuclear fusion, burning must be prompted either by a shockwave (that immediately triggers dynamical burning), or by heating material to $T\approx6\times10^8\,\mrm{K}$ to induce fusion through high temperatures, rather than high densities.  The former is presumed in {\charles the double-detonation (eg. \citealt{fink+07, woosk11, pakm+13, shenm14})}, violent merger (eg. \citealt{pakm+10}) and collision (eg. \citealt{loreig10}) scenarios.  The latter is presumed in the sub-\Mch\ CO WD merger scenario of \citeal{vkercj10}, in which two CO WDs {\charles with similar masses of around $0.65$} \Msun\ merge, forming a merger remnant that is differentially rotating throughout its structure, and so is susceptible to magnetohydrodynamic instabilities \citep{shen+12, ji+13}.  It subsequently becomes strongly magnetized, leading to outward transport of angular momentum over a viscous timescale of $\sim10^4 - 10^8\,\mrm{s}$ that robs the remnant of its rotational support (\citeal{vkercj10}, \citealt{shen+12}; though see Ch. \ref{ch:ch3}, \citealt{kash+15} for evidence of transport through spiral hydrodynamic waves).  This viscous evolution compresses and adiabatically heats the remnant's interior, which the simulation of \cite{ji+13} suggests leads to central carbon ignition.

In the \citeal{vkercj10} scenario, and, indeed, any situation where non-dynamical nuclear burning ignites under degenerate conditions in a sub-\Mch\ WD, a simmering phase ensues just like in the near-\Mch\ case.  Sub-\Mch WDs, however, are less dense, and their degeneracy is at least partially lifted before they reach temperatures where burning becomes dynamical\csout{, which depends only weakly on density}.  There is, in fact, a critical mass \Mcrit\ below which the simmering WD never reaches dynamical burning before becoming non-degenerate enough to expand and cool off.  Even those WDs that do achieve dynamical burning may substantially expand beforehand.  Since the fraction of material fused to peak-iron products in an SN Ia depends on its progenitor's density profile, a pre-expanded WD will produce reduced, or even negligible, amounts of \Ni.  Thus {\charles runaway nuclear burning} at the center of a sub-\Mch\ remnant neither inevitably leads to an explosion, nor is any explosion guaranteed to resemble an SN Ia.

In this chapter, we investigate the simmering phase of sub-\Mch\ WDs undergoing center-lit {\charles carbon burning}.  In Section \ref{sec:c5_modelsim} we detail our semi-analytical model and its assumptions.  While our motivation is to better understand the evolution of merger remnants, we keep these assumptions simple, and we stress that our models do \textit{not} necessarily account for any specific prior evolution.  We present our results in Section \ref{sec:c5_results}, determining both the value of \Mcrit\ and estimating the mass of radioactive \Ni\ produced if a detonation were to follow immediately.  Simmering WDs that formed in mergers are expected to be rotating and magnetized (eg. Ch. \ref{ch:ch4}; \citealt{ji+13, wicktf14}), and we also try to ascertain how \Mcrit\ and \MNi\ change in their presence using simple prescriptions for rotational and magnetic convective suppression.  Lastly, in Section \ref{sec:c5_discussion}, we discuss how our results apply to the \citeal{vkercj10} scenario as well as implications for SN Ia progenitor studies in general.
