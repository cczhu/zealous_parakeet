2c2
< \label{sec:c5_results}
---
> \label{sec:results}
4c4
< We now consider the simmering of sub-\Mch\ WDs represented by models with increasing complexity and features.  We first consider in Sec. \ref{ssec:c5_runaway_ad} models where the superadiabatic deviation \dnabconv\ is neglected, and the temperature gradient is approximated with $\nabla = \nablaad$. These reproduce all the qualitative features of the runaway, and are good approximations of more complex models.  We then move on to ones that include the superadiabatic temperature deviation \dnabconv\ in Sec. \ref{ssec:c5_runaway_superad}, and ones featuring a rough estimate for including rotation or magnetic fields in Sec. \ref{ssec:c5_rotmag}.
---
> We now consider the simmering of sub-\Mch\ WDs represented by models with increasing complexity and features.  We first consider in Sec. \ref{ssec:runaway_ad} models where the superadiabatic deviation \dnabconv\ is neglected, and the temperature gradient is approximated with $\nabla = \nablaad$. These reproduce all the qualitative features of the runaway, and are good approximations of more complex models.  We then move on to ones that include the superadiabatic temperature deviation \dnabconv\ in Sec. \ref{ssec:runaway_superad}, and ones featuring a rough estimate for including rotation or magnetic fields in Sec. \ref{ssec:rotmag}.
7c7
< \label{ssec:c5_runaway_ad}
---
> \label{ssec:runaway_ad}
10c10
< \label{ssec:c5_runaway_ad_analysis}
---
> \label{ssec:runaway_ad_analysis}
12c12
< Fig. \ref{fig:c5_runaway_rhot} depicts evolutionary tracks of the central density and temperature of simmering WDs with $\nabla = \nablaad$ and masses from $1.0$ to $1.35\,\Msun$.  We refer to these as ``simmering tracks''.  The central specific entropy along each track increases from $\lesssim10^6\,\ergpKg$ to $\gtrsim2.3\times10^8\,\ergpKg$.  Black circles along the tracks show when Eqn. \ref{eq:c5_endofsimmering} is first satisfied, which we refer to as the ``\citeal{wooswk04} point'', and the black dotted ``\citeal{wooswk04}'' line is a power-law fit to them.  Also shown are contours of neutrino cooling and carbon fusion heating timescale (dotted blue and red lines, respectively) and contours of constant entropy (dotted green).  The $\taucc = \taunu$ ignition line denotes where the fusion heating timescale becomes equal to the neutrino cooling one, above which a nuclear runaway occurs, while the $\taucc = \taudyn$ explosion line denotes when the fusion and dynamical (Eqn. \ref{eq:c5_taudyn}) timescales are equal, above which an explosive event occurs.  The $P = 2P(T\mrm{=}0)$ line approximates the upper bound of the region where degeneracy pressure dominates over thermal pressure.
---
> Fig. \ref{fig:runaway_rhot} depicts evolutionary tracks of the central density and temperature of simmering WDs with $\nabla = \nablaad$ and masses from $1.0$ to $1.35\,\Msun$.  We refer to these as ``simmering tracks''.  The central specific entropy along each track increases from $\lesssim10^6\,\ergpKg$ to $\gtrsim2.3\times10^8\,\ergpKg$.  Black circles along the tracks show when Eqn. \ref{eq:endofsimmering} is first satisfied, which we refer to as the ``\citeal{wooswk04} point'', and the black dotted ``\citeal{wooswk04}'' line is a power-law fit to them.  Also shown are contours of neutrino cooling and carbon fusion heating timescale (dotted blue and red lines, respectively) and contours of constant entropy (dotted green).  The $\taucc = \taunu$ ignition line denotes where the fusion heating timescale becomes equal to the neutrino cooling one, above which a nuclear runaway occurs, while the $\taucc = \taudyn$ explosion line denotes when the fusion and and dynamical (Eqn. \ref{eq:taudyn}) timescales are equal, above which an explosive event occurs.  The $P = 2P(T\mrm{=}0)$ line approximates the upper bound of the region where degeneracy pressure dominates over thermal pressure.
20c20
< Due to our choice of \Sc\ range, we produce models that populate the region in Fig. \ref{fig:c5_runaway_rhot} beyond the \citeal{wooswk04} line, where our models' assumption of instantaneous convective energy transport breaks down.  While these sections of the tracks cannot be reached during simmering, we still show them in Fig. \ref{fig:c5_runaway_rhot} as dot-dashed lines to indicate the tracks' overall shape.
---
> Due to our choice of \Sc\ range, we produce models that populate the region in Fig. \ref{fig:runaway_rhot} beyond the \citeal{wooswk04} line, where our models' assumption of instantaneous convective energy transport breaks down.  While these sections of the tracks cannot be reached during simmering, we still show them in Fig. \ref{fig:runaway_rhot} as dot-dashed lines to indicate the tracks' overall shape.
24c24
< \begin{figure}
---
> \begin{figure*}
26,29c26,29
< \includegraphics[angle=0,width=1.0\columnwidth]{chapter5_zhu+16/figures/runaway_rhot.pdf}
< \caption{Evolution of the central temperature and density -- ``simmering tracks'' -- of simmering CO WDs with masses from $1.0 - 1.35\,\Msun$ (labeled in-line).  Solid lines represent tracks of WDs with adiabatic temperature gradients, with dash-dotted track segments indicating regions that cannot be reached during simmering.  Dotted lines represent tracks that include the superadiabatic deviation \dnabconv\ (Eqn. \ref{eq:c5_superad_dev}) required to transport the convective luminosity.  Black circles along adiabatic simmering tracks indicate ``\citeal{wooswk04} points'' where Eqn. \ref{eq:c5_endofsimmering} is first satisfied signaling the end of simmering, and the black dotted \citeal{wooswk04} line represents a power-law fit to them.  Red Xs are \citeal{wooswk04} points for \dnabconv-inclusive tracks.  Also plotted are contours of constant neutrino cooling timescale \taunu\ and carbon fusion heating timescale \taucc, both in years, as well as specific entropy $s$ in \ergpKg.  The $\taucc = \taunu$ and $\taucc = \taudyn$ lines denote where the fusion heating timescale becomes equal to the neutrino cooling timescale and dynamical timescale (Eqn. \ref{eq:c5_taudyn}), respectively.  Finally, the $P = 2P(T\mrm{=}0)$ approximates the upper bound of the region where degeneracy pressure dominates.  Timescale contours were calculated using \mesa\ \citep{paxt+11}.}
< \label{fig:c5_runaway_rhot}
< \end{figure}
---
> \includegraphics[angle=0,width=1.5\columnwidth]{runaway_rhot.pdf}
> \caption{Evolution of the central temperature and density -- ``simmering tracks'' -- of simmering CO WDs with masses from $1.0 - 1.35\,\Msun$ (labelled in-line).  Solid lines represent tracks of WDs with adiabatic temperature gradients, with dash-dotted track segments indicating regions that cannot be reached during simmering.  Dotted lines represent tracks that include the superadiabatic deviation \dnabconv\ (Eqn. \ref{eq:superad_dev}) required to transport the convective luminosity.  Black circles along adiabatic simmering tracks indicate ``\citeal{wooswk04} points'' where Eqn. \ref{eq:endofsimmering} is first satisfied signalling the end of simmering, and the black dotted \citeal{wooswk04} line represents a power-law fit to them.  Red Xs are \citeal{wooswk04} points for \dnabconv-inclusive tracks.  Also plotted are contours of constant neutrino cooling timescale \taunu\ and carbon fusion heating timescale \taucc, both in years, as well as specific entropy $s$ in \ergpKg.  The $\taucc = \taunu$ and $\taucc = \taudyn$ lines denote where the fusion heating timescale becomes equal to the neutrino cooling timescale and dynamical timescale (Eqn. \ref{eq:taudyn}), respectively.  Finally, the $P = 2P(T\mrm{=}0)$ approximates the upper bound of the region where degeneracy pressure dominates.  Timescale contours were calculated using \mesa\ \citep{paxt+11}.}
> \label{fig:runaway_rhot}
> \end{figure*}
31a32
> % the 7.8e8 K from WWK04 is from Sec 4.3 Eqn. 45; T = 8.6 is the temperature of the BLOB that runs away, not the environment!  Same with convective velocity being 80 km/s - this is from just underneath Eqn. 37
33c34
< The simmering tracks roughly form a homology parameterized by a track's highest temperature and a density-axis stretch factor.  Tracks for more massive stars reach higher temperatures and are more horizontally stretched -- the latter is due to entropy being a steeper function of density than degeneracy when $\rho\gtrsim10^8\,\gcc$ and $T\lesssim10^8\,\mrm{K}$.  The $\taucc = \taunu$ and \citeal{wooswk04} lines, though, reach lower temperatures at higher densities.  As a result, a 1.0 \Msun\ WD is already significantly less degenerate than a 1.35 \Msun\ one at the start of simmering.  By the point where the 1.0 \Msun\ WD reaches its maximum temperature of $9.5\times10^8\,\mrm{K}$ (well short of the \citeal{wooswk04} line), it has expanded considerably, its central density dropping to $\rhoc = 5.6\times10^6\,\gcc$, a quarter of its value at the onset of simmering.  It subsequently continues to expand while cooling.  A 1.35 \Msun\ WD, on the other hand, expands much less drastically -- its central density has decreased by 35\% by its \citeal{wooswk04} point at $\rhoc = 7.6\times10^8\,\gcc$, $\Tc = 9.2\times10^8\,\mrm{K}$.  This central temperature is comparable to the $\Tc = 7.8\times10^8\,\mrm{K}$ found for a \Mch\ WD by \citeal{wooswk04}, and the convective velocity we find at the top of the nuclear burning region (where $r = \Rcc$), $\vconvrcc = 8\times10^6\,\cmpsec$ ($\sim1$\% of the sound speed $c_s$) is the same value as estimated in \citeal{wooswk04} for their \Mch\ WD.  This value tends to be higher for lower-mass WDs: $\vconvrcc = 1.4\times10^7\,\cmpsec$ ($\sim3$\% of $c_s$) for a 1.15 \Msun\ WD, in agreement with our estimate of Eqn. \ref{eq:c5_vconvest2}.
---
> The simmering tracks roughly form a homology parameterized by a track's highest temperature and a density-axis stretch factor.  Tracks for more massive stars reach higher temperatures and are more horizontally stretched -- the latter is due to entropy being a steeper function of density than degeneracy when $\rho\gtrsim10^8\,\gcc$ and $T\lesssim10^8\,\mrm{K}$.  The $\taucc = \taunu$ and \citeal{wooswk04} lines, though, reach lower temperatures at higher densities.  As a result, a 1.0 \Msun\ WD is already significantly less degenerate than a 1.35 \Msun\ one at the start of simmering.  By the point where the 1.0 \Msun\ WD reaches its maximum temperature of $9.5\times10^8\,\mrm{K}$ (well short of the \citeal{wooswk04} line), it has expanded considerably, its central density dropping to $\rhoc = 5.6\times10^6\,\gcc$, a quarter of its value at the onset of simmering.  It subsequently continues to expand while cooling.  A 1.35 \Msun\ WD, on the other hand, expands much less drastically -- its central density has decreased by 35\% by its \citeal{wooswk04} point at $\rhoc = 7.6\times10^8\,\gcc$, $\Tc = 9.2\times10^8\,\mrm{K}$.  This central temperature is comparable to the $\Tc = 7.8\times10^8\,\mrm{K}$ found for a \Mch\ WD by \citeal{wooswk04}, and the convective velocity we find at the top of the nuclear burning region, $\vconvrcc = 8\times10^6\,\cmpsec$ ($\sim1$\% of the sound speed $c_s$) is the same value as estimated in \citeal{wooswk04} for their \Mch\ WD.  This value tends to be higher for lower-mass WDs: $\vconvrcc = 1.4\times10^7\,\cmpsec$ ($\sim3$\% of $c_s$) for a 1.15 \Msun\ WD, in agreement with our estimate of Eqn. \ref{eq:vconvest2}.
35c36
< The well-ordered nature of the simmering tracks extends to the \citeal{wooswk04} points, which is why they are well-represented by the \citeal{wooswk04} line.  The line falls just short of the $\taucc = \taudyn$ one, lying just underneath the $\taucc = 10^{-6}\,\mrm{yr}$ contour.  Unlike the other contours in Fig. \ref{fig:c5_runaway_rhot}, this line cannot be calculated independently of the assumptions of the runaway, though we find it is a good approximation for all of our models except for some in Sec. \ref{ssec:c5_rotmag}.
---
> The well-ordered nature of the simmering tracks extends to the \citeal{wooswk04} points, which is why they are well-represented by the \citeal{wooswk04} line.  The line falls just short of the $\taucc = \taudyn$ one, lying just underneath the $\taucc = 10^{-6}\,\mrm{yr}$ contour.  Unlike the other contours in Fig. \ref{fig:runaway_rhot}, this line cannot be calculated independently of the assumptions of the runaway, though we find it is a good approxiation for all of our models except for some in Sec. \ref{ssec:rotmag}.
38c39
< \label{sssec:c5_mcritest_adiabatic}
---
> \label{sssec:mcritest_adiabatic}
40c41
< We turn to the central task of this paper: estimating the minimum mass \Mcrit\ required to reach the \citeal{wooswk04} point, and the corresponding mass of radioactive nickel \MNi\ produced if an explosion occurs shortly thereafter.  To find \Mcrit, we generated models spaced apart by $0.005\,\Msun$, and find $\Mcrit=1.145\,\Msun$ (black simmering track in Fig. \ref{fig:c5_runaway_rhot}), which ends its simmering with $\rhoc = 2.0\times10^7\,\gcc$, $\Tc = 1.2\times10^9\,\mrm{K}$.
---
> We turn to the central task of this paper: estimating the minimum mass \Mcrit\ required to reach the \citeal{wooswk04} point, and the corresponding mass of radioactive nickel \MNi\ produced if an explosion occurs shortly thereafter.  To find \Mcrit, we generated models spaced apart by $0.005\,\Msun$, and find $\Mcrit=1.145\,\Msun$ (black simmering track in Fig. \ref{fig:runaway_rhot}), which ends its simmering with $\rhoc = 2.0\times10^7\,\gcc$, $\Tc = 1.2\times10^9\,\mrm{K}$.
42c43
< While an explosive event becomes inevitable once the simmering phase ends, its nature -- be it a deflagration, detonation, or some other phenomenon -- has yet to be constrained and is beyond the scope of this work.  We are, however, motivated by the resemblance of pure detonations of sub-\Mch\ WDs to SNe Ia to make a rough estimate of the mass of \Ni, \MNi, produced if the \Mcrit\ WD detonated immediately after simmering ends (i.e. without any further changes to its density structure).  Since a detonation is supersonic, and the input energy for nuclear burning is provided by the shock itself (eg. \citealt{seit+09}), nucleosynthesis in a pure detonation is, to first order, determined by the density profile of the progenitor before the explosion.  Indeed, from the results of \cite{sim+10}, we find that \MNi\ can be estimated to within a few percent by the mass of progenitor material with density $\rho>10^7\,\gcc$, $M(\rho>10^7)$ (see Fig. \ref{fig:c5_mni}).  We can use this simple relationship to estimate that for $\Mcrit = 1.145\,\Msun$, $\MNi = 0.30\,\Msun$.  
---
> While an explosive event becomes inevitable once the simmering phase ends, its nature -- be it a deflagration, detonation, or some other phenomenon -- has yet to be constrained and is beyond the scope of this work.  We are, however, motivated by the resemblance of pure detonations of sub-\Mch\ WDs to SNe Ia to make a rough estimate of the mass of \Ni, \MNi, produced if the \Mcrit\ WD detonated immediately after simmering ends (i.e. without any further changes to its density structure).  Since a detonation is supersonic, and the input energy for nuclear burning is provided by the shock itself (eg. \citealt{seit+09}), nucleosynthesis in a pure detonation is, to first order, determined by the density profile of the progenitor before the explosion.  Indeed, from the results of \cite{sim+10}, we find that \MNi\ can be estimated to within a few percent by the mass of progenitor material with density $\rho>10^7\,\gcc$, $M(\rho>10^7)$ (see Fig. \ref{fig:mni}).  We can use this simple relationship to estimate that for $\Mcrit = 1.145\,\Msun$, $\MNi = 0.30\,\Msun$.  
49c50
< \label{sssec:c5_runaway_ad_hot}
---
> \label{sssec:runaway_ad_hot}
63c64
< Fig. \ref{fig:c2_deltamcomp} suggests mergers of WDs with similar mass lead to remnants that are heated throughout, with temperatures between $\sim 1 - 3\times10^8\,\mrm{K}$.  To roughly gauge what effect this pre-runaway heating might have, we generate models identical to the ones above, but set $\tempiso = 2\times10^8\,\mrm{K}$.  We find the simmering tracks of these ``hot-envelope'' WDs deviate most widely from their cold counterparts at the start of simmering, where their central densities are lower by $\sim3-7$\%; these differences reduce to $\sim1 - 5$\% at the end of simmering.  By then, almost the entire interior structure of an adiabatic WD, with \Tc\ $\gtrsim10^9\,\mrm{K}$, has $T > 2\times10^8\,\mrm{K}$, making its central properties insensitive to an increase in \tempiso.  Raising \tempiso\ even further may affect simmering track values more substantially, but hydrostatic solutions of WDs with $\tempiso \gtrsim 5\times10^8\,\mrm{K}$ tend to have low-density atmospheres that extend to arbitrary radii, producing objects of infinite mass.  Any WDs that were heated to such high temperatures by prior evolution, such as the remnant of \cite{ji+13} post-viscous evolution, have more complicated temperature structures that, for simplicity, will not be considered in this work.  Instead, they are explored in a companion paper (Heringer et al. in preparation).
---
> Fig. 3 of \citeal{zhu+13} suggests mergers of WDs with similar mass lead to remnants that are heated throughout, with temperatures between $\sim 1 - 3\times10^8\,\mrm{K}$.  To roughly gauge what effect this pre-runaway heating might have, we generate models identical to the ones above, but set $\tempiso = 2\times10^8\,\mrm{K}$.  We find the simmering tracks of these ``hot-envelope'' WDs deviate most widely from their cold counterparts at the start of simmering, where their central densities are lower by $\sim3-7$\%; these differences reduce to $\sim1 - 5$\% at the end of simmering.  By then, almost the entire interior structure of an adiabatic WD, with \Tc\ $\gtrsim10^9\,\mrm{K}$, has $T > 2\times10^8\,\mrm{K}$, making its central properties insensitive to an increase in \tempiso.  Raising \tempiso\ even further may affect simmering track values more substantially, but hydrostatic solutions of WDs with $\tempiso \gtrsim 5\times10^8\,\mrm{K}$ tend to have low-density atmospheres that extend to arbitrary radii, producing objects of infinite mass.  Any WDs that were heated to such high temperatures by prior evolution, such as the remnant of \cite{ji+13} post-viscous evolution, have more complicated temperature structures that, for simplicity, will not be considered in this work.  Instead, they are explored in a companion paper (Heringer et al. in preparation).
65c66
< %The $\sim5$\% ceiling comes from the 1.15 \Msun\ and 1.2 \Msun\ simmering tracks, whose simmering tracks run nearly parallel to the \citeal{wooswk04} line at the end of simmering; the actual shift in \rhoc\ for a 1.15 \Msun\ WD with $\Tc = 1.3\times10^9\,\mrm{K}$ when switching to a hot envelope is $\sim 3$\%.  
---
> %The $\sim5$\% ceiling comes from the 1.15 \Msun\ and 1.2 \Msun\ simmering tracks, whose simmering tracks run nearly parallel to the \citeal{wooswk04} line at the end of simmering; the acutal shift in \rhoc\ for a 1.15 \Msun\ WD with $\Tc = 1.3\times10^9\,\mrm{K}$ when switching to a hot envelope is $\sim 3$\%.  
68c69
< \label{ssec:c5_runaway_superad}
---
> \label{ssec:runaway_superad}
70c71
< To more accurately calculate simmering tracks, we must include the superadiabatic temperature deviation \dnabconv\ (Eqn. \ref{eq:c5_superad_dev}) needed to carry the convective luminosity.  These \dnabconv-inclusive tracks are plotted in Fig. \ref{fig:c5_runaway_rhot} as dotted lines (without indicating track segments unreachable during simmering).  They are, regardless of mass, nearly identical to their adiabatic counterparts during simmering: their \rhoc\ and \Tc\ at the start of simmering match to within floating point precision, and, for WDs of $M > 1.2\Msun$, they also differ by less than 2\% at the end.  While the \dnabconv\ tracks do steepen and arc away from their adiabatic counterparts, this occurs only above their \citeal{wooswk04} points, which are represented by red Xs in Fig. \ref{fig:c5_runaway_rhot} and are well-approximated by the \textit{adiabatic} \citeal{wooswk04} line.  Close to \Mcrit, where the simmering tracks run nearly parallel to the \citeal{wooswk04} line, \dnabconv\ is more influential: the density of the 1.15 \Msun\ WD \dnabconv\ track at its \citeal{wooswk04} point is $\sim15$\% higher than the adiabatic value.  A mass parameter space search finds $\Mcrit=1.135\,\Msun$, which ends its simmering phase with $\rhoc = 1.7\times10^7\,\gcc$, $\Tc = 1.2\times10^9\,\mrm{K}$.  While these values are very close to the ones obtained in Sec. \ref{sssec:c5_mcritest_adiabatic}, $\MNi = M(\rho>10^7) = 0.20\,\Msun$, which is substantially lower, reflecting its sensitivity to the end of simmering criterion.
---
> To more accurately calculate simmering tracks, we must include the superadiabatic temperature deviation \dnabconv\ (Eqn. \ref{eq:superad_dev}) needed to carry the convective luminosity.  These \dnabconv-inclusive tracks are plotted in Fig. \ref{fig:runaway_rhot} as dotted lines (without indicating track segments unreachable during simmering).  They are, regardless of mass, nearly identical to their adiabatic counterparts during simmering: their \rhoc\ and \Tc\ at the start of simmering match to within floating point precision, and, for WDs of $M > 1.2\Msun$, they also differ by less than 2\% at the end.  While the \dnabconv\ tracks do steepen and arc away from their adiabatic counterparts, this occurs only above their \citeal{wooswk04} points, which are represented by red Xs in Fig. \ref{fig:runaway_rhot} and are well-approximated by the \textit{adiabatic} \citeal{wooswk04} line.  Close to \Mcrit, where the simmering tracks run nearly parallel to the \citeal{wooswk04} line, \dnabconv\ is more influential: the density of the 1.15 \Msun\ WD \dnabconv\ track at its \citeal{wooswk04} point is $\sim15$\% higher than the adiabatic value.  A mass parameter space search finds $\Mcrit=1.135\,\Msun$, which ends its simmering phase with $\rhoc = 1.7\times10^7\,\gcc$, $\Tc = 1.2\times10^9\,\mrm{K}$.  While these values are very close to the ones obtained in Sec. \ref{sssec:mcritest_adiabatic}, $\MNi = M(\rho>10^7) = 0.20\,\Msun$, which is substantially lower, reflecting its sensitivity to the end of simmering criterion.
74c75
< The overall tiny effect of \dnabconv\ is due to Eqn. \ref{eq:c5_superad_dev}'s dependence on the square of the ratio of convective velocity to sound speed, $\vconv^2/(g H_P) \approx \vconv^2/c_s^2$.  Like in the adiabatic case, near the end of simmering $(\vconvrcc/c_s(\Rcc))^2 \approx 1\times10^{-3}$ ($\vconvrcc = 1.3\times10^7\,\cmpsec$) for a $1.15\,\Msun$ star, and $1\times10^{-4}$ for a $1.35\,\Msun$ one.  This small number is partly offset by the $1/\delta = -d\ln T/d\ln \rho$ term, which approaches infinity for zero-temperature degenerate material.  Near the $\taucc = \taunu$ line, however, the entropy is already sufficiently high that $1/\delta \sim 30$ for a $1.15\,\Msun$ WD, and $\sim 300$ for a $1.35\,\Msun$ one; these values fall to $\sim 10$ and $\sim 100$, respectively, near the \citeal{wooswk04} line.  Consequently, $\dnabconv \sim 10^{-2}$, an order of magnitude smaller than $\nablaad \approx 0.3 - 0.4$.  Once the \citeal{wooswk04} point is reached, the influence of \dnabconv\ grows to beyond unity at only slightly higher \Sc\ due to the steep dependence of $\vconv^2$ on temperature, resulting in the sharp upward turn in all simmering tracks beyond that of $1.2\,\Msun$ mentioned previously.
---
> The overall tiny effect of \dnabconv\ is due to Eqn. \ref{eq:superad_dev}'s dependence on the square of the ratio of convective velocity to sound speed, $\vconv^2/(g H_P) \approx \vconv^2/c_s^2$.  Like in the adiabatic case, near the end of simmering $(\vconvrcc/c_s(\Rcc))^2 \approx 1\times10^{-3}$ ($\vconvrcc = 1.3\times10^7\,\cmpsec$) for a $1.15\,\Msun$ star, and $1\times10^{-4}$ for a $1.35\,\Msun$ one.  This small number is partly offset by the $1/\delta = -d\ln T/d\ln \rho$ term, which approaches infinity for zero-temperature degenerate material.  Near the $\taucc = \taunu$ line, however, the entropy is already sufficiently high that $1/\delta \sim 30$ for a $1.15\,\Msun$ WD, and $\sim 300$ for a $1.35\,\Msun$ one; these values fall to $\sim 10$ and $\sim 100$, respectively, near the \citeal{wooswk04} line.  Consequently, $\dnabconv \sim 10^{-2}$, an order of magnitude smaller than $\nablaad \approx 0.3 - 0.4$.  Once the \citeal{wooswk04} point is reached, the influence of \dnabconv\ grows to beyond unity at only slightly higher \Sc\ due to the steep dependence of $\vconv^2$ on temperature, resulting in the sharp upward turn in all simmering tracks beyond that of $1.2\,\Msun$ mentioned previously.
77c78
< \label{ssec:c5_rotmag}
---
> \label{ssec:rotmag}
87c88
< \label{eq:c5_dnabrot_est_work}
---
> \label{eq:dnabrot_est_work}
94c95
< \label{eq:c5_dnabrot_est}
---
> \label{eq:dnabrot_est}
97c98
< \noindent where we again use $g H_P \approx c_s^2$.  Eqn. \ref{eq:c5_dnabrot_est} resembles Eqn. \ref{eq:c5_superad_dev}, with one power of $\vconv/c_s$ swapped for $2\Omega H_P/c_s \approx (\Omega^2 H_P/g)^{1/2}$, which is at most $\sim1$ for rotation at break-up.  Calculations of remnant viscous evolution \citep{shen+12, schw+12, ji+13} suggest the remnant spins down to well below critical rotation, however, and so $(\Omega^2 H_P/g)^{1/2}$ is more realistically $\lesssim10^{-1}$.  Then, for a $1.15\,\Msun$ WD near the end of simmering, $\dnabrot \lesssim (10^{-1}/\delta)(\vconv/c_s) \sim 10^{-1.5}$, an order of magnitude smaller than \nablaad; thus rotational convective suppression is a minor effect.  
---
> \noindent where we again use $g H_P \approx c_s^2$.  Eqn. \ref{eq:dnabrot_est} resembles Eqn. \ref{eq:superad_dev}, with one power of $\vconv/c_s$ swapped for $2\Omega H_P/c_s \approx (\Omega^2 H_P/g)^{1/2}$, which is at most $\sim1$ for rotation at break-up.  Calculations of remnant viscous evolution \citep{shen+12, schw+12, ji+13} suggest the remnant spins down to well below critical rotation, however, and so $(\Omega^2 H_P/g)^{1/2}$ is more realistically $\lesssim10^{-1}$.  Then, for a $1.15\,\Msun$ WD near the end of simmering, $\dnabrot \lesssim (10^{-1}/\delta)(\vconv/c_s) \sim 10^{-1.5}$, an order of magnitude smaller than \nablaad; thus rotational convective suppression is a minor effect.  
103c104
< \label{eq:c5_dnabmag_est_work}
---
> \label{eq:dnabmag_est_work}
110c111
< \label{eq:c5_dnabmag_est}
---
> \label{eq:dnabmag_est}
113c114
< \noindent where $\rho g = P/H_P$ (Eqns. \ref{eq:c5_scaleheight} and \ref{eq:c5_hydroeq}).  Calculations of magnetic field amplification in mergers (Ch. \ref{ch:ch4}) or during viscous evolution \citep{ji+13} suggest a saturation field strength of $\sim10^{11}\,\mrm{G}$ at the center of the remnant.  Given that $P \gtrsim 10^{24}\,\mrm{dyn\,cm}^{-2}$ at the centers of $\gtrsim 1.1\,\Msun$ WDs, this gives $B^2/4\pi P \lesssim 10^{-3}$, and $\dnabmag \lesssim 10^{-3}/\delta$.  Near the $\taucc = \taunu$ line, this value can be quite large, but near the end of simmering it becomes minor (eg. $\sim10^{-2}$ for a $1.15\,\Msun$ WD).
---
> \noindent where $\rho g = P/H_P$ (Eqns. \ref{eq:scaleheight} and \ref{eq:hydroeq}).  Calculations of magnetic field amplification in mergers \citep{zhu+15} or during viscous evolution \citep{ji+13} suggest a saturation field strength of $\sim10^{11}\,\mrm{G}$ at the center of the remnant.  Given that $P \gtrsim 10^{24}\,\mrm{dyn\,cm}^{-2}$ at the centres of $\gtrsim 1.1\,\Msun$ WDs, this gives $B^2/4\pi P \lesssim 10^{-3}$, and $\dnabmag \lesssim 10^{-3}/\delta$.  Near the $\taucc = \taunu$ line, this value can be quite large, but near the end of simmering it becomes minor (eg. $\sim10^{-2}$ for a $1.15\,\Msun$ WD).
115c116
< The above suggest that for reasonable rotation rates and magnetic field strengths, their effects on the simmering phase should be small.  To confirm this, we implemented the formulation of \citeauthor{stev79} (\citeyear{stev79}; hereafter \citeal{stev79}).  \citeal{stev79} incorporates rotation and externally imposed magnetic fields (both assumed, as above, to be slowly varying over $H_P$) into Rayleigh-B\'{e}nard convection.  It predicts the convective steady state -- in particular both \deltanab\ and the modified convective velocity -- by finding the growth rates of convective modes using linear stability analysis (\citeal{stev79} Sec. 2), and then equating them to their non-linear cascade rates, picking the mode with the greatest heat flux to represent the motion as a whole.  While this is \textit{ad hoc}, in the rotation-dominated and unmagnetized limit \citeal{stev79}'s theory reproduces well the convection simulations of \cite{barkdl14} over a wide range of rotation rates.  We summarize our results below; further detail can be found in Sec. \ref{sec:c5_suppression}.
---
> The above suggest that for reasonable rotation rates and magnetic field strengths, their effects on the simmering phase should be small.  To confirm this, we implemented the formulation of \citeauthor{stev79} (\citeyear{stev79}; hereafter \citeal{stev79}).  \citeal{stev79} incorporates rotation and externally imposed magnetic fields (both assumed, as above, to be slowly varying over $H_P$) into Rayleigh-B\'{e}nard convection.  It predicts the convective steady state -- in particular both \deltanab\ and the modified convective velocity -- by finding the growth rates of convective modes using linear stability analysis (\citeal{stev79} Sec. 2), and then equating them to their non-linear cascade rates, picking the mode with the greatest heat flux to represent the motion as a whole.  While this is \textit{ad hoc}, in the rotation-dominanted and unmagnetized limit \citeal{stev79}'s theory reproduces well the convection simulations of \cite{barkdl14} over a wide range of rotation rates.  We summarize our results below; further detail can be found in \cite{zhu16thesis}.
117c118
< For WDs with sub-critical uniform rotation, we confirm that the effect of rotation on simmering is small.  During simmering, \vconv\ increases by orders of magnitude, while $\Omega$ decreases by half an order of magnitude as the WD expands (to conserve angular momentum).  This leads $2\Omega H_P/\vconv$ to approach unity, and \dnabrot\ to approach \dnabconv, near the end of simmering for WDs with initial angular speed $\Ominit$ at about a quarter of break up speed \Omcrit.  We also find that $\vconv \propto (\dnabrot/\dnabconv)^{-1/4}$ changes very little, justifying our use of its non-rotating value in the approximation above. For moderate values of $\Omega$, then, simmering tracks shift by only a few percent in density and temperature from their non-rotating versions in Sec. \ref{ssec:c5_runaway_superad}.  For simmering when $\Ominit \rightarrow \Omcrit$, Eqn. \ref{eq:c5_dnabrot_est} suggests \dnabrot\ becomes comparable to \nablaad, but our models show that this is overshadowed by the centrifugal pressure support term in Eqn. \ref{eq:c5_hydroeq}, and the net effect is a slight reduction of \rhoc\ and \Tc\ for a model with a given \Sc.  This effect increases $\Mcrit$ to $1.14\,\Msun$ for WDs with $\Omega/\Omcrit = 0.5$, though we caution that deviations from spherical symmetry not included in our model become significant near critical rotation.
---
> For WDs with sub-critical uniform rotation, we confirm that the effect of rotation on simmering is small.  During simmering, \vconv\ increases by orders of magnitude, while $\Omega$ decreases by half an order of magnitude as the WD expands (to conserve angular momentum).  This leads $2\Omega H_P/\vconv$ to approach unity, and \dnabrot\ to approach \dnabconv, near the end of simmering for WDs with initial angular speed $\Ominit$ at about a quarter of break up speed \Omcrit.  We also find that $\vconv \propto (\dnabrot/\dnabconv)^{-1/4}$ changes very little, justifying our use of its non-rotating value in the approximation above. For moderate values of $\Omega$, then, simmering tracks shift by only a few percent in density and temperature from their non-rotating versions in Sec. \ref{ssec:runaway_superad}.  For simmering when $\Ominit \rightarrow \Omcrit$, Eqn. \ref{eq:dnabrot_est} suggests \dnabrot\ becomes comparable to \nablaad, but our models show that this is overshadowed by the centrifugal pressure support term in Eqn. \ref{eq:hydroeq}, and the net effect is a slight reduction of \rhoc\ and \Tc\ for a model with a given \Sc.  This effect increases $\Mcrit$ to $1.14\,\Msun$ for WDs with $\Omega/\Omcrit = 0.5$, though we caution that deviations from spherical symmetry not included in our model become significant near critical rotation.
121c122
< Likewise, we find that simmering tracks for non-rotating WDs with $M\approx\Mcrit$ shift by only a few percent when including magnetic fields as high as $10^{12}\,\mrm{G}$, consistent with Eqn. \ref{eq:c5_dnabmag_est}.  When performing a parameter-space search for \Mcrit, we alter field strength with mass by keeping the ratio \EBEtot\ fixed.  For $\EBEtot = 2.9\times10^{-5}$, which corresponds to $B(r=0) = 1\times10^{11}\,\mrm{G}$ in a $1.15\,\Msun$ WD at the start of simmering, $\Mcrit = 1.13\,\Msun$.  If we consider field strengths much larger than what is expected for merger remnants, however, we find that, while the simmering track still changes negligibly, the convective velocity \vconv\ becomes proportional to $1/B$ and is dramatically reduced, affecting when Eqn. \ref{eq:c5_endofsimmering} is first satisfied.  A $1.15\,\Msun$ WD threaded by a $10^{12}\,\mrm{G}$ field sees its convective velocity reduced by a factor $\sim 10 - 100$, and reaches the \citeal{wooswk04} point at $T = 8\times10^8\,\mrm{K}$, far lower than the (adiabatic) \citeal{wooswk04} line in Fig. \ref{fig:c5_runaway_rhot}.  We determine \Mcrit\ for WDs with $\EBEtot = 2.8\times10^{-3}$, which corresponds to $B(r=0) = 1\times10^{12}\,\mrm{G}$ in a $1.15\,\Msun$ WD at the start of simmering, to be $1.02\,\Msun$, a reduction of more than $0.1\,\Msun$ from the non-magnetized value.
---
> Likewise, we find that simmering tracks for non-rotating WDs with $M\approx\Mcrit$ shift by only a few percent when including magnetic fields as high as $10^{12}\,\mrm{G}$, consistent with Eqn. \ref{eq:dnabmag_est}.  When performing a parameter-space search for \Mcrit, we alter field strength with mass by keeping the ratio \EBEtot\ fixed.  For $\EBEtot = 2.9\times10^{-5}$, which corresponds to $B(r=0) = 1\times10^{11}\,\mrm{G}$ in a $1.15\,\Msun$ WD at the start of simmering, $\Mcrit = 1.13\,\Msun$.  If we consider field strengths much larger than what is expected for merger remnants, however, we find that, while the simmering track still changes negligibly, the convective velocity \vconv\ becomes proportional to $1/B$ and is dramatically reduced, affecting when Eqn. \ref{eq:endofsimmering} is first satisfied.  A $1.15\,\Msun$ WD threaded by a $10^{12}\,\mrm{G}$ field sees its convective velocity reduced by a factor $\sim 10 - 100$, and reaches the \citeal{wooswk04} point at $T = 8\times10^8\,\mrm{K}$, far lower than the (adiabatic) \citeal{wooswk04} line in Fig. \ref{fig:runaway_rhot}.  We determine \Mcrit\ for WDs with $\EBEtot = 2.8\times10^{-3}$, which corresponds to $B(r=0) = 1\times10^{12}\,\mrm{G}$ in a $1.15\,\Msun$ WD at the start of simmering, to be $1.02\,\Msun$, a reduction of more than $0.1\,\Msun$ from the non-magnetized value.
123c124,126
< At this strong-field limit, however, the \citeal{wooswk04} point is also well-short of the $\taucc = \taudyn$ line, meaning that a WD that reaches the point must continue to heat up before it can explode.  This heating may eventually lead to an extremely steep temperature gradient that allows for rapid convective energy transport before dynamical burning is reached, but this is beyond the ability for our model to follow.  Moreover, we strongly caution that \citeal{stev79}'s magnetic formulation may not accurately reflect non-linear magnetoconvection, except perhaps in the case of weak fields, and does not include magnetic dynamo processes, which are likely to be efficient in simmering WDs.  We discuss these further in Sec. \ref{ssec:c5_magaccuracy}.
---
> %unlike in all the other cases that we consider
> 
> At this strong-field limit, however, the \citeal{wooswk04} point is also well-short of the $\taucc = \taudyn$ line, meaning that a WD that reaches the point must continue to heat up before it can explode.  This heating may eventually lead to an extremely steep temperature gradient that allows for rapid convective energy transport before dynamical burning is reached, but this is beyond the ability for our model to follow.  Moreover, we strongly caution that \citeal{stev79}'s magnetic formulation may not accurately reflect non-linear magnetoconvection, except perhaps in the case of weak fields, and does not include magnetic dynamo processes, which are likely to be efficient in simmering WDs.  We discuss these further in Sec. \ref{ssec:magaccuracy}.
