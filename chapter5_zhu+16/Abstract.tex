\begin{abstract}
%Context
When fusion is lit in the degenerate interior of a carbon-oxygen white dwarf, the resulting nuclear runaway starts with a ``simmering phase'', in which convection transports energy out of the burning region. 
%Aims
While simmering inevitably leads to some form of explosion for white dwarfs near the Chandrasekhar mass, in ones with lower mass it may instead lead to the lifting of degeneracy and expansion into a carbon-burning star.
%Methods/Results
Using analytical arguments and simple models, we determine that the critical mass for explosions to be possible is $\Mcrit = 1.135\,\Msun$.  In principle, effects from rotation and magnetic fields might lead to a change in the critical mass.  For the rotation rates found in merger remnant simulations, the effect is almost certainly minimal, $\lesssim0.01\,\Msun$.  For magnetic fields the case is less clear, since interaction with convection is poorly understood, but simple order-of-magnitude arguments again suggests only a small effect.

\textit{\textbf{Key words:} white dwarfs -- supernovae: general}
\end{abstract}

%Its associated central density during simmering, $\sim3\times10^7\,\gcc$, is relatively high compared to those found in dense cores found in simulations of white dwarf merger remnants, suggesting that those may not generically lead to explosions.  
