2c2
< \label{ssec:c5_conclusions}
---
> \label{ssec:conclusions}
4c4
< We investigated using simple estimates the outcome of simmering for sub-\Mch\ WDs, and find that the minimum mass \Mcrit\ that achieves dynamical burning and explodes, rather than expanding and cooling, is $1.135\,\Msun$.  We also estimate that including rotation or $\lesssim 10^{11}\,\mrm{G}$ magnetic fields alters this value by less than $\sim0.01\,\Msun$ and $\sim0.02\,\Msun$, respectively.  Stronger, $\sim10^{12}\,\mrm{G}$ fields, which may be generated through convective dynamo processes, could affect simmering much more substantially, but are beyond the scope of our models.  Examining merger remnants in Ch. \ref{ch:ch2}, and the simulation of \cite{ji+13}, we estimate that the majority of sub-\Mch\ post-viscous remnants are too underdense to remain degenerate until dynamical burning if simmering were to set in.  Even if they could explode, they would still produce very little \Ni.  This presents an issue for to the viability of the \citeal{vkercj10} channel, as it suggests most sub-\Mch\ remnants do not explode as SNe Ia.
---
> We investigated using simple estimates the outcome of simmering for sub-\Mch\ WDs, and find that the minimum mass \Mcrit\ that achieves dynamical burning and explodes, rather than expanding and cooling, is $1.135\,\Msun$.  We also estimate that including rotation or $\lesssim 10^{11}\,\mrm{G}$ magnetic fields alters this value by less than $\sim0.01\,\Msun$ and $\sim0.02\,\Msun$, respectively.  Stronger, $\sim10^{12}\,\mrm{G}$ fields, which may be generated through convective dynamo processes, could affect simmering much more substantially, but are beyond the scope of our models.  Examining merger remnants in \citeal{zhu+13}, and the simulation of \cite{ji+13}, we estimate that the majority of sub-\Mch\ post-viscous remnants are too underdense to remain degenerate until dynamical burning if simmering were to set in.  Even if they could explode, they would still produce very little \Ni.  This presents an issue for to the viability of the \citeal{vkercj10} channel, as it suggests most sub-\Mch\ remnants do not explode as SNe Ia.
12c12
< We have also estimated the \Mtot-\MNi\ relationship of simmering sub-\Mch\ WDs that reach explosion and find that they, like many other models, do not correspond to the observed SN Ia \Mtot-\MNi\ relationships of \cite{scalzrs14} and \cite{chil+15}.  In particular, our most massive WDs produce too much \Ni\ to reproduce the bottom right of Fig. \ref{fig:c5_mni}, which would not be the case if the temperature structure at the end of simmering were substantially shallower than in our models.  Such a structure could potentially be produced if the burning were not center-lit, but occurred in an off-center shell that remained degenerate throughout the runaway, and we explore this possibility in the companion paper to this work (Heringer et al. in preparation).
---
> We have also estimated the \Mtot-\MNi\ relationship of simmering sub-\Mch\ WDs that reach explosion and find that they, like many other models, do not correspond to the observed SN Ia \Mtot-\MNi\ relationships of \cite{scalzrs14} and \cite{chil+15}.  In particular, our most massive WDs produce too much \Ni\ to reproduce the bottom right of Fig. \ref{fig:mni}, which would not be the case if the temperature structure at the end of simmering were substantially shallower than in our models.  Such a structure could potentially be produced if the burning were not center-lit, but occurred in an off-center shell that remained degenerate throughout the runaway, and we explore this possibility in the companion paper to this work (Heringer et al. in preparation).
16c16
< We are very grateful to Ken Shen, whose discussion with MHvK on the existence and value of \Mcrit\ years ago served as the original inspiration for this paper, and to Henk Spruit for his insights into magnetoconvection and its implementation in our models.  We also thank Chris Matzner, Yuri Levin and Chris Thompson for helpful comments, and Suoqing Ji and Robert Fisher for both valuable feedback and for sharing the end-state of their viscous evolution simulation.  Our calculations made extensive use of Frank Timmes' Helmholtz EOS and sections of \mesa\ by Bill Paxton and the \mesa\ consortium, and we them for making these codes publicly available.  
---
> We are very grateful to Ken Shen, whose discussion with MHvK on the existence and value of \Mcrit\ years ago served as the original inspiration for this paper, and to Henk Spruit for his insights into magnetoconvection and its implementation in our models.  We also thank Chris Matzner, Yuri Levin and Chris Thompson for helpful comments, and Suoqing Ji and Robert Fisher for both valuable feedback and for sharing the end-state of their viscous evolution simulation.  Our calculations made extensive use of Frank Timmes' Helmholtz EOS and sections of \mesa\ by Bill Paxton and the \mesa\ consortium, and we them for making these codes publically available.  
