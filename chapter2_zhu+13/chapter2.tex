\chapter{A Parameter-Space Study of Carbon-Oxygen White Dwarf Mergers}
\label{ch:ch2}

\begin{center}
\begin{minipage}[c]{4.75in}
Chenchong Zhu, Philip Chang, Marten H. van Kerkwijk and James Wadsley\\
The Astrophysical Journal, Volume 767, Issue 2 - article id. 164, 32 pp., 2013 \citep{zhu+13}
\vspace{2em}
\end{minipage}
\end{center}

As we discussed in Sec. \ref{sec:c1_vkchannel} and \ref{sec:c1_hotdqs}, the merger of two carbon-oxygen white dwarfs can lead either to a spectacular transient, stable nuclear burning or a massive, rapidly rotating white dwarf.  Previous simulations of mergers have shown that the outcome strongly depends on whether the white dwarfs are similar or dissimilar in mass \citep{loreig09}.  In the similar-mass case, both white dwarfs merge fully and the remnant is hot throughout, while in the dissimilar case, the more massive, denser white dwarf remains cold and essentially intact, with the disrupted lower mass one wrapped around it in a hot envelope and disk.

In order to determine what constitutes ``similar in mass'' and more generally how the properties of the merger remnant depend on the input masses, we simulated unsynchronized carbon-oxygen white dwarf mergers for a large range of masses using smoothed-particle hydrodynamics.  We find that the structure of the merger remnant varies smoothly as a function of the ratio of the central densities of the two white dwarfs.  A density ratio of 0.6 approximately separates similar and dissimilar mass mergers.  Confirming previous work, we find that the temperatures of most merger remnants are not high enough to immediately ignite carbon fusion.  During subsequent viscous evolution, however, the interior will likely be compressed and heated as the disk accretes and the remnant spins down.  We find from simple estimates that this evolution can lead to ignition for many remnants.  For similar-mass mergers, this would likely occur under sufficiently degenerate conditions that a thermonuclear runaway would ensue.

Aside from redundant parts of the introduction, we also do not reproduce here the extensive Appendix to \cite{zhu+13}, which contains tables of binary input parameters and remnant properties for the simulations, and its online figure set, depiciting merger remnant properties for all of our simulations.
