\section{Introduction}
\label{sec:c2_intro}

A few percent of all white dwarfs (WDs) will eventually merge with another white dwarf.  The outcome of such mergers will depend on the compositions of the WDs involved.  For two helium WDs, a low-mass helium star might result, which would be observed as an sdOB star.  For a helium WD merging with a carbon-oxygen one, a helium giant could form, observable as a hydrogen-deficient giant or R CrB star.  For two carbon-oxygen WDs (CO WDs), the outcome could vary between simply a more massive WD, a carbon-burning star, an explosion, or collapse to a neutron star, depending on whether stable or unstable carbon fusion is ignited, and whether the total mass exceeds the critical mass for pycnonuclear ignition or electron captures (both close to the Chandrasekhar mass \Mch).  For mergers involving an oxygen-neon WD, the mass will always be high, and explosive demise or transmutation seems inevitable.

The outcome of the merger of two CO WDs is uncertain in part because during the merger temperatures do not become hot enough to ignite significant carbon fusion (e.g., \citealt{loreig09}, \citeal{loreig09} hereafter), except possibly for masses above $\sim\!0.9\,M_\odot$ \citep{pakm+11,pakm+12}.  Hence, the final fate depends on subsequent evolution, in which differential rotation is dissipated, the remnant disk accretes, and the whole remnant possibly spins down.  Due to these processes, the remnant could be compressed and heated, which, if it happens faster than the thermal timescale, would lead to increased temperatures and thus potentially to ignition.  

So far, efforts have focused on merging binaries with total mass $M>\Mch$.  The end result of such mergers is believed to be either stable off-center carbon ignition, which would turn the merger remnant into an oxygen-neon WD and possibly eventually result in accretion-induced collapse \citep{saion98}, or slow accretion, which allows the remnant to stay cool and eventually ignite at high central density \citep{yoonpr07}.  Less massive mergers were usually thought to result in more massive, rapidly rotating CO WDs \citep{segrcm97,kube+10}, but more recently it has been realized these might eventually become hot enough to ignite (\citealt{vkercj10}, \citeal{vkercj10} hereafter; \citealt{shen+12,schw+12}).  Indeed, \citeal{vkercj10} argue that type Ia supernovae result generally from mergers of CO WDs with similar masses, independent of whether or not their total mass exceeds \Mch\ (see below).  For all these studies, the conclusions on whether and where ignition takes place depend critically on the structure of the merger remnnant.

The merging process, and the merger remnant, have been studied quite extensively, mostly using smoothed-particle hydrodynamics (SPH; e.g. \citealt{mona05}).  These simulations have shown that the outcome strongly depends on whether the WDs are similar or dissimilar in mass.  In the similar-mass case, both WDs disrupt fully and the remnant is hot throughout, while in the dissimilar case, the more massive, denser WD remains essentially intact and relatively cold, with the disrupted lower mass one wrapped around it in a hot envelope and disk.  Less clear, however, is what constitutes ``similar-mass,'' and, more generally, how the merger remnant properties depend on the initial conditions.  

In principle, for cold WDs of given composition, the remnant properties should depend mostly on the two WD masses, with a second-order effect due to rotation.  In this paper, we try to determine these dependencies using simulations of WD mergers with the Gasoline SPH code, covering the entire range of possible donor and accretor masses, but limiting ourselves to non-rotating WDs.  Our primary aim is to identify trends between mergers of different masses, both to guide analytical understanding and to help scale other, perhaps more precise simulations.  Here, our hope is that while the results of individual simulations may suffer from uncertainties related to the precise techniques and assumptions used, the trends should be more robust.  We also try to provide sufficient quantitative detail on the properties of merger remnants that it becomes possible to make analytical estimates or construct reasonable numerical approximations without having to run new simulations.

Our work is complementary to the recent surveys of remnant properties by \cite{rask+12} and \cite{dan+12}, in that they focus on different scientific questions (e.g., orbital stability; possible detonation).  In contrast to our work, they assume that the WDs are co-rotating with the orbit.  Whether this is a better assumption than no rotation depends on the strength of tidal dissipation, which unfortunately is not yet known (see \citealt{marsns04,fulll12}).

Our work also is part of a series of numerical studies investigating the viability of sub-Chandrasekhar mass (sub-\Mch) CO WD mergers producing SNe Ia, as proposed by \citeal{vkercj10}.  This channel relies on similar-mass mergers producing remnants that are hottest near the center, and on compressional heating by subsequent accretion and/or magnetically mediated spin-down leading to ignition.  The advantages of this channel are that it accounts for the absence of direct evidence for stellar companions, the observed SN Ia rate, and the dependence of SN Ia peak luminosity on the age of the host stellar population (because lower-mass merger constituents take longer to form).  Since pure detonations of sub-\Mch\ CO WDs produce light curves very similar to observed SNe Ia \citep{shig+92,sim+10}, it also removes the need for imposed deflagration-to-detonation transitions.  Important questions, however, remain, including what fraction of mergers leads to remnants that are hot near the center (in highly degenerate conditions), how the subsequent viscous phase proceeds in detail, whether ignition leads to a detonation, and whether the detonation of a remant that may still rotate and be surrounded by a disk would produce an event similar to an SN Ia.  With our work, we attempt to address the first question.

This paper is organised as follows.  In Section~\ref{sec:gasoline}, we describe the SPH code we used, as well as our initial conditions.  In Section~\ref{sec:results}, we present our results and give trends for a number of pertinent remnant properties.  In Section~\ref{sec:variation}, we test the robustness of our results, and in Section~\ref{sec:compwithothers} compare our results with those of \citeal{loreig09} and others.  Lastly, in Section~\ref{sec:postmerger}, we speculate on the further evolution of our remnants, considering in particular whether, as suggested by \citeal{vkercj10}, some might lead to type Ia supernovae.

