\section{Introduction}
\label{sec:c2_intro}

Until recent years, efforts to find SN Ia progenitors among merging CO WD binaries have focused on those with total mass $M>\Mch$.  The end result of such mergers is believed to be either stable off-center carbon ignition, which would turn the merger remnant into an oxygen-neon WD and possibly eventually result in accretion-induced collapse \citep{saion98}, or slow accretion, which allows the remnant to stay cool and eventually ignite at high central density \citep{yoonpr07}.  Less massive mergers were usually thought to result in more massive, rapidly rotating CO WDs \citep{segrcm97,kube+10}, but more recently it has been realized these might eventually become hot enough to ignite (\citeal{vkercj10}; \citealt{shen+12,schw+12}).  Indeed, \citeal{vkercj10} argue that type Ia supernovae result generally from mergers of CO WDs with similar masses, independent of whether or not their total mass exceeds \Mch.  For all these studies, the conclusions on whether and where ignition takes place depend critically on the structure of the merger remnant.

The merging process, and the merger remnant, have been studied quite extensively, mostly using smoothed-particle hydrodynamics (SPH; e.g. \citealt{mona05}).  These simulations have shown that the outcome strongly depends on whether the WDs are similar or dissimilar in mass.  In the similar-mass case, both WDs disrupt fully and the remnant is hot throughout, while in the dissimilar case, the more massive, denser WD remains essentially intact and relatively cold, with the disrupted lower mass one wrapped around it in a hot envelope and disk.  Less clear, however, is what constitutes ``similar-mass,'' and, more generally, how the merger remnant properties depend on the initial conditions.  

In principle, for cold WDs of given composition, the remnant properties should depend mostly on the two WD masses, with a second-order effect due to rotation.  In this paper, we try to determine these dependencies using simulations of WD mergers with the Gasoline SPH code, covering the entire range of possible donor and accretor masses, but limiting ourselves to non-rotating WDs.  Our primary aim is to identify trends between mergers of different masses, both to guide analytical understanding and to help scale other, perhaps more precise simulations.  Here, our hope is that while the results of individual simulations may suffer from uncertainties related to the precise techniques and assumptions used, the trends should be more robust.  We also try to provide sufficient quantitative detail on the properties of merger remnants that it becomes possible to make analytical estimates or construct reasonable numerical approximations without having to run new simulations.

Our work is complementary to the recent surveys of remnant properties by \cite{rask+12} and \cite{dan+12}, in that they focus on different scientific questions (e.g., orbital stability; possible detonation).  In contrast to our work, they assume that the WDs are co-rotating with the orbit.  Whether this is a better assumption than no rotation depends on the strength of tidal dissipation, which unfortunately is not yet known (see \citealt{marsns04,fulll12}).

This chapter is organised as follows.  In Section~\ref{sec:c2_gasoline}, we describe the SPH code we used, as well as our initial conditions.  In Section~\ref{sec:c2_results}, we present our results and give trends for a number of pertinent remnant properties.  In Section~\ref{sec:c2_variation}, we test the robustness of our results, and in Section~\ref{sec:c2_compwithothers} compare our results with those of \citeal{loreig09} and others.  Lastly, in Section~\ref{sec:c2_postmerger}, we speculate on the further evolution of our remnants, considering in particular whether, as suggested by \citeal{vkercj10}, some might lead to type Ia supernovae.
