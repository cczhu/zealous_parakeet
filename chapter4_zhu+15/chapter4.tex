\chapter{Magnetized Moving Mesh Merger of a Carbon-Oxygen White Dwarf Binary}
\label{ch:ch4}

\begin{center}
\begin{minipage}[c]{4.75in}
Chenchong Zhu, R\"{u}diger Pakmor, Marten H. van Kerkwijk, Philip Chang\\
The Astrophysical Journal Letters, Volume 806, Issue 1 - article id. L1, 5pp., 2015 (\citeal{zhu+15})
\vspace{2em}
\end{minipage}
\end{center}

While simulations of white dwarf mergers are numerous (Sec. \ref{XXX}), to date they have not included magnetic fields, even though they are believed to play a significant role in the evolution of the merger remnant.  We simulated a 0.625 - 0.65 {\Msun} carbon-oxygen WD binary merger in the magnetohydrodynamic moving mesh code {\arepo}.  Each WD was given an initial dipole field with a surface value of $\sim10^3$ G.  As in simulations of merging double neutron star binaries, we find exponential field growth within Kelvin-Helmholtz instability-generated vortices during the coalescence of the two stars.  The final field has complex geometry, and a strength $>10^{10}$ G at the center of the merger remnant.  Its energy is $\sim2\times10^{47}$ ergs, $\sim0.2$\% of the remnant's total energy.  The strong field likely influences further evolution of the merger remnant by providing a mechanism for angular momentum transfer and additional heating, potentially helping to ignite carbon fusion.
