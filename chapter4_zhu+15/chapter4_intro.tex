\section{Introduction}
\label{sec:c4_intro}

The merging process has, in the last decade, been investigated with increasingly sophisticated 3D hydrodynamic simulations \citep{loreig09, pakm+10, dan+12, dan+14, rask+12, zhu+13, moll+14}.  However, one fundamental piece missing in WD merger studies so far is magnetic fields.

Mergers (that do not immediately explode) are expected to produce remnants that are susceptible to magnetic dynamo processes such as the magnetorotational instability (MRI; \citealt{balbh91}), Tayler-Spruit dynamo (e.g. \citealt{spru02}), and the $\alpha\omega$ dynamo (if convection occurs in the inner disk; \citealt{garc+12}).  It has therefore long been suspected that they can generate strong fields, and recent 2D simulations of MRI in the remnant \citep{ji+13} have indeed shown amplification of a weak seed field to $>10^{10}$ G.  Magnetic shear from these fields transports angular momentum over a timescale of $\sim10^4 - 10^8$ s \citep{vkercj10, shen+12} -- far shorter than the thermal timescale of the remnant -- and also (non-locally) heats the remnant.  The latter, combined with loss of rotational support from angular momentum transport, could push remnant temperatures past the point of carbon ignition ($\sim6\times10^8$ K for densities between $10^5 - 10^7$ \gcc), leading to either stable nuclear burning or a runaway.  This mechanism could potentially drive nuclear runaways even in remnants with masses below the Chandrasekhar Mass \Mch\ that have traditionally been considered stable \citep{vkercj10}.

While field growth after the merger has been explored, field growth \textit{during} the merger is also expected, and can have a profound impact on the post-merger magnetic evolution.  Magnetohydrodynamic (MHD) double NS binary merger simulations (eg. \citealt{pricr06, kiuc+14, giac+15}) have found that Kelvin-Helmholtz vortices produced along the shear interface between the coalescing stars can amplify field strengths by orders of magnitude \citep{oberam10, zrakm13}.  The same should hold true for WD mergers.  Motivated by this, we present the first MHD simulation of a CO WD binary merger.

%To put it another way, the differential rotation that powers a post-merger field growth already exists during coalescence, so we should not be surprised that the field grows during the merger as well as after.

This chapter is organized as follows: in Section~\ref{sec:c4_codes}, we describe \arepo\ and our initial conditions.  The results of our simulation are in Section~\ref{sec:c4_results}.  In Section~\ref{sec:c4_robust}, we test our simulation's robustness, and finally in Section~\ref{sec:c4_discussion} we discuss implications for merger outcomes and possible avenues for future research.
